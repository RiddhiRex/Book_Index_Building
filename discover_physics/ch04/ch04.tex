\mychapterwithopener{poolskater}{The pool skater trades two forms of
energy back and forth: kinetic and gravitational. More photos of this
insane pastime are at the web site www.sonic.net/$\sim$shawn. When I first came across it
in 1998, I assumed these guys weren't likely to stay alive for long, but they seem to
have survived --- or at least their web site has.}{Conservation of Mass and Energy}\label{ch:energy}
In chapter \ref{ch:symmetry}, I promised that as you learned more and more about physics,
you would see it becoming more and more simple. The unifying principle that brings order and
sanity to all of physics is Noether's theorem, which so far you've only seen stated in a very
rough form: the laws of physics have to be the way they are because of symmetry. This book's presentation of
physics so far has been suffused with symmetry arguments, but much of what you've learned has
consisted of specific, practical applications, like the formation of images by lenses and mirrors.
What have you learned so far that deserves to be called a fundamental law of physics? The only
law of physics you've learned is the principle of inertia:
a ray of light or a material object continues moving in the same direction and
at the same speed if it is not interacting with anything else.

That's all very well, but the universe would be dull if it consisted only of individual
atoms and rays of light crisscrossing space and never coming close enough to interact with each
other --- it would be like a game of pool played on an infinite table, with only one ball in
sight. Your everyday life, to which we'd like to apply physics, involves vast numbers of
particles. Your own body, for instance, contains something like $10^{30}$ atoms (that's
scientific notation for one followed by thirty zeroes). How can we make sense out of
such incredible complexity?

\mysection[0]{Conservation of Mass}\index{mass!conservation of}\index{conservation!of mass}
What makes our complex world comprehensible to the human mind is that the fundamental
laws of physics are all conservation laws: laws stating that the total amount of something
stays the same. You've already discovered some evidence in lab for such a law: the law of
conservation of mass. Even when you carried out complex operations involving huge numbers
of atoms, the total mass of the atoms never changed. The wonderful thing about
conservation laws is that they allow us to make sense out of complex processes.

The law of
conservation of mass probably didn't surprise you very much, since you've known about
atoms since an early age, and in everyday life we don't encounter processes in which atoms
change their masses noticeably, or in which atoms are created or destroyed. That argument wasn't
obvious to your ancestors, however. 
It's not even hard to think of examples that would raise doubts in the minds of modern people.
A log weighs more than its
ashes. Did some mass simply disappear? It seems to be an exception to the rule.

The French chemist Antoine-Laurent Lavoisier was the first scientist to realize\index{Lavoisier!Antoine-Laurent}
that there were no such exceptions. Lavoisier hypothesized that
when wood burns, for example, the supposed loss of mass is actually accounted for by the
escaping hot gases that the flames are made of.
Before Lavoisier, chemists had almost never
weighed their chemicals to quantify the amount of each substance that was undergoing
reactions.\footnote{Isaac Newton was a notable exception.}
They also didn't completely understand that gases were just another
state of matter, and hadn't tried performing reactions in sealed chambers to determine
whether gases were being consumed from or released into the air. For this they had at least one
practical excuse, which is that if you perform a gas-releasing reaction in a sealed chamber
with no room for expansion, you get an explosion! Lavoisier invented a balance that was capable
of measuring milligram masses, and figured out how to do reactions in an upside-down
bowl in a basin of water, so that the gases could expand by pushing out some of the water.
In one crucial experiment, Lavoisier heated a red mercury compound, which we would now
describe as mercury oxide (HgO), in such a sealed chamber.
A gas was produced (Lavoisier later named
it ``oxygen''), driving out some of the water, and the red compound was transformed into
silvery liquid mercury metal. The crucial point was that the total mass of the entire
apparatus was exactly the same before and after the reaction. Based on many observations
of this type, Lavoisier proposed a general law of nature, that mass is always conserved.

\selfcheck{conservation}{In ordinary speech, we say that you should ``conserve'' something, because if
you don't, pretty soon it will all be gone. How is this different from the meaning of
the term ``conservation'' in physics?}

\margup{-150mm}{\fig{lavoisier}{Portrait of Monsieur Lavoisier and His Wife,
        by Jacques-Louis David, 1788. Lavoisier invented the
        concept of conservation of mass. The husband is depicted with his scientific apparatus,
	while in the background
        on the left is the portfolio belonging to Madame Lavoisier, who is thought to have
        been a student of David's.  }
}
Although Lavoisier was an honest and energetic public official, he was caught up
in the Terror and sentenced to death in 1794. 
He requested a fifteen-day
delay of his execution so that he could complete some experiments that he thought might
be of value to the Republic. 
The judge, Coffinhal, infamously replied that ``the state has no need of scientists.''
As a scientific
experiment, Lavoisier decided to try to determine how long his consciousness would continue
after he was guillotined, by blinking his eyes for as long as possible. He blinked twelve times
after his head was chopped off. Ironically, Judge Coffinhal 
was himself executed only three months later, falling victim to the same chaos.

\margup{-18mm}{\fig{faucet}{Example \ref{eg:faucet}.}
\spacebetweenfigs
\fig{earth}{The earth keeps spinning without slowing down. Energy is conserved.}
\spacebetweenfigs
\fig{coin}{The spinning coin slows down. It seems as though energy isn't
conserved, but it is.}}
\begin{eg}{A stream of water}\label{eg:faucet}
The stream of water is fatter near the mouth of the faucet, and skinnier lower down. This can be understood
using conservation of mass. Since water is being neither created nor destroyed, the mass of the
water that leaves the faucet in one second must be the same as the amount that flows past a lower
point in the same time interval. The water speeds up as it falls, so the two quantities of water can
only be equal if the stream is narrower at the bottom.
\end{eg}

\mysection[0]{Conservation of Energy}\index{energy!conservation of}\index{conservation!of energy}
Noether's theorem says that conservation laws result from symmetries, but the connection between
symmetry and conservation of mass won't be clear until the end of the chapter. As our first full-fledged
example of Noether's theorem in action, we'll instead use conservation of energy.
Energy means something specific and technical in physics, but let's start by appealing to your
everyday knowledge. Energy is what you're buying at the gas station, and you also pay for
it in your electric bill. Energy is why we need food.\footnote{Growing
children also need to eat more than they excrete because conservation of mass would
otherwise make it impossible for them to grow.} These forms of energy can be converted into
others, such as the energy your car has when it's moving, the light from a lamp,
or the body heat that we mammals must continuously produce. We'll first develop a
real scientific definition of energy, and then relate it to symmetry in section
\ref{sec:noether-energy}.

\begin{envsubsection}{Kinetic energy}\index{energy!kinetic}\index{kinetic energy}
Symmetry arguments led us to the conclusion that an isolated object or ray of light can never
slow down, change direction or disappear entirely. But that falls short of being a conservation
law. A full-fledged conservation law says that even when we have many objects interacting, the
total amount of something stays constant. Is there any reason to believe that energy is conserved
in general? The planet earth, \figref{earth}, is a large, complex system consisting of a huge
number of atoms. It keeps on spinning without slowing down, which is evidence in favor of
energy conservation. What about the spinning coin in figure \figref{coin}, however? Does its
energy disappear gradually?

Scientists would have thought so until the nineteenth century, when physicist James\index{Joule, James}
Joule (1818-1889) had an important insight. Joule was the wealthy heir to a Scottish brewery, and funded his
own scientific research. As an industrialist, he had a practical interest in replacing steam engines
with electric ones that would be more efficient, and cost less money to run. Scientists already
knew that friction would cause a spinning coin to slow down, and that friction made engines less
efficient. They also knew that friction heated things up, as when you rub your hands together on
a cold day. Joule, however, realized that it went deeper than this: there was a conserved quantity,
which ended up being called energy. When we first start the coin spinning, its energy is in the
form of motion, with its atoms all going in circles. As it slows down, the energy isn't
disappearing, it's being converted into another form: heat. We now know that
heat is the random motion of atoms. As the coin rubs against the ground, the atoms in the
two surfaces bump into each other, and the amount of random atomic motion increases.
The organized motion of the atoms in the spinning coin is being converted into a disorganized
form of motion, heat.

\margup{-23mm}{\fig{joule}{James Joule}}
Energy of motion is called kinetic energy.
The simplest situations for calculating kinetic energy are those in which an object is moving
through space without spinning or moving internally, e.g., a hockey puck sliding across the
ice. All the atoms in the object are moving at the same speed, so the object's kinetic energy
just depends on two numbers, its mass and its speed. The actual equation can't be proved based
on logic; it can only be determined from experiments. Such experiments were first done
by English physicist Thomas Young, and in lab \thechapter\ref{lab:energy} you're going to reproduce Young's work and
discover his equation for yourself.\index{Young, Thomas!equation for kinetic energy}


When energy is being transferred or changed from one form to another, we use the term ``power''\index{power!defined}
to mean the amount of energy transferred per unit time. The metric unit of power is the watt (W),\index{watt (unit)}
defined as one joule per second.

\begin{eg}{Power of a lightbulb}
Every second, a 100 W lightbulb takes 100 J of energy from the wall socket. (Some of that energy
is turned into light, and the rest just heats your house.)
\end{eg}
\end{envsubsection}
%
\begin{envsubsection}{Gravitational energy}\index{energy!gravitational}\index{gravitational energy}
If you toss a ball up in the air, it slows to a stop and then speeds up again on the way back down.
As in the example of the spinning coin, it seems as though conservation of energy is being violated,
but really we're just seeing evidence that there is a new form of energy coming into play,
gravitational energy. This form of energy depends on distance, not motion:
the farther apart the earth and the ball are, the more gravitational energy there is.

\selfcheck{firefighteronpole}{We've discussed three kinds of energy so far: kinetic energy,
heat energy (which is really kinetic energy at the atomic level), and gravitational energy. Energy
can be converted from any of these forms into any other. Suppose a firefighter slides down the
pole at the fire station, using her grip to control her motion so that she neither speeds up nor
slows down. How would you describe this in terms of energy?}

\margup{-74mm}{\fig{hooverdam}{The water help up high behind Hoover Dam has gravitational energy.}}
The metric unit of energy is the joule (J), and we'll define it as the amount of energy needed
to raise the temperature of 0.24 grams of water by $1\degunit$ Celsius. (Don't memorize that
number!)\index{joule (unit)}
Gravity is a universal attraction between things that have mass.
Here where we live on the earth's surface, the atoms in the earth attract the atoms in all the
objects around us, and measurements show that as a result of all that attraction, an energy of
about 10 J is needed in order to lift a one-kilogram mass by one meter.\footnote{A more precise value
is 9.8 J, but that's close to 10, so we'll usually round off to 10 to simplify numerical
examples. In any case, don't memorize the numbers.} We say that the strength of
the gravitational field, $g$, at the earth's surface is 10 joules per kilogram per meter, or,
in abbreviated form, $g=10\ \junit/\kgunit/\munit$.\index{gravitational field}\index{field!gravitational}

\margup{0mm}{\fig{skaterenergy}{example \ref{eg:skaterenergy}}
\spacebetweenfigs
\fig{orionnebula}{example \ref{eg:birthofstars}}
\spacebetweenfigs
\fig{seesaw}{example \ref{eg:lever}}}
\begin{eg}{The pool skater}\label{eg:skaterenergy}
On the way up the side of the pool, the skater on page \pageref{ch:energy}
has converted all of his kinetic energy into gravitational energy. 
Figure \figref{skaterenergy} shows schematically how the two types of energy are traded off.
(The numbers are just my estimates.)
\end{eg}

\begin{eg}{The birth of stars}\label{eg:birthofstars}\index{Orion Nebula}
Orion is the easiest constellation to find. You can see it in the winter, even if you live
under the light-polluted skies of a big city. Figure \figref{orionnebula} shows an interesting
feature of this part of the sky that you can easily pick out with an ordinary camera (that's how
I took the picture) or a pair of binoculars. The three stars at the top are Orion's belt, and the
stuff near the lower left corner of the picture is known as his sword --- to the naked eye, it
just looks like three more stars that aren't as bright as the stars in the belt. The middle ``star''
of the sword, however, isn't a star at all. It's a cloud of gas, known as the Orion Nebula,
that's in the process of collapsing
due to gravity. Like the pool skater on his way down, the gas is losing gravitational energy.
The results are very different, however. The skateboard is designed to be a low-friction device,
so nearly all  of the lost gravitational energy is converted to kinetic energy, and very little
to heat. The gases in the nebula flow and rub against each other, however, so most of the gravitational
energy is converted to heat. This is the process by which stars are born: eventually the core of
the gas cloud gets hot enough to ignite nuclear reactions.
\end{eg}

\begin{eg}{A lever}\label{eg:lever}\index{lever}
Figure \figref{seesaw} shows two sisters on a seesaw. The one on the left has twice as
much mass, but she's at half the distance from the center. No energy input is needed in
order to tip the seesaw. If the girl on the left goes up a certain distance, her gravitational
energy will increase. At the same time, her sister on the right will drop twice the distance,
which results in an equal decrease in energy, since her mass is half as much.
\end{eg}

\begin{eg}{Lifting a weight}
\egquestion At the gym, you lift a mass of 40 kg through a height of 0.5 m. How much gravitational
energy is required? Where does this energy come from?

\eganswer 
The strength of the gravitational field is 10 joules per kilogram per meter, so after you lift the weight,
its gravitational energy will be greater by $10\times40\times0.5=200$ joules.

Energy is conserved, so
if the weight gains gravitational energy, something else somewhere in the universe must have lost some.
The energy that was used up was the energy in your body, which came from the food you'd eaten. This is what we refer to
as ``burning calories,'' since calories are the units normally used to describe the energy in food,
rather than metric units of joules.

In fact, your body uses up even more than 200 J of food energy, because it's not very efficient. The rest
of the energy goes into heat, which is why you'll need a shower after you work out. We can summarize this
as
\begin{equation*}
\text{food energy} \rightarrow \text{gravitational energy} + \text{heat} \qquad .
\end{equation*}
\end{eg}


\begin{eg}{Lowering a weight}
\egquestion After lifting the weight, you need to lower it again. What's happening in terms of energy?

\eganswer 
Your body isn't capable of accepting the energy and putting it back into storage. The gravitational
energy all goes into heat. (There's nothing
fundamental in the laws of physics that forbids this.
Electric cars can do it --- when you stop at a stop sign, the car's kinetic energy is absorbed
back into the battery, through a generator.) 
\end{eg}


\begin{eg}{Heavy objects don't fall faster}\index{free fall}
Stand up now, take off your shoe, and drop it alongside a much less massive object
such as a coin or the cap from your pen.

Did that surprise you? You found that they both hit the ground at the same time.
The Greek philosopher Aristotle wrote that heavier objects fall faster than
lighter\index{Aristotle}\index{Catholic Church}\index{Church!Catholic}
ones. He was wrong, but Europeans believed him for thousands of years, partly because
experiments weren't an accepted way of learning the truth, and partly because the
Catholic Church gave him its posthumous seal of approval as its official
philosopher.

Heavy objects and light objects have to fall the same way, because conservation
laws are additive --- we find the total energy of an object by adding up the energies
of all its atoms. If a single atom falls through a height of one meter, it loses a certain
amount of gravitational energy and gains a corresponding amount of kinetic energy.
Kinetic energy relates to speed, so that determines how fast it's moving at the end
of its one-meter drop. (The same reasoning could be applied to any point along the
way between zero meters and one.)

Now what if we stick two atoms together? The pair has double the mass, so the amount
of gravitational energy transformed into kinetic energy is twice as much. But twice
as much kinetic energy is exactly what we need if the pair of atoms is to have the same
speed as the single atom did. Continuing this train of thought, it doesn't matter how many
atoms an object contains; it will have the same speed as any other object after dropping
through the same height.
\end{eg}

\selfcheck{dropfeather}{Part of the Aristotelian confusion was probably because of examples
like dropping a feather. A feather won't fall as quickly as a rock. Why is this? Our unspoken
assumption was that the only energy transformation going on was
\begin{equation*}
	\text{gravitational energy} \rightarrow \text{kinetic energy} \qquad .
\end{equation*}
Evidently this assumption fails --- most of the feather's gravitational energy is being
converted into something else besides kinetic energy. What other form of energy is there?
}

\margup{-190mm}{\fig{irface}{This photo was made with a special camera that records infrared
	light. The man's warm skin emits quite a bit of infrared light energy, while his
	hair, at a lower temperature, emits less.}
\spacebetweenfigs
\fig{irbike}{An infrared camera distinguishes
	hot and cold areas. As the bike skids to a stop with its brakes locked, the kinetic
	energy of the bike and rider is converted into heat in both the floor (top) and the
	tire (bottom).}}
\begin{envsubsection}{Emission and absorption of light}
The example of the falling feather shows how tricky this can get. Often we miss something
vital because it's invisible. When a guitar string gradually stops vibrating, it may seem as
though its energy was just disappearing; sound has energy, but we may forget that because
sound is invisible. When the feather drops, the heating of the feather and the air are not
only invisible but nearly undetectable without heroic
measures.\index{light!emission and absorption of}\index{light!infrared}\index{emission of light}\index{absorption of light}

Imagine how difficult it was for Joule to figure out all of this for the first time! One
challenge in his experiments is demonstrated in figure
\figref{irface}. In general, light can heat matter
(sunlight on your skin) and matter can also get rid of its heat energy by emitting light
(a candle flame):
\begin{equation*}
	\text{heat} \leftrightarrow \text{light}
\end{equation*}
Light, however, includes more than just the spectrum of visible colors extending from
red to violet on the rainbow. Hot objects, like the sun or a lightbulb filament, do emit
visible light, but matter at lower temperatures gives off
infrared light, a color of light that lies beyond the red end of the visible rainbow. 

\margup{-50mm}{\fig{irball}{A squash ball before and after several minutes of play.}}
Although the emission and
absorption of infrared light was just a source of trouble and confusion for Joule, we can
also use infrared photography to gain insight into phenomena in which other types of energy
are converted into heat.
The heating of the tire and floor in figure \figref{irbike} is something that
the average person might have predicted in advance, but there are other situations
where it's not so obvious. When a ball slams into a wall, it doesn't rebound with the
same amount of kinetic energy. Was some energy destroyed? No.
The ball and the wall heat up. Figure \figref{irball} shows
a squash ball at room temperature (top), and after it has been played with for
several minutes (bottom), causing it to heat up detectably.
\end{envsubsection}
%
\begin{envsubsection}{How many forms of energy?}\index{energy!forms of}
How many different types of energy are there? At this point, you might worry that
you were going to have to memorize a long list of them. The good news is that there
aren't really that many at all.

\widefigsidecaption[c]{archery}{At the atomic level, the energy in the bow is really electrical energy}

In figure \figref{archery}, the bow evidently contains some stored energy, since we observe that
the arrow gets kinetic energy from it. What kind of energy is this? Is it some new and
mysterious ``bow energy?'' No. At the atomic level, things get a lot simpler. The energy
in the bow is electrical energy of the interacting atoms. Just as a rock can have more or
less gravitational energy depending on its distance from the earth, an atom can have more
or less electrical energy depending on its distance from another atom.

 Many other forms\index{energy!electrical}\index{energy!magnetic}\index{electrical energy}\index{magnetic energy}
of energy turn out to be electrical energy in disguise, \figref{electricalenergy}. In particular, chemical reactions are based on
electrical energy: in a reaction, atoms are rearranged like tinker toys, which changes their
distances from one another. Food and gasoline are both fuels that store
electrical energy.

Every type of energy you encounter in your day-to-day life is really
just something from the following short list:

%The following kludge is necessitated by a bug in the description environment.
\newcommand{\energylistitem}[2]{\hspace{4mm}\textbf{#1} #2}

\noindent\energylistitem{kinetic energy}{(including heat)}

\noindent\energylistitem{gravitational energy}{}

\noindent\energylistitem{electrical and magnetic energy}{(including light, which is an}\\
\makebox[0mm][l]{\hspace{12mm} electrical and magnetic wave)}
%        \raisebox{\ymarginscoot}[0mm][0mm]{\makebox[0mm][l]{\hspace{\xmarginscoot}\usebox{\sidecaptionbox}}}

\noindent We'll discuss electricity and magnetism in more detail in chapter 7. Two forms of nuclear
energy can also be added to the list.\index{energy!nuclear}\index{nuclear energy}
 One of the main goals of physics is to
classify all the interactions: gravitational, electrical, and so on. 

\margup{-64mm}{%
\fig{electricalenergy}{All of these energy transformations turn out at the atomic level
to be changes in electrical energy resulting from changes in the distances between atoms.}
}

Physicists
generally believe that there is an underlying simplicity to the laws of physics, and
consider it a triumph when they can reveal part of it. You might wonder, for instance,
why electrical and magnetic energy are shown as a single item on the list above. Well,
just as we learned that ``bow energy'' and ``food energy'' are really both just types of
electrical energy, we'll see in chapter \ref{ch:em} that electricity and magnetism are really
just two sides of the same coin.\label{introunificationofforces}
\end{envsubsection}

\dqheader
\begin{dq}\label{dq:can}
In figure \figref{cancrumples}, a small amount of hot water is poured into the empty can, which
rapidly fills up with hot steam. The can is then sealed tightly, and soon crumples.
How can this be explained based on the idea that heat is a form of random motion of
atoms?

\widefig{cancrumples}{Discussion question \ref{dq:can}.}
\end{dq}


\vfill

\mysection[4]{Newton's Law of Gravity}\index{gravity!Newton's law of}
Why does the gravitational field on our planet have the particular value it does? 
For insight, let's compare with the strength of gravity elsewhere in the
universe:\index{Newton, Isaac!law of gravity}\index{gravitational field!Newton's law of gravity}
\index{field!gravitational!Newton's law of gravity}

\begin{tabular}{|p{50mm}|p{45mm}|}
\hline
location      & $g$ (joules per kg per m) \\
\hline
asteroid Vesta (surface) & 0.3 \\
earth's moon (surface)   & 1.6 \\
Mars (surface)           & 3.7 \\
earth (surface)          & 9.8 \\
Jupiter (cloud-tops)     & 26 \\
sun (visible surface)    & 270 \\
typical neutron star (surface) & $10^{12}$ \\
black hole (center)      & infinite according to some theories, on the
   order of $10^{52}$ according to others \\
\hline
\end{tabular}

A good comparison is Vesta versus a neutron star. They're roughly the same size, but they have
vastly different masses --- a teaspoonful of neutron star matter would weigh a million tons!
The different mass must be the reason for the vastly different gravitational fields. (The notation
$10^{12}$ means 1 followed by 12 zeroes.)
This makes sense, because gravity is an attraction between things that have mass.

The mass of an object, however, isn't the only thing that determines the strength of its
gravitational field, as demonstrated by the difference between the fields of the
sun and a neutron star, despite their similar masses.  The other variable that matters is
distance. Because a neutron star's mass is compressed into such a small space (comparable
to the size of a city), a point on its surface is within a fairly short distance from every
atom in the star. If you visited the surface of the sun, however, you'd be millions of miles
away from most of its atoms.

As a less exotic example, if you travel from the seaport of
Gua\-ya\-quil, Ecuador, to the top of nearby Mt. Cotopaxi, you'll experience
a slight reduction in gravity, from 9.7806 to 9.7624 J/kg/m. This is because
you've gotten a little farther from the planet's mass. Such differences in the
strength of gravity between one location and another on the earth's surface were
first discovered because pendulum clocks that were correctly calibrated in one country
were found to run too fast or too slow when they were shipped to another location.

\pagebreak[4]

The general equation for an object's gravitational field was discovered by
Isaac Newton, by working backwards
from the observed motion of the planets:\footnote{Example \ref{eg:moonorbit} on page \pageref{eg:moonorbit} shows
the type of reasoning that Newton had to go through.}
\begin{equation*}
	g = \frac{GM}{d^2} \qquad ,
\end{equation*}
where $M$ is the mass of the object, $d$ is the distance from the object, and $G$ is
a constant that is the same everywhere in the universe. This is known as Newton's
law of gravity.\footnote{This is not the form
in which Newton originally wrote the equation.}
It's an inverse square law, which is reasonable since an object's gravitational
field is an effect that spreads outward from it in all directions.
Newton's law of gravity really gives the field of an individual atom, and
the field of a many-atom object is the sum of the fields of the atoms.
Newton was able to prove mathematically that this scary sum has an unexpectedly
simple result in the case of a spherical object such as a planet: the result is
the same as if all the object's mass had been concentrated at its center.\index{gravitational constant, $G$}

\margup{-80mm}{%
\fig{newton}{Isaac Newton (1642-1727)}
}
Newton showed that his theory of gravity could explain the orbits of the planets, and
also finished the project begun by Galileo of driving a stake through the heart of
Aristotelian physics. His book on the motion of material objects, the \emph{Mathematical
Principles of Natural Philosophy},\index{Mathematical Principles of Natural Philosophy}\index{Principia Mathematica}
was uncontradicted by experiment for 200 years,
but his other main work, \emph{Optics},\index{Optics}
 was on the wrong track due to his conviction
that light was composed of particles rather than waves. He was an avid alchemist,
an embarrassing fact that modern scientists would like to forget. Newton was
on the winning side of the revolution that replaced King James II with William and Mary
of Orange, which led to a lucrative post running the English royal mint; he worked hard at
what could have been a sinecure, and took great
satisfaction from catching and executing counterfeitors. Newton's personal life was less
happy. Rejected by his mother at an early age, he never married or formed any close
attachments, except for one intense emotional relationship with a younger man; around the time when this
liaison ended, Newton experienced what we would today probably describe as a nervous breakdown.\footnote{The historical
record can't be decoded with certainty. Seventeenth-century England didn't conceive of mental illness in the same way we
do now. Homosexuality was a capital offense, not a personal preference. If Newton was homosexual, he had a
strong motivation not to record the fact.}

\widefigsidecaption[t]{newtonsapple}{example \ref{eg:newtonsapple}}

\begin{eg}{Newton's apple}\label{eg:newtonsapple}\index{moon!gravitational field experienced by}\index{Newton, Isaac!apple myth}
A charming legend attested to by Newton's niece is that he first conceived of
gravity as a universal attraction after seeing an apple fall from a tree. He
wondered whether the force that made the apple fall was the same one that made the
moon circle the earth rather than flying off straight. Newton had astronomical data
that allowed him to calculate
that the gravitational field the moon experienced
from the earth was 1/3600 as strong as the field on the surface of the earth.\footnote{See example
\ref{eg:moonorbit} on page \pageref{eg:moonorbit}.}
(The moon has its own gravitational field, but that's not what we're talking about.)
The moon's distance from the earth is 60 times greater than the earth's radius,
so this fit perfectly with an inverse-square law: $60\times60=3600$.
\end{eg}
\end{envsubsection}
%
\mysection[4]{Noether's Theorem for Energy}\label{sec:noether-energy}
\index{Noether's theorem!for energy}\index{energy!Noether's theorem}
Now we're ready for our first full-fledged example of Noether's theorem.
Conservation of energy is a law of physics, and Noether's theorem
says that the laws of physics come from symmetry. Specifically, Noether's
theorem says that every symmetry implies a conservation law. Conservation
of energy comes from a symmetry that we haven't even discussed yet, but one that
is simple and intuitively appealing: as time goes by, the universe doesn't
change the way it works. This is a kind of translation symmetry, but in time,
not space.\index{symmetry!time-translation}\index{time-translation symmetry}

We have strong evidence for time translation symmetry, because when
we see a distant galaxy through a telescope, we're seeing light that has taken
billions of years to get here. A telescope, then, is like a time machine. For all
we know, alien astronomers with advanced technology may be observing our planet
right now,\footnote{Our present technology isn't good enough to let us pick
the planets of other solar systems out from the glare of their suns, except in a few
exceptional cases.}
but if so, they're seeing it not as it is now but as it
was in the distant past, perhaps in the age of the dinosaurs, or before life
even evolved here. As we observe a particularly distant, and therefore ancient,
supernova, we see that its explosion plays out in exactly the same way as
those that are closer, and therefore more recent.

Now suppose physics really does change from year to year, like politics, pop music,
and hemlines. Imagine, for example, that the ``constant'' $G$ in Newton's
law of gravity isn't quite so constant. One day you might wake up and find that you've
lost a lot of weight without dieting or exercise, simply because gravity has gotten
weaker since the day before. 

If you know about such changes in $G$ over time, it's the ultimate insider
information. You can use it to get as rich as Croesus, or even Bill Gates.\index{Gates, Bill}
On a day when $G$ is low, you pay for the energy needed to lift
a large mass up high. Then, on a day when gravity is stronger,
you lower the mass back down, extracting its gravitational energy.
The key is that the energy you get back out is greater than what you
originally had to put in. You can run the cycle over and over again, always
raising the weight when gravity is weak, and lowering it when gravity is strong.
Each time, you make a profit in energy. Everyone else thinks energy is conserved,
but your secret technique allows you to keep on increasing and increasing the amount
of energy in the universe (and the amount of money in your bank account).

The scheme can be made to work if anything about physics changes over time, not
just gravity. For instance, suppose that the mass of an electron had one value
today, and a slightly different value tomorrow. Electrons are one of the basic
particles from which atoms are built, so on a day when the mass of electrons is low, every
physical object has a slightly lower mass. In problem \ref{hw:changing-electron-mass}
on page \pageref{hw:changing-electron-mass}, you'll work out a way that this could
be used to manufacture energy out of nowhere.\label{text:changing-electron-mass}

Sorry, but it won't work. Experiments show that $G$ doesn't change measurably over
time, nor does there seem to be any time variation in any of the other rules by which the universe
works.\footnote{In 2002, there have been some reports that the properties of atoms
as observed in distant galaxies are slightly different than those of atoms here and
now. If so, then time translation symmetry is weakly violated, and so is conservation
of energy. However, this is a revolutionary claim, and it needs to be examined carefully.
The change being claimed is large enough that, if it's real, it should be detectable
from one year to the next in ultra-high-precision laboratory experiments here on earth.}
The rules of the game are symmetric under time translation. If archaeologists find a
copy of this book thousands of years from now, they'll be able to reproduce all the
experiments you're doing in this course.

I've probably convinced you that if time-translation symmetry was violated, then
conservation of energy wouldn't hold. But does it work the other way around? If time-translation
symmetry is valid, must there be a law of conservation of energy? 
Logically, that's a different question. We may be able to
prove that if A is false, then B must be false, but that doesn't mean that if A is
true, B must be true as well.
For instance, if you're not a criminal, then you're presumably not in jail, but just
because someone is a criminal, that doesn't mean he is in jail
--- some criminals never get caught.

Noether's theorem does work the other way around as well: if physics has a certain
symmetry, then there must be a certain corresponding conservation law. This is
a stronger statement. The full-strength version of Noether's theorem can't
be proved without a  model of light and matter more detailed than the one
currently at our disposal.

\mysection[4]{Equivalence of Mass and Energy}\label{sec:massenergy}
\begin{envsubsection}{Mass-energy}\index{mass-energy}\index{mass!equivalence to energy}\index{energy!equivalence to mass}
You've encountered two conservation laws so far: conservation of mass and conservation
of energy. If conservation of energy is a consequence of symmetry, is there a
deeper reason for conservation of mass?

Actually they're not even separate conservation laws. Albert Einstein found,\index{Einstein, Albert}
as a consequence of his theory of relativity, that mass and energy are equivalent, and
are not separately conserved --- one can be converted into the other. Imagine that
a magician waves his wand, and changes a bowl of dirt into a bowl of lettuce. You'd be
impressed, because you were expecting that both dirt and lettuce would be conserved
quantities. Neither one can be made to vanish, or to appear out of thin air. However,
there are processes that can change one into the other. A farmer changes dirt into
lettuce, and a compost heap changes lettuce into dirt. At the most fundamental
level, lettuce and dirt aren't really different things at all; they're just collections
of the same kinds of atoms --- carbon, hydrogen, and so on.

We won't examine relativity in detail until chapter \ref{ch:relativity}, but mass-energy
equivalence is an inevitable implication of the theory, and it's the only part of the
theory that most people have heard of, via the famous equation $E=mc^2$. This equation
tells us how much energy is equivalent to how much mass: the conversion factor is the square
of the speed of light, $c$. Since $c$ a big number, you get a really really big number
when you multiply it by itself to get $c^2$. This means that even a small amount of mass
is equivalent to a very large amount of energy. 

\begin{eg}{Gravity bending light}\label{eg:eclipse}
Gravity is a universal attraction between things that have mass, and since the energy
in a beam of light is equivalent to a some very small amount of mass, we expect that
light will be affected by gravity, although the effect should be very small.
The first experimental confirmation of relativity
came in 1919 when stars next to the sun during a solar eclipse were
observed to have shifted a little from their ordinary
position. (If there was no eclipse, the glare of the sun
would prevent the stars from being observed.) Starlight had
been deflected by the sun's gravity. Figure \figref{eclipse} is a
photographic negative, so the circle that appears bright is actually the
dark face of the moon, and the dark area is really the bright corona of
the sun. The stars, marked by lines above and below then, appeared at
positions slightly different than their normal ones.
\end{eg}

\widefig{eclipse}{example \ref{eg:eclipse}}

\begin{eg}{Black holes}\index{black hole}
A star with sufficiently strong gravity can prevent light
from leaving. Quite a few black holes have been detected via
their gravitational forces on neighboring stars or clouds of gas and dust.
\end{eg}


Because mass and energy are like two different sides of the same coin, we may speak of
mass-energy, a single conserved quantity, found by adding up all the mass and energy,
with the appropriate conversion factor: $E+mc^2$.

\begin{eg}{A rusting nail}\label{eg:rustingnail}
\egquestion
An iron nail is left in a cup of water
until it turns entirely to rust. The energy released is
about 500,000 joules. In theory, would a sufficiently
precise scale register a change in mass? If so, how much?

\eganswer
 The energy will appear as heat, which will be lost
to the environment. The total mass-energy of the cup,
water, and iron will indeed be lessened by 500,000 joules. (If it
had been perfectly insulated, there would have been no
change, since the heat energy would have been trapped in the
cup.) The speed of light in metric units is
$c=3\times10^8$ meters per second (scientific notation for
3 followed by 8 zeroes), so converting to mass units, we have
\begin{align*}
		m	 &=    \frac{E}{c^2}  \\
			&= \frac{500,000}{\left(3\times10^8\right)^2} \\
			 &=    0.000000000006\  \text{kilograms}   \qquad   .
\end{align*}
(The design of the metric system is based on the meter, the kilogram, and the
second. The joule is designed to fit into this system, so the result comes
out in units of kilograms.)
The change in mass is too small to measure with any
practical technique. This is because the square of the speed
of light is such a large number in metric units.
\end{eg}
\end{envsubsection}
%
\begin{envsubsection}{The correspondence principle}\index{correspondence principle}
The realization that mass and energy are not separately conserved is our first example
of a general idea called the correspondence principle. When Einstein came up with
relativity, conservation of energy had been accepted by physicists for decades, and
conservation of mass for
over a hundred years.\index{correspondence principle!for mass-energy equivalence}

Does an example like this mean that physicists don't know what they're talking about?
There is a recent tendency among social scientists to 
deny that the scientific method even exists, claiming that
science is no more than a social system that
determines what ideas to accept based on an in-group's
criteria. If science is an
arbitrary social ritual, it would seem difficult to explain
its effectiveness in building such useful items as
airplanes, CD players and sewers. If voodoo
and astrology were no less scientific in
their methods than chemistry and physics, what was it that
kept them from producing anything useful?
This silly attitude was effectively skewered by a famous hoax
carried out in 1996 by New York University physicist Alan Sokal.\index{Sokal, Alan} Sokal wrote
an article titled ``Transgressing the Boundaries: Toward a Transformative 
Hermeneutics of Quantum Gravity,'' and got it accepted by a cultural studies
journal called \emph{Social Text}.\footnote{The paper
appeared in \emph{Social Text} \#46/47 (1996) pp. 217-252. The full text
is available on professor Sokal's web page at www.physics.nyu.edu/faculty/sokal/.}
The scientific content of the paper is a carefully constructed soup of
mumbo jumbo, using technical terms to create maximum confusion; I can't make
heads or tails of it, and I assume the editors and peer reviewers at
\emph{Social Text} understood even less. The physics, however, is mixed
in with cultural relativist statements designed to appeal to them ---
``\ldots the truth claims of science are inherently theory-laden and self-referential'' ---
and footnoted references to academic articles such as 
``Irigaray's and Hayles' exegeses of gender encoding in fluid mechanics \ldots
and \ldots Harding's comprehensive critique of the gender ideology underlying
the natural sciences in general and physics in particular\ldots''
On the day the article came out, Sokal published a letter explaining that
the whole thing had been a parody --- one that apparently went over the heads
of the editors of \emph{Social Text}.

What keeps physics from being merely a matter of fashion is that it has to agree
with experiments and observations. If a theory such as conservation of mass or
conservation of energy became accepted in physics, it was because it was supported
by a vast number of experiments. It's just that experiments never have perfect
accuracy, so a discrepancy such as the tiny change in the mass of the rusting nail
in example \ref{eg:rustingnail} was undetectable. The old experiments weren't all
wrong. They were right, within their limitations. If someone comes along with a
new theory he claims is better, it must still be consistent with all the same
experiments. In computer jargon, it must be backward-compatible. This is called
the correspondence principle: new theories must be compatible with old ones in
situations where they are both applicable. The correspondence principle tells us
that we can still use an old theory within the realm where it works, so for instance
I'll typically refer to conservation of mass and conservation of energy in this
book rather than conservation of mass-energy, except in cases where the new theory
is actually necessary.

Ironically, the extreme cultural relativists want to attack what they see
as physical scientists' arrogant claims to absolute truth, but what they
fail to understand is that science only claims to be able to find partial, provisional truth.
The correspondence principle tells us that each of today's scientific truth can be superseded
tomorrow by another truth that is more accurate and more broadly applicable. It also
tells us that today's truth will not lose any value when that happens.
\end{envsubsection}

% Einstein and mass-energy, because otherwise N's thm leaves mass dangling
% could discuss tides, and apply cons of mass to explain why don't get tides in lakes
% hw: numerical example using a table, discretized in time, to show delta-v/delta-t is const
% hw: blood vessel, change in speed
% hw or lab: center of mass, and then something with stability, e.g. the toys that look unstable but aren't;
%    they really need this /before/ they do the KE lab
% hw: SN 1, e.g. thermometer, hydraulic jack, river splits
% hw involving E=mc2? if so, need to give numerical value for c

%===============================================================================
%===============================================================================

\begin{hwsection}

\begin{hw}{jumpkeandpe}
You jump up straight up in the air. When do you have the greatest
gravitational energy? The greatest kinetic energy?
(Based on a problem by Serway and Faughn.)
\end{hw}

\begin{hw}{throw-down-and-up}
	Anya and Ivan lean over a balcony side by side. Anya throws a penny downward with
	an initial speed of 5 m/s. Ivan throws a penny upward with the same speed. Both pennies
	end up on the ground below. Compare their kinetic energies and velocities on impact.
\end{hw}

\marg{\fig{hw-pulley}{Problem \ref{hw:pulley}.}}
\begin{hw}{pulley}
(a) If weight B moves down by a certain amount, how much does weight A
move up or down?\hwendpart
(b) What should the ratio of the two weights be if they are to balance?
Explain in terms of conservation of energy.
\end{hw}

\begin{hw}{rocketweight}
How high above the surface of the earth should a rocket be in order to
have 1/100 of its normal weight? Express your answer in units of earth radii.
\end{hw}

\begin{hw}{slidingmagnets}
	(a) You release a magnet on a tabletop near a big piece
	of iron, and the magnet leaps across the table to the iron.
	Does the magnetic energy increase or decrease?
	Explain.\hwendpart
	(b) Suppose instead that you have two repelling
	magnets. You give them an initial push towards each other,
	so they decelerate while approaching each other. Does the
	magnetic energy increase or decrease? Explain.
\end{hw}

\begin{hw}{spacesuit}
A closed system can be a bad thing --- for an astronaut
sealed inside a space suit, getting rid of body heat can be
difficult. Suppose a 60-kg astronaut is performing vigorous
physical activity, expending 200 watts of power. If none of
the heat can escape from her space suit, how long will it
take before her body temperature rises by $6\degcunit$
($11\degunit\textup{F}$), an amount  sufficient to kill her?  Assume
that the amount of heat required to raise her body
temperature by $1\degcunit$ is the same as it would be for an
equal mass of water. Express your answer in units of minutes.
\end{hw}

\begin{hw}{changing-electron-mass}
As suggested on page \pageref{text:changing-electron-mass}, imagine that the mass of
the electron rises and falls over time. (Since all electrons are identical, physicists
generally talk about ``the electron'' collectively, as in ``the modern man wants more
than just beer and sports.'') The idea is that all electrons are increasing and
decreasing their masses in unison, and at any given time, they're all identical. They're
like a litter of puppies whose weights are all identical on any given day, but who all
change their weights in unison from one month to the next.
Suppose you were the only person who knew about these small day-to-day changes in the
mass of the electron. Find a plan for violating conservation of energy and getting
rich.
\end{hw}

\begin{hw}{rustingnailsmallc}
A typical balance like the ones used in school classes can be read to an accuracy
of about plus or minus $0.1$ grams, or $10^{-4}$ kg. What if the laws of physics had been
designed around a different value of the speed of light? To make mass-energy
equivalence detectable in example \ref{eg:rustingnail} on page \pageref{eg:rustingnail}
using an ordinary balance, would $c$ have to be smaller than it is in our universe,
or bigger? Find the value of $c$ for which the effect would be just barely detectable.
\end{hw}

\begin{hw}[2]{perpetualmotion}
Physics in the modern sense of the word began in the seventeenth century, with Galileo
and Newton, but conservation of energy wasn't discovered until the nineteenth century.
In the intervening period, there was no scientific reason to think that it was impossible
to make a perpetual motion machine, which today we would describe as a machine that
creates more energy than it takes in. For instance, people tried to make cars that would
run forever without requiring fuel. We now know this is impossible because of conservation
of energy; as a car rolls, a great deal of frictional heating occurs, and the amount of
heat created must be the same as the amount of energy consumed by burning the fuel.
Even so, people still try to make perpetual motion machines. The U.S. patent office long
ago eliminated its general requirement that a working model accompany a patent application, but the
requirement still applies to attempts to patent a perpetual motion machine; since a working
model is forbidden by the laws of physics, this has the effect of making it impossible to
patent a perpetual motion machine. Nowadays, enthusiasts tend to talk about ``free energy''
or ``vacuum energy'' 
rather than ``perpetual motion.'' 
(Vacuum energy is legitimate physics, but these people are trying to say
it can be used to violate conservation of energy, which is wrong.)
Websurf, and try to find some examples of people
promoting or selling perpetual motion machines or designs for them. Is it clear where the
border lies between science and pseudoscience? If you form opinions about which people's
web pages are scams, would you be able to convince someone who hadn't taken a physics
course? Can you find any free-energy nuts within otherwise respectable organizations such
as NASA? --- in Google (google.com), for instance, you can do an advanced search in which you ask only
for results from a specific domain like nasa.gov. What about category-based guides to the
Web, such as Open Directory (dmoz.org) or Yahoo (yahoo.com)? How do their editors seem to
treat pseudoscience sites? Do you agree with their decisions? Back up all your statements with
specific descriptions of the data you collected by websurfing.
\end{hw}


\hwnote{%
Problem \ref{hw:colliding-balls}
is to be done after you've completed lab \ref{ch:energy}\ref{lab:energy}, and know the
equation for an object's kinetic energy in terms of its mass and speed.
}

\widefigsidecaption{psscballs}{Problem \ref{hw:colliding-balls}.}
\begin{hw}{colliding-balls}
	The multiflash photograph below shows a collision
	between two pool balls. The ball that was initially at rest
	shows up as a dark image in its initial position, because
	its image was exposed several times before it was struck and
	began moving. By making measurements on the figure,
	determine whether or not energy appears to have been
	conserved in the collision. What systematic effects would
	limit the accuracy of your test? (From an example in PSSC
	Physics.)
\end{hw}

\end{hwsection}

%========================================== labs ===============================================

%--------------------------------- conservation laws lab --------------------------------------

\begin{lab}{Conservation Laws}

\apparatus
\emph{Part A:}\\
\equipn{vacuum pump}{1}\\
\equipn{electronic balance (large capacity)}{1}\\
\equipn{plastic-coated flask}{1/group}\\
\emph{Part B: }\\
\equipn{propyl alcohol}{200 mL/group}\\
\equipn{canola oil}{200 mL/group}\\
\equipn{funnels}{2/group}\\
\equipn{100-mL volumetric flask}{1/group}\\
\equip{rubber stopper, fitting in}\\
\equipn{volumetric flask}{1/group}\\
\equipn{1-ml pipette and bulb}{1/group}\\
\equipn{magnetic stirrer}{1/group}\\
\equipn{triple-beam balance}{1/group}

\index{conservation!of mass}\index{conservation laws!lab}

\labintroduction

Styles in physics come and go, and once-hallowed principles
get modified as more accurate data come along, but some of
the most durable features of the science are its conservation
laws.  A conservation law is a statement that something
always remains constant when you add it all up.  Most people
have a general intuitive idea that the amount of a substance
is conserved.  That objects do not simply appear or
disappear is a conceptual achievement of babies around the
age of 9-12 months.  Beginning at this age, they will for
instance try to retrieve a toy that they have seen being
placed under a blanket, rather than just assuming that it no
longer exists. Conservation laws in physics have the
following general features:

\begin{itemize}
\item[] Physicists trying to find new conservation laws will try
to find a measurable, numerical quantity, so that they can
check quantitatively whether it is conserved.  One needs an
operational definition of the quantity, meaning a definition
that spells out the operations required to measure it.

\item[] Conservation laws are only true for closed systems.  For
instance, the amount of water in a bottle will remain
constant as long as no water is poured in or out.  But if
water can get in or out, we say that the bottle is not a
closed system, and conservation of matter cannot be applied to it.

\item[] The quantity should be additive.  For instance, the amount
of energy contained in two gallons of gasoline is twice as
much as the amount of energy contained in one gallon; energy
is additive.  An example of a non-additive quantity is
temperature.  Two cups of coffee do not have twice as high a
temperature as one cup.

\item[] Conservation laws always refer to the total amount of the
quantity when you add it all up.  If you add it all up at
one point in time, and then come back at a later point in
time and add it all up, it will be the same.
\end{itemize}

How can we pin down more accurately the concept of the
``amount of a substance''?  Should a gallon of shaving cream
be considered ``more substantial'' than a brick? At least
two possible quantities come to mind: mass and volume.  Is
either conserved?  Both?  Neither?  To find out, we will
have to make measurements.

We can measure mass by the ``see-saw method'' --- when two
children are sitting on the opposite sides of a see-saw, the
more massive one has to move closer in to the fulcrum to
make it balance.  If we enslave some particular child as our
permanent mass standard, then any other child's mass can
simply be measured by balancing them on the other side and
measuring their distance from the fulcrum.  A more practical
version of the same basic principle that does not involve
human rights violations is the familiar pan balance
with sliding weights.

Volume is not necessarily so easy to measure.  For instance,
shaving cream is mostly air, so should we find a way to
measure just the volume of the bubbly film itself?  Precise
measurements of volume can most easily be done with liquids
and gases, which conform to a vessel in which they are placed.

Should a gas, such as air, be counted as having any
substance at all?  Empedocles of Acragas (born ca. 492 BC)
was the originator of the doctrine that all material
substances are composed of mixtures of four elements: earth,
fire, water and air.  The idea seems amusingly naive now
that we know about the chemical elements and the periodic
table, but it was accepted in Europe for two thousand years,
and the inclusion of air as a material substance was
actually a nontrivial concept.  Air, after all, was
invisible, seemed weightless, and had no definite shape. 
Empedocles decided air was a form of matter based on
experimental evidence: air could be trapped under water in
an inverted cup, and bubbles would be released if the cup
was tilted.  It is interesting to note that in China around
300 BC, Zou Yan came up with a similar theory, and his five
elements did not include air.

Does air have weight?  Most people would probably say no,
since they do not feel any physical sensation of the
atmosphere pushing down on them.  A delicate house of cards
remains standing, and is not crushed to the floor by the
weight of the atmosphere.

Compare that to the experience of a dolphin, though.  A
dolphin might contemplate a tasty herring suspended in front
of it and conjecture that water had no weight, because the
herring did not involuntarily shoot down to the sea floor
because of the weight of the water overhead.  Water does
have weight, however, which a sufficiently skeptical dolphin
physicist might be able to prove with a simple experiment. 
One could weigh a 1-liter metal box full of water and then
replace the water with air and weigh it again.  The
difference in weight would be the difference in weight
between 1 liter of water of and 1 liter of air.  Since air
is much less dense than water, this would approximately
equal the weight of 1 liter of water.

Our situation is similar to the dolphin's, as was first
appreciated by Torricelli, whose experiments led him to
conclude that ``we live immersed at the bottom of a sea
of...air.''  A human physicist, living her life immersed in
air, could do a similar experiment to find out whether air
has weight.  She could weigh a container full of air, then
pump all the air out and weigh it again.  When all the
matter in a container has been removed, including the air,
we say that there is a vacuum in the container.  In
reality, a perfect vacuum is very difficult to create.  A
small fraction of the air is likely to remain in the
container even after it has been pumped on with a vacuum
pump.  The amount of remaining air will depend on how good
the pump is and on the rate at which air leaks back in to
the container through holes or cracks.

Galileo gave the first experimental proof that air had
weight by the opposite method of compressing the air in a
glass bulb to stuff more air than normal into it, and
comparing its weight to what it had been when ordinary,
uncompressed air was in it.

\labsubsubsection{Cautions}

Please do not break the glassware!  The vacuum flasks and
volumetric flasks are expensive.

The alcohol you will be using in this lab is chemically
different from the alcohol in alcoholic beverages.  It is
poisonous, and can cause blindness or death if you drink it.
 It is not hazardous as long as you do not drink it.

\labobservations

\labpart{ Density of air}

You can remove the air from the flask by attaching the
vacuum pump to the vacuum flask with the rubber and glass
tubing, then turning on the pump.  You can use the scale to
determine how much mass was lost when the air was evacuated.

Make any other observations you need in order to find out
the density of air.

\labpart{ Is volume and/or mass conserved when two fluids are mixed?}

The idea here is to find out whether volume and/or mass is
conserved when water and alcohol are mixed.  The obvious way
to attempt this would be to measure the volume and mass of a
sample of water, the volume and mass of a sample of alcohol,
and their volume and mass when mixed.  There are two
problems with the obvious method: (1) when you pour one of
the liquids into the other, droplets of liquid will be left
inside the original vessel; and (2) the most accurate way to
measure the volume of a liquid is with a volumetric flask,
which only allows one specific, calibrated volume to be measured.

\labfig{lab-flask}

Here's a way to get around those problems.  Put the magnetic
stirrer inside the flask.  Pour water through a funnel into
a volumetric flask, filling it less than half-way.  (Do not
use the pipette to transfer the water.)  A common mistake is
to fill the flask more than half-way.  Now pour a thin layer
of cooking oil on top.  Cooking oil does not mix with water,
so it forms a layer on top of the water.  (Set aside one
funnel that you will use only for the oil, since the oil
tends to form a film on the sides.)  Finally, gently pour
the alcohol on top.  Alcohol does not mix with cooking oil
either, so it forms a third layer.  By making the alcohol
come exactly up to the mark on the calibrated flask, you can
make the total volume very accurately equal to 100 mL.  In
practice, it is hard to avoid putting in too much alcohol
through the funnel, so if necessary you can take some back
out with the pipette.

If you put the whole thing on the balance now, you know both
the volume (100 mL) and the mass of the whole thing when the
alcohol and water have been kept separate.  Now, mix
everything up with the magnetic stirrer.  The water and
alcohol form a mixture.  You can now test whether the volume
or mass has changed.

If the mixture does not turn out to have a volume that looks
like exactly 100 mL, you can use the following tricks to
measure accurately the excess or deficit with respect to 100
mL.  If it is less than 100 mL, weigh the flask, pipette in
enough water to bring it up to 100 mL, weigh it again, and
then figure out what mass and volume of water you added
based on the change in mass.  If it is more than 100 mL,
weigh the flask, pipette out enough of the mixture to bring
the volume down to 100 mL, weigh it again, and make a
similar calculation using the change in mass and the density
of the oil.  If you need to pipette out some oil, make sure
to wash and rinse the pipette thoroughly afterwards.

\labwriteup

A. If your results show that air has weight, determine the
(nonzero) density of air, taking into account the accuracy of
your data.

B. Decide whether volume and/or mass is conserved when
alcohol and water are mixed, taking into account the accuracy of
your data.

\end{lab}

%--------------------------------- conservation of energy lab --------------------------------------

\begin{lab}{Conservation of Energy}\label{lab:energy}

\apparatus
\equip{air track}\\
\equip{carts, large and small}\\
\equip{photogate (PASCO) (under lab benches in rm. 418)}\\
\equip{computer}\\
\equip{air blowers}\\
\equip{string}\\
\equip{cylindrical pendulum bobs}\\
\equip{hook}\\
\equip{meter sticks and rulers}\\
\equip{wood blocks}

\begin{goals}
\item[] Learn how a new form of energy is discovered and
analyzed.
\item[] Discover the equation that relates an object's kinetic energy to its mass $m$
and speed $v$.
\end{goals}

\labintroduction

What \emph{is} energy? It's hard to give a pithy, clear definition.
In a published lecture, physicist Richard Feynman wrote, ``It is
important to realize that in physics today, we have no knowledge of
what energy \emph{is}.'' Conservation of energy, he wrote, ``states that there is
a certain quantity, which we call energy, that does not change in the manifold
changes which nature undergoes... It is not a description of a mechanism, or anything
concrete; it is just a strange fact that we can calculate some number and when we finish
watching nature go through her tricks and calculate the number again, it is the same. (Something
like the bishop on a red square, and after a number of moves --- details unknown --- it is still on
some red square...)''\index{Feynman, Richard}\index{energy!conservation of!lab}\index{energy!kinetic!lab}%
\index{energy!kinetic!dependence on speed}\index{kinetic energy!lab}\index{kinetic energy!dependence on speed}


In fact, all the conserved quantities have this elusive quality, but it's just more obvious
when it comes to energy. Nineteenth-century physicists thought they knew what momentum was,
but they found out later that there was a less obvious form of it, which they had left out
of their definition: light carries momentum, just not enough to notice in everyday life.
Twentieth-century physicists thought they knew what mass was, but recent astronomical observations
have shown that 95\% of the universe's mass is in the form of ``dark matter,'' which isn't really
matter at all, in the usual sense of protons, neutrons, and electrons.

Mass, momentum, and energy are not things that were revealed to physicists centuries ago on stone tablets.
Physicists had to determine \emph{by experiment} what forms they took, and what mathematical
rules to use for calculating each of these forms. To see how this open-ended process works, we're
going to pretend that we only know about gravitational energy, and see how we
can extend energy to include a new form, kinetic energy.

We already know about gravitational energy, which is useful all by itself.
For instance, if two children are balanced on a see-saw, the total gravitational energy remains the same as one
goes up and the other goes down. What if they're unbalanced? If the heavier child sits down first,
the lighter one will not be able to budge the see-saw by sitting down on the other end. Again, our
theory works: motion is impossible in this situation, because energy would not be conserved: if the
heavier child went up and the lighter child went down, the total energy would not stay the same.

But nothing is as sad as a beautiful theory confronted with an ugly fact. If the lighter child gets
on first, and then the heavier one, we \emph{do} get motion. Since gravitational energy is the only form of energy
we know about, our theory is violated. As the heavy child falls, and the light one rises an equal
distance, we are losing total gravitational energy. The only way to fix our broken theory is to notice that in this
new process, unlike the previous ones, there is a change in speed. We therefore hypothesize that there
is some new form of energy, which is possessed by objects in motion. The net loss in gravitational is, we
guess, canceled by a gain in total motion-of-energy, which we decide to call kinetic energy.

\labfig{lab-inclined-plane}

\pagebreak[4]

\labobservations


\labpart{Speed dependence}

\labsubsubsection{Setting up the photogate}
This new form of energy depends on motion, but how exactly? To find out,
you'll use the air track apparatus shown in the figure.
The speed of the cart at any given point can be measured as
follows.  The photogate consists of a light and a sensor on
opposite sides of the track.  When the cart passes by, the
cardboard vane on top blocks the light momentarily, keeping
light from getting to the sensor.  The computer detects the
electric signal from the sensor, and records the amount of
time for which the photogate was blocked.  Given
the time, you can determine the speed that cart had
when it passed through the photogate.

The point of the air track is to eliminate friction, and
you need to check that friction has really been eliminated.
First, level the track by adjusting the feet. When the track is
level, the cart should not accelerate in either direction
when released from rest. (The track may not be perfectly
flat. For instance, if it's a little bowed in the middle, you
may find that even when the track is leveled as well as possible,
the cart always accelerates very gently toward the center.)
Once you've leveled the track, you can check for friction by
setting the cart in motion at a very low speed. If there is friction,
the cart will tend to slow down perceptibly regardless of which
direction you start it going. If there's no friction, then the
cart will only speed up or slow down very gradually because of imperfections
in the leveling or straightness of the track, and the results
will depend on the direction of motion.

Here's how to get the photogate running.
Make sure the interface box is turned on before you
boot up the computer. Plug the photogate into DG1 on the
interface box.
From the Start menu at the lower left corner of the screen,
run Logger Pro (in Programs$>$Vernier Software).
Make sure that the interface box is plugged into COM1 (the first
COM port) at the
back of the computer, not COM2.
If the computer presents you with a dialog box saying ``Set
Up Interface,''  choose COM1.
(If it complains that it can't find the port, you may be able
to fix the problem if you quit Logger Pro, power the interface off
and on again, and then get back in Logger Pro and try again.)
From the File menu, do Open, and locate the setup file you need:\\
   Probes \& Sensors $>$ Photogate $>$ One Gate Timer

If there is no button for collecting data, it's because the
interface box wasn't turned on when you booted up. Reboot.

At this point, you can test whether the photogate is working by
blocking it with your hand for a certain number of seconds. The
time should read out in the spreadsheet window under the Delta-T
column. (``Delta,'' the Greek uppercase letter $\Delta$, is a notation
meaning ``the change in,'' i.e., you're measuring the change in time
between one clock reading and another.)

You may find that the software rounds off too
severely. If you want more
than the three decimal places it offers by default in the
Delta-T column. To fix this, double-click on the title of
the Delta-T column, and select a greater number of
significant figures.   

The software will also give you a column in the spreadsheet labeled
V for velocity.\footnote{Velocity and speed are almost synonyms in
physics, and similar algebra notation, $v$ or \textit{\textbf{v}}, is used for both. There is
a technical distinction, which is that for motion in one dimension,
a number giving a velocity includes
a plus or minus sign giving direction information, while a speed is always
positive by definition.} This information will be \emph{incorrect} unless
you've told the software the width of the vane.\index{velocity!as opposed to speed}\index{speed!as opposed to velocity}
In fact, it generally won't even be necessary to calculate speeds in this
lab, because you'll be dealing only with ratios. For instance, if it takes
half as much time for the cart to get through the gate, then the speed must have
been twice as great.

\labsubsubsection{Measuring the speed dependence}

Position the photogate near the bottom, and release the cart from
a short distance (say 20 cm) up the slope. You can read off the time
from the computer. Think for a second about the order of magnitude of
this time. Does it make sense if it's supposed to tell you how long the cart
took to get from the release point to the photogate, or is it telling
you how long the vane took to pass through the photogate?

Since the cart accelerates on its way down the slope,
we expect that if we release it from higher up the slope, it will pass
through the photogate faster, and the time measured on the computer will
be shorter. Now imagine --- \emph{don't do this yet} --- that you
release the cart from farther upslope, searching by trial and error for
a release point that will result in double the speed, corresponding to half
the time on the computer. How many times farther upslope do you think you
will have to release it from?
Discuss this with your group, form a hypothesis,
and write it down here:  \_\_\_\_\_\_\_\_\_\_\_\_\_\_\_\_\_\_\_\_\_\_\_\_
Discuss this with your instructor before going on.

A couple of hints: (1) If you air pump has a knob that varies the speed of
the air, make sure to put it on its highest setting. (2) Don't turn on the
air and just let the cart lift off and start moving by itself. If you do
this, the cart will be dragging at first, and you'll get bad data.

OK, now carry out the experiment. Once you found the correct release point,
how many times greater was the gravitational energy consumed, compared to what was consumed
in your original setup? What does this tell about the amount of KE released?
Summarizing, how does KE seem to relate to speed? Discuss with your
instructor how to write this relationship as a proportionality.

\labpart{Mass dependence}
Now explore the dependence of kinetic energy on mass, by releasing the small
cart and the large one from the same distance upslope. The large cart has
double the mass, so how many times greater is the gravitational energy it consumes and
turns into KE? Compare the two times on the photogate, making sure that
the vanes on the two carts have the same width. How do the two velocities
compare? What does this tell you about how KE depends on mass?
Discuss with your
instructor how to write the dependence of KE on $m$ and $v$ as a proportionality.

\labpart{Reversing the motion}
Candles burn out. A bouncing ball eventually stops bouncing. Everything seems
to run down naturally. By analogy, suppose you shove the cart gently uphill,
so that is passes through the photogate, comes to a stop, and then slides back
down and passes through the photogate again on the way down. Form a hypothesis
about what you'll observe when you compare the two times measured on the
computer. Write down your hypothesis and show it to your instructor.
Hypothesis: \_\_\_\_\_\_\_\_\_\_\_\_\_\_\_\_\_\_\_\_\_\_\_

Now try it. How does this relate to the way conservation laws work?

\enlargethispage{-\baselineskip}

\labpart{Changing the path}
Suppose you release the cart on the air track from a certain height, $h$, and
measure its speed as it passes through the photogate.
(Note that $h$ is different from --- is less than --- the distance measured along
the slope.) Now imagine that you replace the air track setup with a pendulum,
flipping the photogate upside-down to form a U. The pendulum swings along an
arc of a circle, not a straight line. Imagine that you release the pendulum bob
so that its center will drop through the same height, $h$, as the cart did.
Because the bob is traveling along a curved path, it will move farther --- it
isn't traveling ``as the crow flies.'' What do you think you will observe about
the velocity of the pendulum bob compared to that of the cart? Does it matter
that they differ in mass? Try it.

\labfig{lab-two-paths}

Notes:
\begin{itemize}
\item[] The bob's diameter is
same as the width of the vane.

\item[] The point here is to compare two different paths, but an arc of
a circle that covers a sufficiently small angle is nearly a straight line.
To get a good test, you'll want to use an arc covering the greatest possible
angle.

\item[] Every atom of the cart moves an equal distance in an equal amount of
time, but that's not true for the bob, so identical atoms in different parts of
the bob will contribute different amounts of kinetic energy. The parts of the bob farthest from the
center of the circle are going faster than the parts nearer the center. To minimize
this ambiguity, you want the string to be fairly long compared to the size of
the bob. Also, which point on the bob is most representative of the whole thing?

\item[] Similarly, not every atom loses the same amount of gravitational energy,
since they don't all drop through the same height. Similar considerations apply.

\item[] For the reasons described above, you want a fairly long string and a fairly
long arc. The result is that the bob will drop through a big height. However, it may be
awkward to match this great height using the air track, so you may need to compromise
a little.

\end{itemize}

\enlargethispage{-\baselineskip}

How do your results relate to the way conservation laws work?

\labpart{Are we done?}
Now release the pendulum, and let it swing freely back and forth many times,
passing through the photogate twice on each cycle. What do you observe over
many cycles as you watch the computer print out the list of numbers? What
does this tell you? Discuss this with your instructor.

\labsection{Writeup}
In your writeup, one of the most important results you'll summarize is
the outcome of parts A and B: how kinetic energy depends on mass $m$ and speed $v$.
Based only on this experiment, all you could get would be a proportionality, not an
actual equation. The difference is that the actual equation would have some numerical
factor out in front that would make the equation consistent with the system of units
you're using. Similarly, people in different countries use different currencies, so
although they'd agree that the price of a gold bar was directly proportional to its
mass, one person would say it was this many dollars per ounce, while the other would
state it as so many euros per gram.

In this book we're using metric units, and I've presented the energy scale as being
based on the amount of heat required to raise the temperature of a certain amount of
water by a certain amount. In fact, the metric system was designed so that the relationship
between kinetic energy, mass, and speed would have a nice simple numerical factor out
in front, and I want you to find that numerical factor. To find it, use the fact that
a one-kilogram object moving at a speed of one meter per second has a kinetic energy
of exactly $1/2$ of a joule.

\end{lab}

%========================================== self-check answers ===============================================


\startscanswers{ch:energy}
\scanshdr{conservation} A conservation law in physics says that the total
amount of something always remains the same. You can't get rid of it even if you want to.
\scanshdr{firefighteronpole}{Her gravitational energy is being transformed
into heat energy. Friction heats up her body and the pole.}
\scanshdr{dropfeather}{The feather experiences air resistance, which is a form of friction.
Friction produces heat, and that's the missing form of energy. In a vacuum chamber, the
feather will not fall any more slowly than any other object.}

%========================================== toc decoration ===============================================


\addtocontents{toc}{\protect\figureintoc{poolskater}}
