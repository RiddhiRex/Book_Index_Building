\mychapterwithopener{dizzy}{Why do I get dizzy? Am I really spinning, or is the world
going around me? Humans are naturally curious about the universe they live in.}{The Rules of the Rules}\label{ch:symmetry}

Since birth, you've wanted to discover things. You started out by putting every available object in your
mouth. Later you began asking the grownups all those ``why'' questions. None of this makes you unique ---
humans are naturally curious animals. What's unusual is that you've decided
to take a physics course. There are easier ways to satisfy a science requirement, so evidently you're
one of those uncommon people who has retained the habit of curiosity into adulthood, and
you're willing to tackle a subject that requires sustained intellectual effort. Bravo!

A reward of curiosity is that as you learn more, things get simpler. ``Mommy, why do you have to go
to work?'' ``Daddy, why do you need keys to make the car go?'' ``Grandma, why can't I have that toy?''
Eventually you learned that questions like these, which as a child you thought to be unrelated,
were actually closely connected: they all had to do with capitalism and property. 
As a scientific example, William Jones\index{Jones, William} announced in 1786
the discovery that many languages previously thought
to be unrelated were actually connected. Jones realized, for example, that there was a relationship
between the words
``bhratar,'' ``phrater,'' ``frater,'' and ``brother,'' which mean the same thing
in Sanskrit, Greek, Latin, and English. Many apparently unrelated languages of Europe and
India could thus be brought under the same roof and understood in a simple way.
For an even more dramatic example, imagine trying to learn chemistry hundreds of years
ago, before anyone had discovered the periodic table or even the existence of atoms.\index{atoms}
Chemistry has gotten a lot simpler since then!\index{chemistry}

Sometimes the subject gets simpler, but it takes a while for the textbooks to catch up.
For hundreds of years after Hindu mathematicians incorporated negative numbers into
algebra, European texts still avoided them, which meant that students had to endure
a lot of confusing mumbo jumbo when it came to solving an equation like $x+7=0$.
Physics has been getting simpler, but most physics books still haven't caught up.
(Can you detect the sales pitch here?) The newer, simpler way of understanding physics
involves symmetry.

\pagebreak[4]

\mysection[0]{Symmetry}\index{symmetry}

\margup{-12mm}{\fig{noether}{Emmy Noether (1882-1935). The daughter of a prominent German mathematician, she did not show
any early precocity at mathematics --- as a teenager she was more interested in music and dancing.
She received her doctorate in 1907 and rapidly built a world-wide reputation, but the University
of G\"{o}ttingen refused to let her teach, and her colleague Hilbert had to advertise her courses in the
university's catalog under his own name. A long controversy ensued, with her opponents asking
what the country's soldiers would think when they returned home and were expected to learn
at the feet of a woman. Allowing her on the faculty would also mean letting her vote
in the academic senate. Said Hilbert, ``I do not see that the sex of the candidate is against
her admission as a privatdozent [instructor]. After all, the university senate is not a bathhouse.''
She was finally admitted to the faculty in 1919. A Jew, Noether fled Germany in 1933 and
joined the faculty at Bryn Mawr in the U.S.}}
%
The concept of symmetry goes back to ancient times, but the deep link between
physics and symmetry was 
discovered by Emmy Noether.\index{Noether, Emmy}
What do we mean by symmetry?\index{symmetry!defined}\index{symmetry!mirror}\index{symmetry!reflection}\index{symmetry!180-degree rotation}
Figure \figref{singlesymmetries} shows
two examples. The galaxy has a symmetry because it looks the same when
you turn your book upside-down. The orchid has a different type of symmetry: it looks the
same in a mirror. Reflection and 180-degree rotation are examples of transformations, i.e.,
changes in which every point in space is systematically relocated to some other
place. We say that a thing has symmetry when transforming it doesn't change it.
As shown in figure \figref{moresymmetries}, some objects have more
than one symmetry, although most have none.

\widefig{singlesymmetries}{Two types of symmetries.}
\widefig{moresymmetries}{Most object have no symmetries. Some have more than one.}

\selfcheck{playingcards}{What symmetry is possessed by most of the designs in a deck of cards? Why are
they designed that way?}

\begin{eg}{Palindromes}\index{palindromes}
A palindrome is a sentence that is the same when you reverse it: \\
I maim nine men in Saginaw; wan, I gas nine men in Miami.
\end{eg}

\pagebreak[4]

\dqheader
\begin{dq}
What symmetries does a human have? Consider internal features,
external features, and behavior. 
If you woke up one morning after having been reflected, would
you be able to tell? Would you die? What if the rest of the world
had been reflected as well?
\end{dq}

\vfill

\mysection[4]{A Preview of Noether's Theorem}\index{Noether's theorem!rough statement}

\margup{-150mm}{\fig{skaters}{What will happen when the two ice skaters push off from each
other?}
\spacebetweenfigs
\fig{checkersinitial}{The starting position in checkers.}
\spacebetweenfigs
\fig{checkerslater}{The board after two moves.}}
How does symmetry relate to physics? Long before Noether's work, it had been
recognized that some physical systems had symmetry, and their symmetries could
be helpful for predicting their behavior. If the skaters in
figure \figref{skaters} have equal masses, symmetry tells us that they will
move away from each other at equal speeds after they push off. 
The one on the right looks bigger, however, so the symmetry argument doesn't
quite work. If you look at the world around you, you will see many approximate
examples of symmetry, but none that are perfect. Most things have
no symmetry at all. Until Noether's work, that was the whole story.
Symmetry was on the sidelines of physics.\label{original-skaters}

Noether's approach was different. The universe is made out of particles, and
these particles are like the players on a soccer field
or the pieces on a checkerboard. The arrangement of the players on the
soccer field normally has no symmetry at all. The symmetry is in the rules:
the rules apply equally to both sides. Likewise, the physical arrangement
of the checkers on the board in figure \figref{checkersinitial} has
180-degree rotation symmetry, but this is spoiled in figure \figref{checkerslater}
after a couple of moves. We don't care about the asymmetry of the pieces.
In Noether's approach, what's important is the symmetry of the rules.
If we think of the checkerboard as a little universe, then these rules are like
the laws of physics, and their symmetry allows us to predict certain things
about how the universe will behave. For instance, suppose we balanced the board
carefully on a knife edge running from left to right below its centerline. The
position in figure \figref{checkersinitial} balances, and so does the one
in figure \figref{checkerslater}. The rules required both red and black to
move one piece diagonally forward one step, so we were guaranteed that after
each side had made one move, the setup would balance again.\footnote{This
symmetry won't continue indefinitely, because at some point one player will
jump one of the other player's pieces, or get a king and make a backwards
move. That just shows that a game like checkers is an imperfect metaphor for
the laws of physics. The particles in the universe don't take turns moving,
so we don't have situations where one particle sits still while another
one ``jumps'' it. It is possible for a particle of matter and a particle of
antimatter to annihilate one another --- the process is probably occurring in
the room you're in right now, due to natural radioactivity --- but neither particle
exists afterwards, so the symmetry is more perfect than in checkers.
The laws of physics are also deterministic; there is
no choice involved, as in a game.}

Noether's greatest achievement was a principle known as Noether's theorem.
We are not yet ready to state Noether's theorem exactly, but roughly speaking,
here's what it says: The laws of physics have to be the way they are because
of symmetry.

\mysection[0]{What Are The Symmetries?}
What are the actual symmetries of the laws of physics? It's tempting to try to
determine them by pure reason, or by aesthetic arguments. Why, for example, would
God have chosen laws of physics that didn't treat right and left the same way? That
would seem ugly. The trouble with this approach is that it doesn't work.

For example, prehistoric peoples observed the rising and setting of the sun, the moon,
the stars, and the four naked-eye planets. They all appeared to be going in circles,
and a circle is a very symmetric shape: it remains the same under rotation through
any angle at all. It became accepted dogma among the ancient astronomers that these
heavenly bodies were attached to spinning crystal spheres. When careful observations
showed that the motion of the planets wasn't quite circular, they patched things up by imagining
smaller crystal spheres riding on the big ones. This bias toward spheres and circles
was hard to shake because the symmetry of the shapes was so appealing. The astronomer
Johannes Kepler (1571-1630) inherited from his predecessor Tycho Brahe (1546-1601)\index{Kepler, Johannes}\index{Brahe, Tycho}
a set of the best observations ever made of the motions of the planets. Kepler
labored for years trying to make up a set of spheres riding on spheres that would
fit the data, but because the data were so accurate, he finally realized what nobody
could have known based on the older, less precise observations: it simply wasn't
possible. Reluctantly, Kepler gave up his mystical reverence for the symmetry of the
circle. He eventually realized that the planets' orbits were ovals of a specific
mathematical type called an ellipse. The new observations showed that the laws of physics
were less symmetric than everyone had believed.

\margup{-145mm}{\fig{startrails}{Due to the earth's rotation, the stars appear to go in circles. In this
time-exposure photograph, each star makes an arc.}
\spacebetweenfigs
\fig{checkerstranslational}{A chess board has a kind of translational symmetry: it
looks the same if we slide it one square over and one square up.}
\spacebetweenfigs
\fig{sodastraw}{The soda straw has translational symmetry. The flea exploring along its length
doesn't see anything different from one location to another.}
}
Sometimes experiments show that physics is \emph{more} symmetric than expected. One good
example of this is translational symmetry.\index{symmetry!translational} A translation is a type of transformation
in which we slide everything without rotating it, as in figure \figref{checkerstranslational},
where we can slide the chess board so that the black squares are again in the places previously
occupied by black squares.\footnote{The chess board lacks complete translational symmetry
because it has edges. As far as we know, the laws of physics don't specify that there are
edges to the universe beyond which nothing can go. However, this is different from the
question of whether the universe has infinite volume. We can easily make a chessboard that
is finite but has no edges. We simply wrap the right and left edges around to form a tube,
and then bend the tube into a doughnut. We still don't know with certainty whether the
universe is finite or infinite, although the latest data seem to show it's infinite.
Einstein's theory of general relativity allows either possibility.}
The ancient Greek philosopher Aristotle\index{Aristotle} believed that the rules were different in some
parts of the universe than in others. In modern terminology, we say that he didn't
believe in translational symmetry. When you drop a rock, it falls. Aristotle explained
this by saying that the rock was trying to go back to its ``natural'' place, which is
the surface of the earth. He applied the same kind of explanation to rising smoke: it rises
because it wants to go to its own natural place, which is higher up. In Aristotle's
theory, different parts of the universe had their own special characteristics.
Only after an interval of two thousand years was the true translational symmetry of
the laws of physics uncovered by Isaac Newton.\index{Newton, Isaac} In Newton's theory of gravity,
a rock falls because every atom in the universe is attracted to every other atom.
The rock's atoms are attracted to the planet earth's atoms. We don't prefer Newton's
version just because it sounds better. Aristotle was proved wrong by
experiments. The original evidence was indirect, but we have more straightforward
proof now. If Aristotle had been right, the huge boulder in figure \figref{schmidtonmoon}
would long since have fallen to its ``natural'' place on the surface of our planet
(and so would the astronaut!).\label{space-translation-symmetry}

\widefigsidecaption{schmidtonmoon}{Astronaut Harrison Schmidt on the moon in 1972.}

Translational symmetry is also deeply embedded in the way we practice the scientific
method. One of the assumptions of the scientific method is that experiments should
be reproducible. For example, a group at Berkeley recently claimed to have produced
three atoms of a new element, with atomic number 118. Other labs, however, were
unable to reproduce the experiment, and eventually suspicious members of the Berkeley
team checked and found that one of their own scientists had fabricated the data.
Although the episode (and another case of fraud at Bell Laboratories around the same
time) caused considerable editorializing about what might be wrong with the
scientific profession, I see it as a textbook example of how the scientific method
is supposed to work, since the fraud was eventually discovered. A basic assumption
here is that scientists in different places should be able to get the same
results. If translational symmetry was violated, then the results might be different
because the laws of physics were different in different places. The assumption of
translational symmetry is so deeply ingrained that normally it doesn't
even occur to us that we were making it. When engineers design a space probe to
go to Mars, they don't even stop to ask themselves whether the laws of physics
are the same on Mars as on earth.

\dqheader

\begin{dq}\label{dq:ozma}
Imagine that you establish two-way radio communication with aliens. You laboriously
teach each other your languages, e.g., by sending two beeps followed by the word
``two.'' However, neither of
you is able to figure out exactly where the other's planet is, and you can't
come up with any celestial landmarks that you both recognize. Can you communicate
the definition of the terms ``right'' and ``left'' to them? The wonderful popular science
writer Martin Gardner proposes calling this the ``Ozma problem.'' (The name comes
from the Ozma project, which was the first serious attempt to detect signals from
aliens using radio telescopes. The Ozma project was in turn named after a character
in one of L. Frank Baum's Oz stories.) In general, every
symmetry of the laws of physics can be stated as an Ozma problem.\index{Ozma problem!for left and right}
\end{dq}

%===============================================================================
%===============================================================================

\vfill\pagebreak[4]
\noindent\includegraphics[width=100mm]{\chapdir/figs/flowers}\label{fig:flowers}
\marg{%
\raisebox{100mm}[0mm][0mm]{%
\parbox{52mm}{
\noindent\formatlikecaption{These flowers are referred to in homework problems
\ref{hw:flowersa} and \ref{hw:flowersb}.}\\
\noindent\includegraphics[width=52mm]{\chapdir/figs/flowernames}
}}}

\vfill\pagebreak[4]
\begin{hwsection}

\hwnote{%
Problems \ref{hw:flowersa} and \ref{hw:flowersb} refer to the photos of flowers on
page \pageref{fig:flowers}. Since the flowers are living things, they don't have
exact, perfect mathematical symmetry. Just think in terms of approximate symmetries.
}

\begin{hw}{flowersa}
(a) Which of the flowers shown in the photos have reflection symmetry but not 180-degree
rotation symmetry? \hwendpart
(b) Which have 180-degree rotation symmetry but not reflection symmetry?\hwendpart
(c) Which have both\hwendpart
(d) Which have neither?\hwendpart
Note that in flowers 1 and 2, the lobes of
the petals overlap in a clockwise or counterclockwise screw pattern.
You can tell from the photo that flower 1 has a curved tube. Flower 2 doesn't have a curved tube.
\end{hw}

\begin{hw}{flowersb}
In the text, I've only discussed rotational symmetry with an angle of 180 degrees.
Some of the flowers in the photos have symmetry with respect to other angles. Discuss
these.
\end{hw}

\begin{hw}[2]{flowersresearch}
The following are questions about the symmetries of plants that you can try to
answer by collecting data at an arboretum, nursery, botanical garden, or florist.
(You could also websurf, but it wouldn't be as enjoyable.) You probably won't be
able to answer all of them. You can't do this problem without actually going out
and collecting detailed data; you'll have to turn in the data (drawings, notes
on which plants you looked at, etc.) and then base your conclusions on your data.
\begin{itemize}
 \item[] Symmetry of flowers is an easy way to classify plants. Is it also a good way?
  To be a good way, it should correspond to evolutionary relationships, and it
  should therefore correlate with other features of plants. Another feature
  that's easy to check is leaf structure: are the fibers in the leaves all
  parallel (e.g., grass), or do they branch out (e.g., a maple). Does leaf structure
  seem to correlate at all with flower symmetry?
 \item[] The photos on page \pageref{fig:flowers} include some flowers whose petals or
  petal-lobes overlap in a pattern like a clockwise or counterclockwise screw. When this
  happens, how systematic is the pattern of overlapping? Do you observe right-handed and
  left-handed screw-patterns in different flowers on the same plant? In different plants
  that are genetically identical (e.g., grown from cuttings from the same parent) but
  have been exposed to different environments? In genetically different plants of the same
  species?
 \item[] Can you find any plants in which the arrangement of the leaves follows a definite
  pattern, but lacks reflection symmetry?
\end{itemize}
\end{hw}

\begin{hw}{noethernotobjects}
Noether's theorem refers to symmetries of the laws of physics, not symmetries of
objects.
Which of the following do you think could qualify as a law of physics, and which
are mere facts about objects? In other words, which ones are \emph{not} true
in some situations, at some times, on different planets, etc? They are all true
where I live!
\begin{enumerate}
 \item The sun rises in the east and sets in the west.
 \item High tide occurs when the moon is overhead or underfoot, and low tide
   when it's on the horizon.
 \item Inheritance works through genes, so an acquired trait can't be inherited.
 \item In a chemical reaction, if you weigh all the products, the total is the same
   as what you started with.
 \item A gas compressed to half its original volume will have twice its original pressure
	(assuming the temperature is the same).
\end{enumerate}
In each case, explain your reasoning.
\end{hw}

\begin{hw}{ninetydegreerotation}
If an object has 90-degree rotation symmetry, what other symmetries must it have as well?\\
\end{hw}

\begin{hw}{onethirtyfivedegreerotation}
Someone describes an object that has symmetry under 135-degree rotation (3/8 of a circle).
What's a simpler way to describe the same symmetry? (Hint: Draw a design on a piece of
paper, then trace it onto another piece of paper. Rotate the top piece of paper, then
copy the new design. Keep going. What happens?)
\end{hw}

\begin{hw}{oneeightyandreflection}
(a) Give an example of an object that has 180-degree rotation symmetry, and also has reflection
symmetry. \\
(b) Give an example with symmetry under 180-degree rotation, but \emph{not} under reflection.
\end{hw}

\begin{hw}{elephant}
Suppose someone tells you that the reason the Ozma problem for left and right is
difficult is because you can't get together with the aliens and show them what you're
referring to. Is this correct? How is this different from trying to describe an elephant
over the radio to someone who's never seen an elephant or a picture of one?
\end{hw}

\end{hwsection}


%========================================== labs ===============================================

\begin{lab}{Scaling}\index{scaling}

\apparatus
\equip{paper and card stock}\\
\equip{ruler}\\
\equip{scissors}

\goal{Find out whether the laws of physics have scaling symmetry.}

\labintroduction

From Gulliver to Godzilla, people have always been fascinated with
scaling.\index{Gulliver}\index{Godzilla}\index{Galileo}
Gulliver's large size relative to the Lilliputians obviously had some strong
implications for the story. But is it only \emph{relative} size that matters?
In other words, if you woke up tomorrow, and both you and your house had been
shrunk to half their previous size, would you be able to tell before stepping
out the door? Galileo was the first to realize that this type of question was
important, and that the answer could only be found by experiments, not by looking
in dusty old books. In his book \emph{The Two New Sciences}, he illustrated
the question using the idea of a long wooden plank, supported at one end, that
was just barely strong enough to keep from breaking due to gravity. The testable
question he then posed was whether this just-barely-strong-enough plank would
still have the just-barely-strong-enough property if you scaled it up or down,
i.e., if you multiplied all its dimensions --- length, width, and height ---
by the same number.

\labfigcaption{lab-galileo-board}{Galileo's illustration of his idea.}

You're going to test the same thing in lab, using the slightly less picturesque
apparatus shown in the photo: strips of paper. The paper bends rather than breaking,
but by looking at how much it droops, you can see how able it is to support its
own weight.
The idea is to cut out different
strips of paper that have the same proportions, but different sizes. If the
laws of physics are symmetric with respect to scaling, then they should all
droop the same amount. Note that it's important to scale all three dimensions
consistently, so you have to use thicker paper for your bigger strips and
thinner paper for the smaller ones. Paper only comes in certain thicknesses,
so you'll have to determine the widths and lengths of your strips based on
the thicknesses of the different types of paper you have to work with. 
In the U.S., some common thicknesses of paper and card-stock are
78, 90, 145, and 200 grams per square meter.\footnote{A student at Ohlone College,
using the same brand of paper I use at Fullerton College,
noticed that the numbers given on the packaging in units of pounds
do not correspond at all closely to the thickness or weight of the paper. The
densities are also a little different, but not too different, so it's not such
a bad assumption to assume that weight relates directly to thickness.}
We'll assume that these numbers also correspond
to thicknesses. For instance, 200 is about 2.56 times greater than 78, so the
strip you cut from the heaviest card stock should have a length and
width that are 2.56 times greater than the corresponding dimensions of
the strip you make from the lightest paper.

\labfig{lab-paper-strip}

\labsection{To Think About Before Lab}

1. If the laws of physics are symmetric with respect to
scaling, would each strip droop by the same number of centimeters,
or by the same angle? In other words, how should you choose to
define and measure the ``droop?''

2. If you find that all the strips have the same droop,
that's evidence for scaling symmetry, and if you find that they droop
different amounts, that's evidence against it. Would either observation
amount to a proof? What if some experiments showed scaling symmetry and
others didn't?


\end{lab}


%========================================== self-check answers ===============================================
\startscanswers{ch:symmetry}
\scanshdr{playingcards}{They have 180-degree rotation symmetry. They're designed that way
so that when you pick up your hand, it doesn't matter which way each card is turned.}

%========================================== toc decoration ===============================================
\addtocontents{toc}{\protect\figureintoc{schmidtonmoon}}
