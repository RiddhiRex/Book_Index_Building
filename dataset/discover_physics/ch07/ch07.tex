\mychapterwithopener{sunspot}{This sunspot is a product of the sun's magnetic fields. The
darkest region in the center is about the size of our planet.}{Electricity and Magnetism}\label{ch:em}\label{sunspotphoto}\index{sunspots}
\mysection[0]{Electrical Interactions}
\epigraph{Newton was not the first of the age of reason. He was the
	last of the magicians.}{John Maynard Keynes}\index{Newton, Isaac}\index{Keynes, John Maynard}

	Keynes' language isn't as figurative as you might think. Newton
	had a lifelong obsession with alchemy, a pseudoscience that bears
	the same relationship to chemistry that astrology has
	to astronomy.\index{astrology}\index{Galileo!astrology}\index{Newton!astrology myth}
\footnote{There's an urban folktale that Newton also practiced astrology. Wrong!
Newton wrote that as a young student, he had read a book on astrology, and was
``soon convinced of the vanity \& emptiness of the pretended science of Judicial astrology''
(Whiteside, Hoskin, and Prag, eds., 
\emph{The Mathematical Papers of Isaac Newton}
Cambridge University Press, Cambridge, 1967-81, vol. 1, pp. 15-19).
Galileo did calculate horoscopes for money, and Newton was born the same year Galileo died,
1642, so this year represents a dividing line in the history of the astrological
supersition --- since Newton's lifetime, belief in astrology has become essentially
extinct among physicists and astronomers. It's no coincidence that the dividing
line is represented by Galileo and Newton.
Galileo had pioneered the use of the
scientific method to study the heavens, while Newton's greatest achievement
was to show that the motion of the planets could be explained using his law of
gravity. The success of this naturalistic description made it clear that it was silly
to look for supernatural links between the skies and human concerns.
Astrology has also failed every empirical test; a particularly well-constructed
study by Rob Nanninga is described at http://home.planet.nl/$\sim$skepsis/astrot.html.
}
	To the modern science educator, this
	may seem an embarrassment, a distraction from
	his main achievement, which was the creation the modern science of
	mechanics. To Newton, however, his alchemical researches
	were naturally related to his investigations of force and
	motion. What was radical about Newton's analysis of motion
	was its universality: it succeeded in describing both the
	heavens and the earth with the same equations, whereas
	previously it had been assumed that the sun, moon, stars,
	and planets were fundamentally different from earthly
	objects. But Newton realized that if science was to describe
	all of nature in a unified way, it was not enough to unite
	the human scale with the scale of the cosmos: he would not
	be satisfied until he fit the microscopic universe into
	the picture as well.

\begin{envsubsection}{Newton's quest}
	It shouldn't surprise us that Newton failed. Although he
	was a firm believer in the existence of atoms, there was no
	more experimental evidence for their existence than there
	had been when the ancient Greeks first posited them on
	purely philosophical grounds. \index{alchemy}Alchemy labored
	under a tradition of secrecy and mysticism. Newton had
	already almost single-handedly transformed the fuzzyheaded
	field of ``natural philosophy'' into something we would
	recognize as the modern science of physics, and it would be
	unjust to criticize him for failing to change alchemy into
	modern chemistry as well. The time was not ripe. The
	microscope was a new invention, and it was cutting-edge
	science when Newton's contemporary \index{Hooke}Hooke
	discovered that living things were made out of cells.

	Nevertheless it will be instructive to pick up Newton's
	train of thought and see where it leads us with the benefit
	of modern hindsight. In uniting the human and cosmic scales
	of existence, he had reimagined both as stages on which the
	actors were objects (trees and houses, planets and stars)
	that interacted through attractions and repulsions. He was
	already convinced that the objects inhabiting the microworld
	were atoms, so it remained only to determine what kinds of
	forces they exerted on each other.

	His next insight was no less brilliant for his inability to
	bring it to fruition. He realized that the many human-scale
	forces --- friction, sticky forces, the forces that
	keep objects from occupying the same space, and so on ---
	must all simply be expressions of a more fundamental force
	acting between atoms. Tape sticks to paper because the atoms
	in the tape attract the atoms in the paper. My house doesn't
	fall to the center of the earth because its atoms repel the
	atoms of the dirt under it.

	Here he got stuck. It was tempting to think that the atomic
	force was a form of gravity, which he knew to be universal,
	fundamental, and mathematically simple. Gravity, however, is
	always attractive, so how could he use it to explain the
	existence of both attractive and repulsive atomic forces?
	The gravitational force between objects of ordinary size is
	also extremely small, which is why we never notice cars and
	houses attracting us gravitationally. It would be hard to
	understand how gravity could be responsible for anything as
	vigorous as the beating of a heart or the explosion of
	gunpowder. Newton went on to write a million words of
	alchemical notes filled with speculation about some other
	force, perhaps a ``divine force'' or ``vegetative force''
	that would for example be carried by the sperm to the egg.

	Luckily, we now know enough to investigate a different
	suspect as a candidate for the atomic force: electricity.
	\index{electric forces}Electrical forces are often observed
	between objects that have been prepared by rubbing (or other
	surface interactions), for instance when clothes rub against
	each other in the dryer. 
	Electrical forces are similar in certain ways to
	gravity, the other force that we already know to be fundamental:

	\begin{itemize}
	\item Electrical forces are \emph{universal}. Although some
	substances, such as fur, rubber, and plastic, respond more
	strongly to electrical preparation than others, all matter
	participates in electrical forces to some degree. There is
	no such thing as a ``nonelectric'' substance. Matter is both
	inherently gravitational and inherently electrical.

	\item Experiments show that the electrical force, like the
	gravitational force, is an \emph{inverse square} force. That
	is, the electrical force between two spheres is proportional
	to $1/r^2$, where $r$ is the center-to-center distance between them.
	\end{itemize}
\end{envsubsection}
%
\begin{envsubsection}{Charge and electric field}
	``\index{charge}Charge'' is the technical term used to
	indicate that an object has been prepared so as to
	participate in electrical forces. This is to be distinguished
	from the common usage, in which the term is used indiscriminately
	for anything electrical. For example, although we speak
	colloquially of ``charging'' a battery, you may easily
	verify that a battery has no charge in the technical sense,
	e.g., it does not exert any electrical force on a piece of
	tape that has been prepared as described in the previous section.
	The metric unit of charge is the\index{charge!coulomb unit}\index{coulomb (unit)}
	coulomb (rhymes with ``drool on''), defined as follows:
	one coulomb (C) is the amount of charge such that
	a force of $9.0\times10^9$ newtons\footnote{Don't memorize this number.} occurs between two pointlike
	objects with charges of 1 coulomb separated by a distance of 1 meter.
	Nine billion newtons is a tremendous amount of force, so we can see that
	the amount of charge on your socks when they come out of the dryer must be
	a tiny fraction of a coulomb.

	Just as we think of a planet as being surrounded by a gravitational field,
	we can imagine an electric field surrounding your sock.\index{electric field}\index{field!electric}
	When the air crackles and your hair stands on end in an electrical storm,
	you're experiencing an electric field.
	Charge plays the role in electrical interactions that is played by
	mass in gravitational interactions. The gravitational field has units
	of energy per meter per kilogram, so by analogy the electric field has
	units of energy per meter per coulomb.

	You've already investigated charge in lab, so I won't bore you by
	recapitulating the relevant facts normally presented in textbooks:
	the number of types of charge, the rules for attraction or repulsion,
	and the question of whether charge is conserved.\index{charge!number of types}\index{charge!conservation of}

\subsubsection{Quantization of charge and a charged particle model}
	One fact about charge that is not immediately apparent in ordinary
	electrical experiments is that it is quantized. When we say something
	is quantized, we mean that\index{charge!quantization of}
	it comes in a certain minimum unit. For instance, the U.S. currency
	is quantized in units of pennies; you can't write a check for half
	a penny. The quantization of charge makes sense if we imagine a model
	in which charge is carried by microscopic, identical particles. In the same
	way, a person who studied accounting but had never seen actual currency
	might hypothesize that people actually carried out monetary transactions by
	exchanging some physical object as a token --- the penny.

	\margup{-54.5mm}{\fig{millikan}{A young Robert Millikan.}
	\spacebetweenfigs
	\begin{minipage}[t]{52mm}
	\begin{tabular}{ll}
					& charge\\
					& $/(1.64$\\
		charge (C)		& $\times10^{-19}\ \zu{C})$ \\
		\hline
		$1.970\times10^{-18}$	& $12.02$ \\
		$0.987\times10^{-18}$	& $6.02$ \\
		$2.773\times10^{-18}$	& $16.93$ \\
	\end{tabular}
	\docaption{A few samples of Millikan's data. The letter C stands
	for units of coulombs.} %
	\label{fig:millikandata} %
	\end{minipage}
        \spacebetweenfigs
        \fig{millikan-apparatus}{Millikan's oil drop experiment.}
	}
	Strong support for the charged-particle model came from a
	1911 experiment by physicist Robert \index{Millikan,
	Robert}Millikan at the University of Chicago. Consider a jet
	of droplets of perfume or some other liquid made by blowing
	it through a tiny pinhole. The droplets emerging from the
	pinhole must be smaller than the pinhole, and in fact most
	of them are even more microscopic than that, since the
	turbulent flow of air tends to break them up. Millikan
	reasoned that the droplets would acquire a little bit of
	electric charge as they rubbed against the channel through
	which they emerged, and if the charged-particle model of
	electricity was right, the charge might be split up among so
	many minuscule liquid drops that a single drop might have a
	total charge amounting to only a few charged
	particles.

	Millikan's ingenious apparatus was a small box with metal plates
	for its ceiling and floor. These plates could be electrically
	charged as needed. He sprayed a cloud of oil droplets into
	the space between the plates, and selected one drop through
	a microscope for study. First, with no charge on the plates,
	he would determine the drop's mass by letting it fall
	through the air and measuring its terminal velocity, i.e.,
	the velocity at which the force of air friction canceled out
	the force of gravity. The force of air drag on a slowly
	moving sphere had already been found by experiment, so he
	could determine the force of gravity on the drop, and therefore
	its mass.

	Next Millikan charged the metal plates, adjusting the amount
	of charge so as to exactly counteract gravity and levitate
	the drop. He then knew that the electric field and the magnetic
	field were making forces on the drop in equal directions, and
	canceling out; the gravitational energy the drop would lose by
	dropping one millimeter would be exactly canceled by the electrical
	energy it would gain.
	Since he knew the strengths of the fields, and
	also the mass of the drop, he could determine the drop's charge.

	Table \figref{millikandata} shows a few of the results from
	Millikan's 1911 paper. 
	Even a quick look at the
	data leads to the suspicion that the charges are not simply
	a series of random numbers. For instance, the second charge
	is almost exactly equal to half the first one. Millikan
	explained the observed charges as all being integer
	multiples of a single number, $1.64\times10^{-19}$ coulombs.
	(The modern value is $1.60\times10^{-19}$ coulombs. Don't memorize it!) In the
	second column, dividing by this constant gives numbers that
	are essentially integers, allowing for the random errors
	present in the experiment. Millikan states in his paper that
	these results were a

	\begin{quote}
	\ldots direct and tangible demonstration \ldots of the correctness of
	the view advanced many years ago and supported by evidence
	from many sources that all electrical charges, however
	produced, are exact multiples of one definite, elementary
	electrical charge, or in other words, that an electrical
	charge instead of being spread uniformly over the charged
	surface has a definite granular structure, consisting, in
	fact, of \ldots specks, or atoms of electricity, all precisely
	alike, peppered over the surface of the charged body.
	\end{quote}

	In other words, he had provided direct evidence for the
	charged-particle model of electricity and against models in
	which electricity was described as some sort of fluid. 

	\subsubsection{A historical note on Millikan's fraud}
	Very few undergraduate physics textbooks mention
	 the well-documented fact that although
	Millikan's conclusions were correct, he was guilty of scientific
	fraud. His technique was difficult and painstaking to perform, and his
	original notebooks, which have been preserved, show that the data were
	far less perfect than he claimed in his published scientific papers.
	In his publications, he stated categorically that every single oil
	drop observed had had a charge that was a multiple of the same basic unit, with no
	exceptions or omissions. But his notebooks are replete with notations
	such as ``beautiful data, keep,'' and ``bad run, throw out.'' Millikan,
	then, appears to have earned his Nobel Prize by advocating a correct
	position with dishonest descriptions of his data.
	
	Why do textbook
	authors fail to mention Millikan's fraud? It may be that they think
	students are too unsophisticated to correctly evaluate the
	implications of the fact that scientific fraud has sometimes existed
	and even been rewarded by the scientific establishment. Maybe they're
	afraid students will reason that fudging data is OK, since Millikan
	got the Nobel Prize for it. But falsifying history in the name of
	encouraging truthfulness is a little ironic. English
	teachers don't edit Shakespeare's tragedies so that the bad characters
	are always punished and the good ones never suffer! 	

\subsubsection{Agnosticism about the specific particles}\index{reductionism}\label{reductionism}
	One of the themes of this book has been the concept of a scientific
	model, and the idea that science never really deals with reality,
	only with models of it. The charged particle model of electricity
	does a good job of explaining quantization of charge, and it's natural
	to ask next what kinds of particles they are. This is the attitude known
	as reductionism: take everything apart until you get down to the building
	blocks. Many of the greatest accomplishment of physics have been
	due to reductionism, and for example if you take a look at the chapters of this
	book on energy and momentum, you'll see that their
	logical structure depends heavily on a
	reductionist theory, the theory that matter is made of atoms. However,
	it can also be beneficial sometimes to adopt an attitude that is the
	opposite of reductionism. That's what we'll do throughout this chapter
	when it comes to the question of what the charged particles really are.
	It turns out that we can understand all the important facts about
	electricity and magnetism without worrying at all about this issue.
\end{envsubsection}

\dqheader
\begin{dq}
In lab, you determined how many types of electrical charge there were,
and it's natural to want to invent names for the different ``flavors.''
Imagine, as in the discussion question on page
\pageref{dq:ozma} that you establish two-way radio communication with aliens
but you can't
come up with any celestial landmarks that you both recognize. Can you communicate
the definitions of the terms you've invented for the flavors of charge?
Could you tell if the aliens had gotten your English labels switched around?
This is another example of an Ozma problem, introduced
in discussion question \ref{dq:ozma} on page \pageref{dq:ozma}.\index{Ozma problem!for charge}
\end{dq}

\mysection{Circuits}
\begin{envsubsection}{Current}
\subsubsection{Unity of all types of electricity}
We're surrounded by things we've been \emph{told} are
``electrical,'' but it's far from obvious what they have in
common to justify being grouped together. What relationship
is there between the way socks cling together and the way a
battery lights a lightbulb? We have been told that both an
electric eel and our own brains are somehow electrical in
nature, but what do they have in common?

British physicist Michael \index{Faraday, Michael}
Faraday (1791-1867) set out to address this
problem. He investigated electricity from a variety of
sources --- including electric eels! --- to see whether they
could all produce the same effects, such as shocks and
sparks, attraction and repulsion. ``Heating'' refers, for
example, to the way a lightbulb filament gets hot enough to
glow and emit light. Magnetic induction is an effect
discovered by Faraday himself that connects electricity and
magnetism. We'll study this effect, which is the basis
for the electric generator, later in this chapter.

\noindent\begin{tabular}{|p{20mm}|p{18mm}|p{18mm}|p{18mm}|p{18mm}|}
\hline
\textbf{source of electricity} & \textbf{shocks} & \textbf{sparks} & \textbf{attraction and repulsion} & \textbf{heating}\\
\hline
rubbing & \checkmark & \checkmark & \checkmark & \checkmark \\
\hline
battery & \checkmark & \checkmark & \checkmark & \checkmark \\
\hline
animal & \checkmark & \checkmark & (\checkmark) & \checkmark \\
\hline
magnetically induced & \checkmark & \checkmark & \checkmark & \checkmark \\
\hline
\end{tabular}


The table shows a summary of some of Faraday's results.
Check marks indicate that Faraday or his close contemporaries
were able to verify that a particular source of electricity
was capable of producing a certain effect. (They evidently
failed to demonstrate attraction and repulsion between
objects charged by electric eels, although modern workers
have studied these species in detail and been able to
understand all their electrical characteristics on the same
footing as other forms of electricity.)

Faraday's results indicate that there is nothing fundamentally
different about the types of electricity supplied by the
various sources. They are all able to produce a wide variety
of identical effects. Wrote Faraday, ``The general
conclusion which must be drawn from this collection of facts
is that electricity, whatever may be its source, is
identical in its nature.''

If  the types of electricity are the same thing, what thing
is that? The answer is provided by the fact that all the
sources of electricity can cause objects to repel or attract
each other. We use the word ``charge'' to describe the
property of an object that allows it to participate in such
electrical forces, and we have learned that charge is
present in matter in the form of nuclei and electrons.
Evidently all these electrical phenomena boil down to the
motion of charged particles in matter.

\subsubsection{Electric current}
 If the fundamental phenomenon is the motion of charged
particles, then how can we define a useful numerical
measurement of it? We might describe the flow of a river
simply by the velocity of the water, but velocity will not
be appropriate for electrical purposes because we need to
take into account how much charge the moving particles have,
and in any case there are no practical devices sold at Radio
Shack that can tell us the velocity of charged particles.
Experiments show that the intensity of various electrical
effects is related to a different quantity: the number of
coulombs of charge that pass by a certain point per second.
By analogy with the flow of water, this quantity is called
the electric \index{current!defined}current:
\begin{equation*}
	\text{current} = \frac{\text{charge}}{\text{time}}
\end{equation*}
Its units of coulombs/second are more conveniently
abbreviated as \index{ampere (unit)}amperes, 1 A=1 C/s.
(In informal speech, one usually says ``amps.'')

\margup{-20mm}{\fig{riverssamecurrent}{The same current can be created by
a large amount of charge flowing slowly (top) or a small amount
flowing quickly (bottom).}}
\selfcheck{foo}{How does figure \figref{riverssamecurrent} relate
mathematically to the definition of current as charge divided by time?}

\begin{eg}{Number of electrons flowing through a lightbulb}
\egquestion Suppose a certain lightbulb has one amp flowing through it.
In a metal, like the filament of a lightbulb, the moving charged
particles are particles called electrons, and the size of the
charge on each electron is equal to the fundamental unit of
charge found by Millikan, $1.60\times10^{-19}$ coulombs.
How many electrons will pass through the filament in one second?

\eganswer 
An amp is one coulomb per second, so this boils down to finding
how many electrons there are in a coulomb.

The number of coulombs per electron is $1.60\times10^{-19}$,
so the number of electrons per coulomb is one over that:
\begin{equation*}
	\frac{1}{1.60\times10^{-19}} = 6.2\times10^{18} \qquad ,
\end{equation*}
or about six quadrillion. That's a lot of electrons!
This is a good example of the correspondence principle at work.
Before Millikan's discovery of quantization of charge, many
people had accomplished many useful things with electricity while thinking
of it as a nice smooth fluid. Their lightbulbs didn't
suddenly stop working just because Millikan published his paper. The
number of electrons flowing through a lightbulb is so great that we don't
even need to know that there's a certain granularity to it.\index{correspondence principle!for quantization of charge}
\end{eg}

In lab, you determined how many types of charge there were, and the
question naturally arises of how to incorporate the different types
of charge into the definition of current. Discussion question
\ref{dq:signsofcurrent} on page \pageref{dq:signsofcurrent} addresses
this point.
\end{envsubsection}
\begin{envsubsection}{Circuits}
How can we put electric currents to work? The only method of
controlling electric charge we have studied so far is to
charge different substances, e.g. rubber and fur, by rubbing
them against each other. Figure \figref{basiccircuits}/1 shows an attempt to use
this technique to light a lightbulb. This method is
unsatisfactory. True, current will flow through the bulb,
since electrons can move through metal wires, and the excess
electrons on the rubber rod will therefore come through the
wires and bulb due to the attraction of the positively
charged fur and the repulsion of the other electrons. The
problem is that after a zillionth of a second of current,
the rod and fur will both have run out of charge. No more
current will flow, and the lightbulb will go out.

\margup{-30mm}{\fig{basiccircuits}{In a practical circuit, charge has to be
recycled, as in figures 2 and 4.}}
Figure  \figref{basiccircuits}/2 shows a setup that works. The battery pushes
charge through the circuit, and recycles it over and over
again. (We'll have more to say later in this chapter about
how batteries work.) This is called a \index{circuit!complete}\index{complete
circuit}complete \index{circuit}circuit. Today, the
electrical use of the word ``circuit'' is the only one that
springs to mind for most people, but the original meaning
was to travel around and make a round trip, as when a
circuit court judge would ride around the boondocks,
dispensing justice in each town on a certain date.

Note that an example like  \figref{basiccircuits}/3 doesn't work. The wire will
quickly begin acquiring a charge, because it has no way
to get rid of the charge flowing into it. The repulsion of
this charge will make it more and more difficult to send any
more charge in, and soon the electrical forces exerted by
the battery will be canceled out completely. The whole
process would be over so quickly that the filament wouldn't
even have enough time to get hot and glow. This is known as
an \index{circuit!open}\index{open circuit}open circuit.
Exactly the same thing would happen if the complete circuit
of figure  \figref{basiccircuits}/2 was cut somewhere with a pair of scissors, and
that's essentially how an ordinary light switch
works: by opening up a gap in the circuit.

The water company has a meter that measures the rate of flow
of water into your house. Imagine trying to use such a meter
to measure the flow of water when you spit on
the sidewalk --- it would be impossible, because the flow wouldn't
last long enough, and wouldn't be steady. In electrical terms, a
meter that measures current is called an ammeter,\footnote{presumably
because ``ampmeter'' is hard to pronounce} and it only works
if you have the kind of steady flow that exists in
a complete circuit, \figref{basiccircuits}/4
To use an ammeter, we break
into the path of the electric current and interpose the
meter like a tollbooth on a road. There is still a
complete circuit, and as far as the battery and bulb are
concerned, the ammeter is just another segment of wire.
\end{envsubsection}
\begin{envsubsection}{Voltage}
	Electrical circuits can be used for sending signals, storing
	information, or doing calculations, but their most common
	purpose by far is to manipulate energy, as in the battery-and-bulb
	example. We know that lightbulbs are
	rated in units of watts, i.e. how many joules per second of
	energy they can convert into heat and light, but how would
	this relate to the flow of charge as measured in amperes? By
	way of analogy, suppose your friend, who didn't take
	physics, can't find any job better than pitching bales of
	hay. The number of calories he burns per hour will certainly
	depend on how many bales he pitches per minute, but it will
	also be proportional to how much energy he has to expend
	on each bale. If his job is to toss them up into a
	hayloft, he'll got tired a lot more quickly than someone
	who merely tips bales off a loading dock into trucks. In metric units,
	\begin{equation*}
		\frac{\text{joules}}{\text{second}}
		 = \frac{\text{haybales}}{\text{second}}
			 \times \frac{\text{joules}}{\text{haybale}}    \qquad   .  
	\end{equation*}
	Similarly, the rate of energy transformation by a battery
	will not just depend on how many coulombs per second it
	pushes through a circuit but also on how much energy
	it expends on each coulomb of charge:
	\begin{equation*}
		\frac{\text{joules}}{\text{second}}
		 = \frac{\text{coulombs}}{\text{second}}
			 \times \frac{\text{joules}}{\text{coulomb}}
	\end{equation*}
	\noindent or
	\begin{equation*}
			\text{power}	 = 	\text{current} \times \text{energy per unit charge}   \qquad   .  
	\end{equation*}
	Units of joules per coulomb are abbreviated as \index{volt (unit)}volts, 1 V=1 J/C, named after the Italian
	physicist Count Volta. 

	To summarize, we have the definition of voltage
	\begin{equation*}
		\text{voltage} = \frac{\text{energy}}{\text{charge}}
	\end{equation*}
	and the equation for electric power
	\begin{equation*}
		\text{power} = \text{current} \times \text{voltage} \qquad .
	\end{equation*}

\widefigsidecaption{waterwheels}{In these drawings, heights represent voltages. The currents
in figures 1 and 2 are the same, but more power can be extracted from waterwheel 2, because
of the greater voltage difference. Only differences in voltage are physically meaningful; waterwheels
1 and 3 extract the same amount of power.}

\begin{eg}{Your electric bill}
To charge you for the right amount of electricity, the electric company has
to know how much energy you used. For instance, if you use a power of 1000 watts
for one hour, the energy you use is
\begin{align*}
	\text{energy} &= \text{power} \times \text{time} \\
		      &= (1000\ \text{watts})(3600\ \text{seconds})\\
		      & = 3600000\ \text{joules} \qquad .
\end{align*}
This is just the definition of power --- so far we haven't even used any
knowledge about electricity.

But how do they know you're using 1000 watts on a particular afternoon?
The only direct way to find out would be an energy measurement. For instance,
they could send someone to stand next to you while you heated a pot of
water, monitoring the rate at which the water heated up. Not very practical!

Instead, they exploit the equation for electric power,
\begin{equation*}
	\text{power} = \text{current} \times \text{voltage} \qquad .
\end{equation*}
At an electrical outlet, the voltage difference between one hole and
the other is 110 volts; for every coulomb of charge that flows out of one hole,
through your stove, and back in the other hole, 110 joules worth of heat energy
are deposited in your house.\footnote{In the U.S., most outlets are 110 volts, but
washers and dryers use special 220-volt outlets. I'm also ignoring the fact that
household circuits use alternating current (AC): the flow of electricity is first
in one direction and then in the other, switching back and forth 60 times a second.}
Since they know the voltage, they just have to
monitor the current flowing into your house, and they can then determine
how much power you're using.
\end{eg}
\end{envsubsection}
%
\begin{envsubsection}{Resistance}
%
\margup{-35mm}{\fig{riversdifferentresistances}{The voltage (height) difference is the same
in both cases, but the shallower river has less current, because there is less
water in it that is available to flow.}}
What's the physical difference between a 100-watt lightbulb and a 200-watt one?
They both plug into a 110-volt outlet, so according to the equation
$\text{power} = \text{current} \times \text{voltage}$, the only way to explain the
double power of the 200-watt bulb is that it must pull in, or ``draw,''
twice as much current. By analogy, a fire hose and a garden hose might
be served by pumps that give the same pressure (voltage), but more water
will flow through the fire hose, because there's simply more water in the hose
that can flow. Likewise, a wide, deep river could flow down the same slope
as a tiny creek, but the number of liters of water flowing through the big
river is greater. If you look at the filaments of a 100-watt bulb and a 200-watt
bulb, you'll see that the 200-watt bulb's filament is thicker. In the charged-particle
model of electricity, we expect that the thicker filament will contain more
charged particles that are available to flow. We say that the thicker
filament has a lower electrical resistance than the thinner one.

\widefigsidecaption{fat-and-skinny-pipes}{A fat pipe has less resistance than a skinny pipe.}

Although it's harder to pump water rapidly through a garden hose than through
a fire hose, we could always compensate by using a higher-pressure pump.
Similarly, the amount of current that will flow through a lightbulb
depends not just on its resistance but also on how much of a voltage difference
is applied across it. For many substances, including the tungsten metal that
lightbulb filaments are made of, we find that the amount of current that flows
is proportional to the voltage difference applied to it, so that the ratio
of voltage to current stays the same. We then use this ratio as a numerical
definition of resistance,
\begin{equation*}
	\text{resistance} = \frac{\text{voltage difference}}{\text{current}} \qquad ,
\end{equation*}
which is known as Ohm's law.\index{Ohm's law}
The units of resistance are ohms, symbolized with an uppercase Greek letter
Omega, $\Omega$. Physically, when a current flows through a resistance,
the result is to transform electrical energy into heat. In a lightbulb filament,
for example, the heat is what causes the bulb to glow.

	Ohm's law states that many substances, including many solids
	and some liquids, display this kind of behavior, at least
	for voltages that are not too large. The fact that Ohm's law
	is called a ``law'' should not be taken to mean that all
	materials obey it, or that it has the same fundamental
	importance as the conservation laws, for example. Materials are
	called \index{ohmic!defined}\emph{ohmic} or \emph{nonohmic},
	depending on whether they obey Ohm's law.

	On an intuitive level, we can understand the idea of
	resistance by making the sounds ``hhhhhh'' and ``ffffff.''
	To make air flow out of your mouth, you use your diaphragm
	to compress the air in your chest. The pressure difference
	between your chest and the air outside your mouth is
	analogous to a voltage difference. When you make the ``h''
	sound, you form your mouth and throat in a way that allows
	air to flow easily. The large flow of air is like a large
	current. Dividing by a large current in the definition of
	resistance means that we get a small resistance. We say that
	the small resistance of your mouth and throat allows a large
	current to flow. When you make the ``f'' sound, you increase
	the resistance and cause a smaller current to flow. In this
	mechanical analogy, resistance is like friction: the air rubs
	against your lips. Mechanical friction converts 
	mechanical forms of energy to heat, as when you rub your
	hands together. Electrical friction --- resistance ---
	converts electrical energy to heat.

	If objects of the same size and shape made from two
	different ohmic materials have different resistances, we can
	say that one material is more resistive than the other, or
	equivalently that it is less conductive. Materials, such as
	metals, that are very conductive are said to be good
	\index{conductor!defined}\emph{conductors}. Those that are
	extremely poor conductors, for example wood or rubber, are
	classified as \index{insulator!defined}\emph{insulators}. There
	is no sharp distinction between the two classes of
	materials. Some, such as silicon, lie midway between the two
	extremes, and are called semiconductors.
\end{envsubsection}
%
\begin{envsubsection}{Applications}
\subsubsection{Superconductors}
All materials display some variation in resistance according
to temperature (a fact that is used in thermostats to make a
thermometer that can be easily interfaced to an electric
circuit). More spectacularly, most metals have been found to
exhibit a sudden change to \emph{zero} resistance when
cooled to a certain critical temperature. They are then said
to be superconductors. A current flowing through a superconductor
doesn't create any heat at all.

Theoretically, superconductors should
make a great many exciting devices possible, for example
coiled-wire magnets that could be used to levitate trains.
In practice, the critical temperatures of all metals are
very low, and the resulting need for extreme refrigeration
has made their use uneconomical except for such specialized
applications as particle accelerators for physics research.

But scientists have recently made the surprising discovery
that certain ceramics are superconductors at less extreme
temperatures. The technological barrier is now in finding
practical methods for making wire out of these brittle
materials. Wall Street is currently investing billions of
dollars in developing superconducting devices for cellular
phone relay stations based on these materials. In 2001, the
city of Copenhagen replaced a short section of its
electrical power trunks with superconducing cables, and they
are now in operation and supplying power to customers.

There is currently no satisfactory theory of superconductivity
in general, although superconductivity in metals is
understood fairly well. Unfortunately I have yet to find a
fundamental explanation of superconductivity in metals that
works at the introductory level.

\subsubsection{Constant voltage throughout a conductor}
The idea of a superconductor leads us to the question of how
we should expect an object to behave if it is made of a very
good conductor. Superconductors are an extreme case, but
often a metal wire can be thought of as a perfect conductor,
for example if the parts of the circuit other than the wire
are made of much less conductive materials. What happens if
the resistance equals zero in the equation 
\begin{equation*}
	\text{resistance} = \frac{\text{voltage difference}}{\text{current}} \qquad \text{?}
\end{equation*}
The result
of dividing two numbers can only be zero if the number on
top equals zero. This tells us that if we pick any two
points in a perfect conductor, the voltage difference
between them must be zero. In other words, the entire
conductor must be at the same voltage. Using the water
metaphor, a perfect conductor is like a perfectly calm
lake or canal, whose surface is flat. If you take an
eyedropper and deposit a drop of water anywhere on the
surface, it doesn't flow away, because the water is still.
In electrical terms, a charge located anywhere in the interior
of a perfect conductor will always feel a total electrical
force of zero.

Suppose, for example, that you build up
a static charge by scuffing your feet on a carpet, and then
you deposit some of that charge onto a doorknob, which is a
good conductor.  How can all that charge be in the doorknob
without creating any electrical force at any point inside
it? The only possible answer is that the charge moves around
until it has spread itself into just the right configuration.
In this configuration, the forces exerted by all the charge
on any charged particle within the doorknob
exactly cancel out.

We can explain this behavior if we assume that the charge
placed on the doorknob eventually settles down into a stable
equilibrium. Since the doorknob is a conductor, the charge
is free to move through it. If it was free to move and any
part of it did experience a nonzero total force from the
rest of the charge, then it would move, and we would not
have an equilibrium.

It also turns out that charge placed on a conductor, once it
reaches its equilibrium configuration, is entirely on the
surface, not on the interior. We will not prove this fact
formally, but it is intuitively reasonable (see
discussion question \ref{dq:chargeonsurface}).\label{text:chargeonsurface}

\subsubsection{Short circuits}
So far we have been assuming a perfect conductor. What if it's
a good conductor, but not a perfect one? Then we can
solve for
\begin{equation*}
	\text{voltage difference} = (\text{current})\times(\text{resistance}) \qquad .
\end{equation*}
An ordinary-sized current
will make a very small result when we multiply it by the
resistance of a good conductor such as a metal wire. The
voltage throughout the wire will then be nearly constant.
If, on the other hand, the current is extremely large, we
can have a significant voltage difference. This is what
happens in a \index{circuit!short}\index{short circuit!defined}short-circuit:
a circuit in which a low-resistance pathway connects the two
sides of a voltage source. Note that this is much more
specific than the popular use of the term to indicate any
electrical malfunction at all. If, for example, you
short-circuit a 9-volt battery as shown in the figure, you
will produce perhaps a thousand amperes of current, leading
to a very large value of $\text{power}=(\text{current})\times(\text{voltage difference})$.
The wire gets hot!

\subsubsection{The voltmeter}
A voltmeter is nothing more than an ammeter with an
additional high-value resistor through which the current is
also forced to flow, \figref{voltmeterconstruction}. Ohm's law relates the current through
the resistor is related directly to the voltage difference
across it, so the meter can be calibrated in units of volts
based on the known value of the resistor. The voltmeter's
two probes are touched to the two locations in a circuit
between which we wish to measure the voltage difference,
\figref{voltmeteruse}. Note how cumbersome this type of drawing is, and how
difficult it can be to tell what is connected to what. This
is why electrical drawing are usually shown in schematic\index{schematics}
form. Figure \figref{voltmeteruseschematic} is a schematic representation of figure 
\figref{voltmeteruse}.

\margup{-23mm}{%
\fig{voltmeterconstruction}{Under the hood,
a voltmeter is really an ammeter combined with a
high-value resistor.}
\spacebetweenfigs
\fig{voltmeteruse}{Measuring the voltage difference across a
lightbulb.}
\spacebetweenfigs
\fig{voltmeteruseschematic}{The same setup drawn in schematic form.}
\spacebetweenfigs
\fig{ammeteruseschematic}{The setup for measuring current is different.}
\spacebetweenfigs
}
The setups for measuring current and voltage are different.
When we're measuring current, we're finding ``how much
stuff goes through,'' so we place the ammeter where all the
current is forced to go through it. Voltage, however, is not
``stuff that goes through,'' it is a measure of electrical
energy. If an ammeter is like the meter that measures your
water use, a voltmeter is like a measuring stick that tells
you how high a waterfall is, so that you can determine how
much energy will be released by each kilogram of falling
water. We don't want to force the water to go through the
measuring stick! The arrangement in figure \figref{voltmeteruseschematic} is a
\index{circuit!parallel}\index{parallel circuit!defined}parallel
circuit: one in there are ``forks in the road'' where some
of the current will flow one way and some will flow the
other. Figure \figref{ammeteruseschematic} is said to be wired in
\index{circuit!series}\index{series circuit!defined}series:
all the current will visit all the
circuit elements one after the other.

If you inserted a voltmeter incorrectly, in series with the
bulb and battery, its large internal resistance would cut
the current down so low that the bulb would go out. You
would have severely disturbed the behavior of the circuit by
trying to measure something about it.

Incorrectly placing an ammeter in parallel is likely to be
even more disconcerting. The ammeter has nothing but wire
inside it to provide resistance, so given the choice, most
of the current will flow through it rather than through the
bulb. So much current will flow through the ammeter, in
fact, that there is a danger of burning out the battery or
the meter or both! For this reason, most ammeters have fuses
or circuit breakers inside. Some models will trip their
circuit breakers and make an audible alarm in this
situation, while others will simply blow a fuse and stop
working until you replace it.
\end{envsubsection}

\dqheader
\begin{dq}\label{dq:signsofcurrent}
In lab, you determined how many types of charge there were, and the
question naturally arises of how to incorporate the different types
of charge into the definition of current.
Fundamentally, charge measures the ability of an object to make
electrical forces. If you start with an uncharged object, and then
start letting more than one type of charge flow into it simultaneously,
what happens? Discuss some examples and decide how these ideas should
be incorporated into the definition of current.
\end{dq}
\begin{dq}
In figure \figref{basiccircuits}/4 on page \pageref{fig:basiccircuits}, what would happen if you had
the ammeter on the left rather than on the right?
\end{dq}
\begin{dq}\label{dq:chargeonsurface}
Imagine a charged doorknob, as described on page \pageref{text:chargeonsurface}.
Why is it intuitively reasonable to believe that all the charge will end up on the surface
of the doorknob, rather than on the interior?
\end{dq}


\mysection{Electromagnetism}
	\epigraph{Think not that I am come to destroy the law, or the prophets:
	I am not come to destroy, but to fulfill.}{Matthew 5:17}
\begin{envsubsection}{Magnetic interactions}
At this stage, you understand roughly as much about the classification of interactions as physicists
understood around the year 1800. There appear to be three fundamentally different types
of interactions: gravitational, electrical, and magnetic. As discussed on
page \pageref{introunificationofforces}, many types of interactions that appear superficially to be
distinct --- stickiness, chemical interactions, the energy an archer stores in a bow --- are
really the same: they're manifestations of electrical interactions between atoms.
Is there any way to shorten the list any further? The prospects seem dim at first. For instance,
we find that if we rub a piece of fur on a rubber rod, the fur does not attract or repel a magnet.
The fur has an electric field, and the magnet has a magnetic field. The two are completely separate,
and don't seem to affect one another. Likewise we can test whether magnetizing a piece of iron
changes its weight. The weight doesn't seem to change by any measurable amount, so magnetism and
gravity seem to be unrelated.

That was where things stood until 1820, when the Danish physicist Hans Christian\index{Oersted, Hans Christian}
Oersted was delivering a lecture at the University of Copenhagen, and he wanted to give his
students a demonstration that would illustrate the cutting edge of research. He generated
a current in a wire by making a short circuit across a battery, and held the wire near a
magnetic compass. The ideas was to give an example of how one could search for a previously undiscovered
link between electricity (the electric current in the wire) and magnetism. One never knows how much
to believe from these dramatic legends, but the story is\footnote{Oersted's paper
describing the phenomenon says that ``The first experiments on the subject \ldots
were set on foot in the classes for electricity, galvanism, and magnetism, which were
held by me in the winter just past,'' but that doesn't tell us whether the result was
really a surprise that occurred in front of his students.} that the experiment he'd expected to turn out
negative instead turned out positive: when he held the wire near the
compass, the current in the wire caused the compass to twist!

\margup{-170mm}{%
\fig{oersted}{1. When the circuit is incomplete, no current flows through the wire, and the magnet is
unaffected. It points in the direction of the Earth's magnetic field. 2. The circuit is completed, and
current flows through the wire. The wire has a strong effect on the magnet, which turns almost perpendicular
to it. If the earth's field could be removed entirely, the compass would point exactly perpendicular to the
wire; this is the direction of the wire's field.}
\spacebetweenfigs
\fig{magnetized}{A schematic representation of an unmagnetized material, 1, and a magnetized one, 2.}
}
People had tried similar experiments before, but only with static electricity, not with
a moving electric current. For instance, they had hung batteries so that they were free to
rotate in the earth's magnetic field, and found no effect; since the battery was not connected
to a complete circuit, there was no current flowing. With Oersted's own setup, \figref{oersted},
the effect was only produced when the ``circuit was closed, but not
when open, as certain very celebrated physicists in vain attempted several years ago.''\footnote{All
quotes are from the 1876 translation are by J.E. Kempe.}

Oersted was eventually
led to the conclusion that magnetism was an interaction between moving charges and
other moving charges, i.e., between one current and another.  \index{magnetism!caused by moving charges}
A permanent magnet, he inferred, contained currents on a microscopic
scale that simply weren't practical to measure with an ammeter. Today this seems natural
to us, since we're accustomed to picturing an atom as a tiny solar system, with the electrons
whizzing around the nucleus in circles. As shown in figure \figref{magnetized},
a magnetized piece of iron is different from an
unmagnetized piece because the atoms in the unmagnetized piece are jumbled in random
orientations, whereas the atoms in the magnetized piece are at least partially organized
to face in a certain direction.

\margup{0mm}{
\fig{magdeflects}{Magnetism is an interaction between moving charges and moving charges. The moving
charges in the wire attract the moving charges in the beam of charged particles in the vacuum tube.}
\spacebetweenfigs
\fig{fulfill}{One observer sees an electric field, while the other sees both an electric
field and a magnetic one.}
}
Figure \figref{magdeflects} shows an example that is conceptually
simple, but not very practical. If you try this with a typical vacuum tube, like a TV
or computer monitor, the current in the wire probably won't be enough to produce a visible
effect. A more practical method is to hold a magnet near the screen. We still have
an interaction between moving charges and moving charges, but the swirling electrons
in the atoms in the magnet are now playing the role played by the moving charges in the wire
in figure \figref{magdeflects}. Warning: if you do this, make sure your monitor has a
demagnetizing button! If not, then your monitor may be permanently ruined.
\end{envsubsection}
%
\begin{envsubsection}{Relativity requires magnetism}\index{magnetism!and relativity}\index{relativity!and magnetism}
So magnetism is an interaction between moving charges and moving charges. But how
can that be?
Relativity tells us that
motion is a matter of opinion. Consider figure \figref{fulfill}. In this figure and in figure
\figref{magrelativity}, the dark and light coloring of the particles represents the fact that
one particle has one type of charge and the other particle has the other type.
Observer \figref{fulfill}/2 sees the two particles as flying through space side by side, so they
would interact both electrically (simply because they're charged) and magnetically
(because they're charges in motion). But an observer moving along with them,  \figref{fulfill}/1, would
say they were both at rest, and would expect only an electrical interaction. This seems
like a paradox.
Magnetism, however, comes not to destroy relativity but to fulfill it. Magnetic interactions
	\emph{must} exist according to the theory of relativity. To understand how this can be,
	consider how time and space behave in relativity. Observers in different frames of reference
	disagree about the lengths of measuring sticks and the speeds of clocks, but the laws
	of physics are valid and self-consistent in either frame of reference.
	Similarly, observers in different frames of reference disagree about what electric and magnetic
	fields there are, but they agree about concrete physical events.
	An observer in frame of reference \figref{fulfill}/1
	says there are electric fields around the particles, and predicts that as time goes on, the
	particles will begin to accelerate towards one another, eventually colliding. She explains the
	collision as being due to the electrical attraction between the particles.
	A different observer, \figref{fulfill}/2, says the particles are moving. This observer
	also predicts that the particles will collide, but explains their motion in terms of both
	an electric field and a magnetic field. As we'll see shortly, the
	magnetic field is \emph{required} in order to maintain consistency between the predictions made
	in the two frames of reference.
	
\marg{
\fig{magrelativity}{A model of a charged particle and a current-carrying wire, seen in
	two different frames of reference. The relativistic length contraction is highly
	exaggerated. The force on the lone particle is purely
	magnetic in 1, and purely electric in 2.}
}
	To see how this really works out, we need to find a nice simple example.
	An example like figure \figref{fulfill} is \emph{not} easy
	to handle, because in the second frame of reference, the moving charges
	create fields that change over time at any given location, like when the V-shaped wake of a speedboat
	washes over a buoy. Examples like
	figure \figref{magdeflects} are easier, because there is a steady flow of charges, and
	all the fields stay the same over time.
        Figure \figref{magrelativity}/1 shows a simplified and idealized model of figure \figref{magdeflects}.
	The charge by itself is like one of the charged particles
	in the vacuum tube beam of figure \figref{magdeflects}, and instead of the wire, we have
	two long lines of charges moving in opposite directions. Note that,
	as discussed in discussion question \ref{dq:signsofcurrent} on page \pageref{dq:signsofcurrent},
	the currents of the two lines of charges do not cancel out. The dark balls represent particles with
	one type of charge, and the light balls have the other type. Because of this, the total current in
	the ``wire'' is double what it would be if we took away one line.

	As a model of figure \figref{magdeflects}, figure \figref{magrelativity}/1 is partly realistic and
	partly unrealistic. In a real piece of copper wire, there are indeed charged particles of both types,
	but it turns out that the particles of one type (the protons) are locked in place, while only some 
	of the other type (the electrons) are free to move. The model also shows the particles moving in
	a simple and orderly way, like cars on a two-lane road, whereas in reality most of the particles are
	organized into copper atoms, and there is also a great deal of random thermal motion.
	The model's unrealistic features aren't a
	problem, because the point of this exercise is only to find one particular situation that shows
	magnetic effects must exist based on relativity.

	What electrical force does the lone particle in figure \figref{magrelativity}/1 feel? Since the
	density of ``traffic'' on the two sides of the ``road'' is equal, there is zero overall
	electrical force on the lone particle. Each ``car'' that attracts the lone particle is paired with a partner on the other
	side of the road that repels it. If we didn't know about magnetism, we'd think this
	was the whole story: the lone particle feels no force at all from the wire.

	Figure \figref{magrelativity}/2
        shows what we'd see if we were observing all this from a frame of reference moving
	along with the lone charge.
	Here's where the relativity comes in. Relativity tells us that moving objects
	appear contracted to an observer who is not moving along with them.
	Both lines of charge are in motion in both frames of reference, but in frame 1
	they were moving at equal speeds, so their contractions were equal.
	In frame 2, however, their speeds are unequal. The dark
	charges are moving more slowly than in frame 1, so in frame 2 they are less contracted.
	The light-colored charges are moving more quickly, so their contraction is greater now.
	The ``cars'' on the two sides of the ``road'' are no longer paired off, so the electrical
	forces on the lone particle no longer cancel out as they did in \figref{magrelativity}/1.
	The lone particle is attracted to the wire, because the particles attracting it are more
	dense than the ones repelling it. Furthermore, the attraction felt
	by the lone charge must be purely electrical, since the lone charge is at rest in this
	frame of reference, and magnetic effects occur only between moving charges and other
	moving charges.

	Now observers in frames 1 and 2 disagree about many things, but they do agree on
	concrete events. Observer 2 is going to see the lone particle drift toward the wire
	due to the wire's electrical attraction, gradually speeding up, and eventually hit
	the wire. If 2 sees this collision, then 1 must as well. But 1 knows that the total
	electrical force on the lone particle is exactly zero. There must be some new type
	of force. She invents a name for this new type of force: magnetism. This was a particularly
	simple example, because the fields were purely magnetic in one frame of reference, and
	purely electrical in another. In general, an observer in a certain frame of reference
	will measure a mixture of electric and magnetic fields, while an observer in another
	frame, in motion with respect to the first, says that the same volume of space contains a different mixture.

\margup{-50mm}{\fig{magtwobody}{Magnetic interactions involving only two particles at a time. In these figures, unlike figure
\figref{magrelativity}/1, there are electrical forces as well as magnetic ones. The electrical forces are
not shown here. Don't memorize these rules!}
\spacebetweenfigs
\fig{weathervane}{Example \ref{eg:weathervane}}}
We therefore arrive at the conclusion that electric and magnetic phenomena aren't
separate. They're different sides of the same coin. We refer to electric and magnetic interactions
collectively as electromagnetic interactions. Our list of the fundamental interactions
of nature now has two items on it instead of three: gravity and electromagnetism.\index{magnetism!related to electricity}\index{electromagnetism}

The basic rules for magnetic attractions and repulsions, shown in figure \figref{magtwobody}, aren't
quite as simple as the ones for gravity and electricity. Rules \figref{magtwobody}/1 and
\figref{magtwobody}/2 follow directly from our previous analysis of figure \figref{magrelativity}.
Rules 3 and 4 are obtained by flipping the type of charge
that the bottom particle has. For instance, rule 3 is like rule 1, except that the bottom charge
is now the opposite type. This turns the attraction into a repulsion. (We know that flipping the charge
reverses the interaction, because that's the way it works for electric forces, and magnetic forces
are just electric forces viewed in a different frame of reference.)

\begin{eg}{A magnetic weathervane placed near a current.}\label{eg:weathervane}
Figure \figref{weathervane} shows a magnetic weathervane, consisting of two charges that spin
in circles around the axis of the arrow. (The magnetic field doesn't cause them to spin; a motor
is needed to get them to spin in the first place.) Just like the magnetic compass in figure \figref{oersted},
the weathervane's arrow tends to align itself in the direction perpendicular to the wire. This
is its preferred orientation because the charge close to the wire is attracted to the
wire, while the charge far from the wire is repelled by it.
\end{eg}
\end{envsubsection}
%
\begin{envsubsection}{Magnetic fields}\index{magnetism!magnetic field}\index{field!magnetic}\index{magnetic field}
How should we define the magnetic field? When two objects attract each other gravitationally, their
gravitational energy depends only on the distance between them, and it seems intuitively reasonable
that we define the gravitational field arrows like a street sign that says ``this way to lower
gravitational energy.'' The same idea works fine for the electric field. But what if two charged
particles are interacting magnetically? Their interaction doesn't just depend on the distance, but
also on their motions.

\marg{\fig{swirly}{The magnetic field curls around the wire in circles. At each point in space, the magnetic
compass shows the direction of the field.}}
We need some way to pick out some direction in space, so we can say, ``this is the direction
of the magnetic field around here.'' A natural and simple method is to define the magnetic field's
direction according to the direction a compass points. Starting from this definition we can, for example,
do experiments to show that the magnetic field of a current-carrying wire forms a circular pattern,
\figref{swirly}.

But is this the right definition? Unlike the
definitions of the gravitational and electric fields' directions, it involves a particular human-constructed
tool. However, compare figure \figref{oersted} on page \pageref{fig:oersted} with figure
\figref{weathervane} on page \pageref{fig:weathervane}. Note that both of these tools line themselves
up along a line that's perpendicular to the wire. In fact, no matter how hard you try, you will never
be able to invent any other electromagnetic device that will align itself with any other line. All you
can do is make one that points in exactly the opposite direction, but along the same line. For instance,
you could use paint to reverse the colors that label the ends of the magnetic compass needle, or you
could build a weathervane just like figure \figref{weathervane}, but spinning like a left-handed screw
instead of a right-handed one. The weathervane and the compass aren't even as different as they appear.
Figure \figref{barmagnethanging} shows their hidden similarities.\label{arbitrarydirectionofmag}

\widefigsidecaption{barmagnethanging}{1. The needle of a magnetic compass is nothing more than a bar magnet that
is free to rotate in response to the earth's magnetic field.
2. A cartoon of the bar magnet's structure at the atomic level. Each atom is very much like the
weathervane of figure \figref{weathervane}.}

Nature is trying to tell us something: there really is something special
about the direction the compass points. Defining the direction of the magnetic field in terms of this
particular device isn't as arbitrary as it seems. The only arbitrariness is that we could have
built up a whole self-consistent set of definitions that started by defining the magnetic field
as being in the opposite direction.

\begin{eg}{Head-to-tail alignment of bar magnets}\label{eg:headtotail}
\egquestion
If you let two bar magnets like the one in figure \figref{barmagnethanging} interact, which way do
they want to line up, head-to-head or head-to-tail?

\eganswer
Each bar magnet contains a huge number of atoms, but that won't matter for our result; we can imagine
this as an interaction between two individual atoms. For that matter, let's model the atoms as weathervanes
like the one in figure \figref{weathervane}. Suppose we put two such weather vanes side by side, with their
arrows both pointing away from us. From our point of view, they're both spinning clockwise. As one of the
charges in the left-hand weather vane comes down on the right side, one of the charges in the right-hand
vane comes up on the left side. These two charges are close together, so their magnetic interaction is
very strong at this moment. Their interaction is repulsive, so this is an unstable arrangement of the two
weathervanes.

On the other hand, suppose the left-hand weathervane is pointing away from is, while its partner on the
right is pointing toward us. From our point of view, we see the one on the right spinning counterclockwise.
At the moment when their charges come as close as possible, they're both on the way up. Their interaction
is attractive, so this is a stable arrangement.

Translating back from our model to the original question about bar magnets, we find that bar magnets
will tend to align themselves head-to-tail. This is easily verified by experiment.
\end{eg}


\widefigsidecaption{righthandrule}{The force on a charged particle moving through a magnetic field is
perpendicular to both the field and its direction of motion. The relationship is right-handed for
one type of charge, and left-handed for the other type.}

If you go back and apply this definition to all the examples we've encountered so far, you'll find
that there's a general rule: the force on a charged particle moving through a magnetic field is
perpendicular to both the field and its direction of motion. A force perpendicular to the direction
of motion is exactly what is required for circular motion, so we find that a charged particle in a
vacuum will go in a circle around the magnetic field arrows (or perhaps a corkscrew pattern, if it
also has some motion along the direction of the field). That means that magnetic fields tend to trap
charged particles.

\marg{\fig{circularorbit}{A beam of electrons circles around the magnetic field arrows.}}
Figure \figref{circularorbit} shows this principle in action. A beam of electrons is
created in a vacuum tube, in which a small amount of hydrogen gas has been left.
A few of the electrons strike hydrogen molecules, creating light and letting us see
the path of the beam. A magnetic
field is produced by passing a current (meter) through the circular
coils of wire in front of and behind the tube. In the bottom figure,
with the magnetic field turned on, the force perpendicular to the
electrons' direction of motion causes them to move in a circle.
	

\begin{eg}{Sunspots}\index{sunspots}
Sunspots, like the one shown in the photo on page \pageref{sunspotphoto}, are places where the sun's
magnetic field is unusually strong. Charged particles are trapped there for months at a time. This
is enough time for the sunspot to cool down significantly, and it doesn't get heated back up because
the hotter surrounding material is kept out by the same magnetic forces.
\end{eg}

\begin{eg}{The aurora and life on earth's surface}\index{aurora}\index{Mars!life on}
A strong magnetic field seems to be one of the prerequisites for the existence of life on the surface
of a planet. Energetic charged particles from the sun are trapped by our planet's
magnetic field, and harmlessly spiral down to the earth's surface at the poles. In addition to protecting us,
this creates the aurora, or ``northern lights.''

 The astronauts who went to the moon
were outside of the earth's protective field for about a week, and suffered significant doses of radiation
during that time. The problem would be much more serious for astronauts on a voyage to Mars, which would
take at least a couple of years. They would be subjected to intense radiation while in interplanetary space,
and also while on Mars's surface, since Mars lacks a strong magnetic field.

Features in one Martian rock have been interpreted by some scientists as fossilized bacteria.
If single-celled life evolved on Mars, it has presumably been forced to stay below the surface.
(Life on Earth probably evolved deep in the oceans, and most of the Earth's biomass consists of
single-celled organisms in the oceans and deep underground.)
\end{eg}
\end{envsubsection}
%
\mysection{Induction}
\begin{envsubsection}{Electromagnetic signals}\index{electromagnetism!signals}
You may have noticed that as we've progressed in our discussion of electromagnetism,
I've been referring to the electric and magnetic fields more and more as if they
were real things permeating all of space. When I first introduced the concept of
a field --- the gravitational field --- it played a minor role. It was nothing more
than a convenient way of calculating the energy required to bring a rock farther away
from the earth. Newton never even felt the need to invent such a concept. To him, the
only real actors on the stage were atoms. Like Romeo and Juliet, they were real,
material objects. Like Romeo and Juliet's love, the gravitational interactions
were just a way of describing the relationship between the atoms.

\widefigsidecaption{signalwithmagnets}{An impractical, but conceptually simple, scheme for
sending signals with magnets.}
Suppose Romeo and Juliet,
separated by a paper-thin wall, use a pair of
bar magnets to signal to each other. As discussed in example 
\ref{eg:headtotail} on page \pageref{eg:headtotail}, the magnets want to line up head-to-tail,
\figref{signalwithmagnets}/1. Each person feels
his or her own magnet trying to twist around in response to any
rotation performed by the other person's magnet. If the person
on the right flips her magnet, \figref{signalwithmagnets}/2, the person
on the left can feel the signal. The
practical range of communication would be very short for
this setup, but a sensitive detector could pick
up magnetic signals from much farther away. 

A question now naturally arises as to whether there is any
time delay in this kind of electromagnetic communication.
Newton would have thought not, since he
conceived of physics in terms of instantaneous action at a
distance. If, on the other hand, there is such a time delay,
then what is it that is traveling across the space between the
two magnets? It would presumably be a disturbance in the electric
and magnetic fields that rippled out from the twisting magnet, like
ripples made by a wriggling bug on the surface of a pond. We would
then be more inclined to grant the electric and magnetic fields ``real thing
status.''

There is such a time\index{relativity!limit on speed of signals}
delay. Relativity says that not only is there an upper
limit on the speed of a material object --- the speed of
light, $3\times10^8$ m/s --- but the same limit applies to signals
as well. Here's why. Imagine that we could send a signal without any time delay
at all. Alice sends a signal from planet A to Bob, on planet B.
Alice and Bob agree that events A and B are simultaneous.
 But as shown
in figure \figref{simultaneity} on page \pageref{fig:simultaneity},
observers in different frames of reference disagree
about simultaneity. An observer moving in the direction from B to
A says B happens after A, but an observer moving in the opposite
direction says B happens before A. According to this observer, Bob
might get the signal before Alice had even made up her mind to send it!
This is just like a time machine, and it results in all the same paradoxes
that time machines cause. Bob could, for instance, send a signal back in
time to Alice, telling her to hire gangsters to come and smash
his radio transmitter. If the gangsters smashed the radio before Bob sent
the signal to Alice, then it wasn't possible for the gangsters to
get hired in the first place. Since instantaneous transmission of signals
leads to these crazy paradoxes, we conclude that instantaneous signaling isn't
possible.\footnote{This isn't quite as ironclad an argument as it appears.
We've only discussed the special theory of relativity, not the general theory,
which incorporates gravity. The general theory leads to some apparently
reasonable recipes by which an advanced civilization, with the ability to
manipulate vast amounts of matter, could build a time machine. Careful investigation,
however, shows that there are some effects, which physicists are presently
unable to calculate accurately, that might cause such a gateway in time to
be useless for sending either material objects or signals back in time.
This has led physicist Stephen Hawking to postulate that the laws of physics
conspire to strictly forbid backward time travel. He refers to this as the principle
of chronology protection, and jokes that it will ``keep the world safe for
historians.''}\index{Hawking, Stephen}\index{time travel}\index{chronology protection}

This may all sound like pure science fiction, but it's not.
If you make a long-distance phone call that is routed
through a communications satellite, you should easily be
able to detect a delay of about half a second over the
signal's round trip of 50,000 miles. Radar, which was arguably
the technology that won World War II, is based on measuring the
time delay for a radio ``echo'' to come back.
As we'll soon see, the radio waves used in these signaling methods
are actually disturbances in the electric and magnetic fields, but the
relativistic argument applies regardless of the method used for
signaling.\index{radar}

An even stronger reason to think of fields as real things\label{energy-in-fields}
comes from the fact that field-ripples carry energy.
First suppose that the person holding the bar magnet on the
right decides to reverse hers, resulting in configuration
\figref{signalwithmagnets}/2. To twist it, she has to convert some of her
body's chemical energy into magnetic energy.
If she then releases the magnet, this magnetic energy will be released as it flips
back to position \figref{signalwithmagnets}/1. She has apparently stored energy by going from
1 to 2.
So far everything is easily explained without
the concept of a field of force: the distances between the poles
are simply different in figures
1 and 2. In figure 2, for instance, the distances between the two north poles are
shorter than in figure 1. This is like a description of gravity where we speak
only of the changing distance between a rock and the earth, without referring to
a gravitational field at all.

But now imagine that the two people start in position 1
and then, at a prearranged time, flip their magnets extremely quickly
to position 3, keeping them lined up with each other the
whole time. Imagine, for the sake of argument, that they can
do this so quickly that each magnet is reversed while the
force signal from the other is still in transit. (For a more
realistic example, we'd have to have two radio antennas, not
two magnets, but the magnets are easier to visualize.)
During the flipping, each magnet is still feeling the forces
arising from the way the other magnet \emph{used} to be
oriented. Even though the two magnets stay aligned during
the flip, the time delay causes each person to feel
resistance as she twists her magnet around. How can this be?
Both of them are using up the chemical energy in their bodies.
Conservation of energy says that if this form of energy decreases,
then some other form of energy must increase. They
must be storing magnetic energy somehow. But in the
traditional Newtonian conception of matter interacting via
instantaneous forces at a distance, magnetic energy could only arise
from the relative positions of objects that are interacting
via magnetic forces. If the magnets never changed their orientations
relative to each other, how can any magnetic energy have been stored?

The only possible answer is that the energy must have gone
into the magnetic force ripples crisscrossing the space
between the magnets. Fields of force apparently carry energy
across space, which is strong evidence that they are real things.

This is perhaps not as radical an idea to us as it was to
our ancestors. We're used to the idea that a radio
transmitting antenna consumes a great deal of power, and
somehow spews it out into the universe. A person working
around such an antenna needs to be careful not to get too
close to it, since all that energy can easily cook flesh (a
painful phenomenon known as an ``RF burn'').\footnote{Many people
are also needlessly concerned that they'll get brain cancer from their
cell phones. We know enough about the physics of how these electromagnetic signals
interact with matter to be certain that they're incapable of altering a cell's
DNA to produce a cancerous mutation. Furthermore, people who work near radio
transmitters are exposed to signals that are similar, but many orders of magnitude
stronger, and they do not experience any increased incidence of cancer. One of
the most telling characteristics of pseudoscience is that it doesn't scale
properly. If the signals caused cancer, then making them much stronger should
have a much higher probability of causing cancer.}

By the way, if you retrace the logic of this section, you can verify that
in my argument that field-ripples must take time to get from one place to
another, I never used any facts that were specific to electromagnetic fields.
You could take a pen, cross out ``electromagnetic'' everywhere, and replace it
with ``gravitational'' or ``nuclear,'' and it would still be a valid argument.
Thus the thing we've been referring to as ``the speed of light'' could instead
be thought of as ``the maximum speed of anything.'' In 2002, astronomers
Sergei Kopeikin and Edward Fomalont verified that as Jupiter circles the sun,
its gravitational field travels outward from it at the speed of light. 
If the result had been to the contrary, it would have disproved relativity!
(There is some controversy about their analysis, although the result is
what everyone expected based on relativity.)


\end{envsubsection}	
%
\begin{envsubsection}{Induction}\index{induction}
Now that I've made the case for the reality of the electric and magnetic fields,
let's consider an example of their relationship to each other, one which will have
some very practical applications.

You're loafing around your apartment one afternoon, munching potato chips and
idly watching the needle on your magnetic field meter. Suddenly, the needle starts
to go up. The magnetic field in your apartment is getting stronger. You hypothesize
that someone is driving toward you with a big magnet in the back of her pickup truck.
As the magnet gets closer, you feel its field more and more strongly. Your roommate,
however, pauses her video game for long enough to offer an alternative explanation:
maybe the junkyard down the street has a big electromagnet they use for picking up
cars. According to her theory, the magnetic field is getting stronger because they're
slowly turning the knob up; the magnet isn't getting any closer at all. Your roommate
offers to bet you some take-out Chinese food that her explanation is right.

Without walking around town and investigating, how can you settle the bet?
Well, according to your explanation, the truck is coming your way. In the driver's
frame of reference, the magnet is at rest, so there's only a magnetic field, no
electric field. But the frames of reference of her truck and your couch are not
at rest relative to one another, so you know what what she perceives as a pure
magnetic field, you should see as a mixture of magnetic and electric fields.
Your theory makes a definite prediction: if you fire up your electric field meter,
you should detect something.
You offer to use such a measurement to settle the bet, but
your roomate has taken physics already, and wisely refuses. ``Look,'' she says, ``the
electric and magnetic fields are just different sides of the same coin. Doesn't it seem a little
goofy to you that there would be one relationship between the electric
and magnetic fields inside our apartment if they were from a certain kind of distant source,
but a different relationship if they came from a different type of source? No matter which
of us is right, there's going to be an electric field in this room.'' She then turns on the
electric field meter, shows you that there is an electric field, goes to the window,
opens the shades, shows you the electromagnet at the junkyard (which she already knew
about), and informs you that she'll be having kung pao chicken and ma po dofu.

There was no pickup truck with a big magnet in the back. There was nothing moving at
all. The person at the junkyard turning up the knob on the electromagnet is in the same
frame of reference as you and your roommate.
What you've just bought for the price of some Chinese food is a lesson in the principle
of induction:

\begin{important}[the principle of induction]
Any magnetic field that changes over time will create an electric field. The induced
electric field is perpendicular to the magnetic field, and forms a curly pattern around
it.\\
Any electric field that changes over time will create a magnetic field.
The induced
magnetic field is perpendicular to the electric field, and forms a curly pattern around
it.
\end{important}

The first part was discovered experimentally by
Michael Faraday in 1831. Relativity was still 70 years in the future, so the
argument made by your roommate wasn't available to Faraday --- to him, it was just
a surprising empirical fact. Since relativity tells us that electricity and
magnetism aren't really separate things, it's also not so surprising that
the second part is true.

\begin{eg}{The \index{generator}generator}
	A  basic generator, \figref{generator}, consists of a permanent magnet that
	rotates within a coil of wire. The magnet is turned by a
	motor or crank, (not shown). As it spins, the nearby
	magnetic field changes. This changing magnetic field results in an
	electric field, which has a curly pattern. This electric
	field pattern creates a current that whips around the coils
	of wire, and we can tap this current to light the lightbulb.
	
	If the magnet was on a frictionless bearing, could we light the bulb
	for free indefinitely, thus violating conservation of energy? No.
	It's hard work to crank the magnet, and that's where
	the energy comes from. If we break the light-bulb circuit, it suddenly
	gets easier to crank the magnet! This is because the current in the coil
	sets up its own magnetic field, and that field exerts a torque on the magnet.
	If we stopped cranking, this torque would quickly make the magnet stop turning.
\end{eg}

	\margup{-100mm}{\fig{generator}{A generator.}}
	\selfcheck{alternator}{When you're driving your car, the engine recharges the
	battery continuously using a device called an alternator,
	which is really just a generator. Why can't you use the
	alternator to start the engine if your car's battery is dead?}

	\margup{-20mm}{\fig{transformer}{A transformer.}}
\begin{eg}{The transformer\index{transformer}}
	It's more efficient for the electric company to transmit
	power over electrical lines using high voltages and low
	currents. However, we don't want our wall sockets to operate
	at 10000 volts! For this reason, the electric company uses a
	device called a transformer, \figref{transformer}, to
	convert everything to lower voltages and higher currents inside your
	house. The coil on the input side creates a magnetic field.
	Transformers work with alternating current (currents that reverses its
	direction many times a second), so the magnetic
	field surrounding the input coil is always changing. This
	induces an electric field, which drives a current around the output coil.
	
	Since the electric field is curly, an electron can keep gaining more
	and more energy by circling through it again and again. 
	Thus the
	output voltage can be controlled by changing the number of turns
	of wire on the output side. 
	In any case, conservation of energy guarantees
	that the amount of power on the output side must equal the
	amount  put in originally, 
	\begin{equation*}
	  (\text{input current})\times(\text{input voltage}) = 
	  (\text{output current})\times(\text{output voltage}) 
	\end{equation*}
	so no matter what factor the voltage is reduced by, the current is
	increased by the same factor.  This is analogous to a lever. A crowbar allows
	you to lift a heavy boulder, but to move the boulder a centimeter, you may have
	to move your end of the lever a meter. The advantage in force comes with a
	disadvantage in distance. It's as though you were allowed to lift a small weight
	through a large height rather than a large weight through a small height. Either way,
	the energy you expend is the same.
\end{eg}

\begin{eg}{Fun with sparks}
Unplug a lamp while it's turned on, and watch the area around the wall outlet. You should see
a blue spark in the air at the moment when the prongs of the plug lose contact with the electrical
contacts inside the socket.

This is evidence that, as discussed on page \pageref{energy-in-fields}, fields contain energy.
Somewhere on your street is a transformer, one side of which is connected to the lamp's circuit.
When the lamp is plugged in and turned on, there's a complete circuit, and current flows.
as current flows through the coils in the transformer, a magnetic field is formed --- remember,
any time there's moving charge, there will be magnetic fields. Because there is a large number turns
in the coils, these fields are fairly strong, and store quite a bit of energy.

When you pull the plug, the circuit is no longer complete, and the current stops. Once the current
has disappeared, there's no more magnetic field, which means that some energy has disappeared.
Conservation of energy tells us that if a certain amount of energy disappears, an equal amount
must reappear somewhere else. That energy goes into making the spark. (Once the spark is gone,
its energy remains in the form of heat in the air.)
\end{eg}

\end{envsubsection}	
%
\begin{envsubsection}{Electromagnetic waves}\index{electromagnetism!waves}\index{Maxwell, James Clerk}
	Theorist James Clerk Maxwell was the first to work out the principle of induction (including
	the detailed numerical and geometric relationships, which we won't go into here).
	Legend has it that
	 it was on a starry night that he first realized the most important implication of his equations:
	light itself is an electromagnetic wave, a ripple spreading outward from a disturbance
	in the electric and magnetic fields.
	 He went for a walk with his wife, and told her she was the only other person in
	 the world who really knew what starlight
	was.\index{Maxwell, James Clerk}\index{wave!electromagnetic}\index{light!as an electromagnetic wave}

	\margup{-65mm}{\fig{maxwell}{James Clerk Maxwell (1831-1879)}}
	The principle of induction tells us that there can be no such thing as a purely electric or
	purely magnetic wave. As an electric wave washes over you, you feel an electric field that
	changes over time. By the principle of induction, there must also be a magnetic field
	accompanying it. It works the other way, too. It may seem a little spooky that the electric
	field causes the magnetic field while the magnetic field causes the electric field, but the
	waves themselves don't seem to worry about it.

	 The distance from one ripple to the next
	is called the wavelength of the light.\index{wavelength} Light with a certain wavelength
	(about quarter a millionth of a meter) is at the violet end of the rainbow spectrum, while
	light with a somewhat longer wavelength (about twice as long) is red.
	Figure \subfigref{emspectrum}{1} shows the complete spectrum of light waves.
	Maxwell's equations predict that all light waves have the same structure,
	regardless of wavelength and frequency, so even though radio and x-rays, for example,
	hadn't been discovered, Maxwell predicted that such waves would have to exist.
	Maxwell's 1865 prediction passed an important test in 1888, when
	Heinrich Hertz\index{Hertz, Heinrich} published the results of experiments in which he showed
	that radio waves could be manipulated in the same ways as visible light waves. Hertz
	showed, for example, that radio waves could be reflected from a flat surface, and
	that the directions of the reflected and incoming waves were related in the same
	way as with light waves, forming equal angles with the normal. Likewise,
	light waves can be focused with a curved, dish-shaped mirror, and Hertz demonstrated
	the same thing with a dish-shaped radio antenna.

	\widefigsidecaption{emspectrum}{%
           Panel \textbf{1} shows the electromagnetic spectrum.
           \qquad Panel \textbf{2} shows how an electromagnetic wave is put together.
           Imagine that this is a radio wave, with a wavelength of a few meters. If you were standing inside
           the wave as it passed through you, you could theoretically hold a compass in your hand, and it would wiggle back
           and forth as the magnetic field pattern (white arrows) washed over you. 
           (The vibration would actually be much to rapid to detect this way.) Similarly, you'd experience
           an electric field alternating between up and down.
           \qquad Panel \textbf{3} shows how this relates to the principle of induction. The changing electric field (black arrows) should
           create a curly magnetic field (white). Is it really curly? Yes, because if we inserted a paddlewheel that
           responded to electric fields, the field would make the paddlewheel spin counterclockwise as seen from above. Similarly, the
           changing magnetic field (white) makes an electric field (black) that curls in the clockwise direction as seen from the front.%
        }\index{electromagnetism!spectrum}


\end{envsubsection}
%
\mysection{What's Left?}
One mark of wisdom is to know what it is that you don't know. Now that you're at the end
of this book, what you don't know is, roughly speaking, what physicists didn't know in
1905. Here's a bare-bones outline of what's missing from your education so far --- just
enough of a taste, I hope, to convince you to take another physics course!

First of all, I've already warned you on page \pageref{reductionism} that this book\index{reductionism}
basically ignores one main current in physics, which is reductionism. I've frequently
made use of the fact that matter is built out of atoms, but that's about it. Around 1905,
physicists learned that atoms were made out of nuclei and electrons. Shortly thereafter,
they found out that the nuclei were made out of protons and neutrons, and not long after that
they found out that neutrons and protons were themselves not fundamental:
they're made of triplets of tinier particles called quarks. If we keep breaking things into smaller
and smaller pieces, will we ever bottom out? We don't know.\footnote{There's a story about a wise sage who was
asked what held up the earth. ``Elephants,'' he replied, ``it's held up by elephants.''
When he was interrogated about what held up the elephants, he replied, ``Ah, you're
tricky, very tricky, but the answer is quite simple. It's elephants all the way down!''}
By the way, the astronomers now tell us that 90\% of the matter in the universe isn't even atoms,
so we have more mysteries to solve even without breaking ordinary atomic matter down into smaller
and smaller pieces!

Physicists also learned that there was a new type of force, the strong nuclear force,
holding the quarks together to form the protons and neutrons, and holding the neutrons and
protons together to form nuclei. Another type, the weak nuclear force, is responsible
for\index{force!strong nuclear}\index{force!weak nuclear}\index{nuclear force}\index{strong nuclear force}\index{weak nuclar force}
certain types of radioactive decay. At this stage, the list of fundamental forces was like this:
gravity, electromagnetism, strong nuclear, weak nuclear. However, later investigations showed
that the weak nuclear force could be unified with electromagnetism in the same way that
electricity was unified with magnetism, resulting in a single thing referred to as the
electroweak force. The list is therefore down to three interactions: gravitational, electroweak,
and strong nuclear. Many physicists would dearly love to get the three down to one.

It might seem like everything was getting pretty tidy, but there was this one crazy experimental
fact that wouldn't go away: sometimes, it seemed, physics was random. For instance, take
two atoms of the element uranium 238, which occurs naturally in the earth's crust. (The number
238 means that the number of protons plus the number of neutrons equals 238.) This
element undergoes radioactive decay, but which atom will decay first? The answer is that we
can't tell. It's random. At first, physicists assumed that this apparent randomness was
just caused by some complicated unknown mechanism inside the nucleus. Once the mechanism
was understood, everything would be perfectly predictable. Physicists wanted to preserve
their determinism, which they'd been cherishing ever since Laplace's famous claim in 1776 that
``Given for one instant an intelligence which could comprehend all the forces by which nature
is animated and the respective positions of the things which compose it...nothing would be
uncertain, and the future as the past would be laid out before its eyes.''

As they dug deeper,
however, they uncovered more randomness, not less (much to the discomfort of Einstein,\index{Einstein, Albert!randomness}
who kibitzed that he could never believe God would ``play dice'').
Eventually they realized that the randomness was not evidence of something distasteful
and complicated, but rather of something simple and beautiful. In this chapter, we've
developed a picture in which there are two types of actors on the stage:
particles and fields. Both can have energy, and both can travel from place to place,
but they seem fundamentally different in many ways. Isn't this a little ugly?
The deeper, more beautiful truth is that the particles are also fields, and the fields
are also particles. Just as light is a ripple, so is an electron! (A ripple in what? Don't
ask --- you won't get a satisfying answer.) You yourself are a wave, but your wave properties
aren't ordinarily evident because you're so big. A wave, for instance, has fuzzy edges.
Your body has fuzzy edges, but the fuzziness is on a microscopic scale, so you don't notice
it. All the basic building blocks of the universe are like this: they're both waves and
particles, at the same time. It's a little like Christian theology: Jesus is both fully\index{Jesus}
human and fully divine.

Here's how this wave-particle dualism relates to randomness. Suppose you're sitting inside
at night, next to a window with the curtains open. People outside can see you, which means
that their eyes are getting light from your body. But you can also see your own reflection
in the window, which means that while a certain percentage of the light energy gets out,
there's also a certain percentage that's reflected back in. Waves always behave this way.
For simplicity, let's imagine that 50\% of the light is being reflected, while the other
50\% gets out.\index{waves}

But everything is both a wave and a particle, right? So a light wave coming from your
body to the window has a certain granularity to it. It's made out of little chunks, like
a stream of bullets from a machine gun. Now what if we send out a single light-particle
all by itself? Remember, it's both fully waveish and fully particleful. Since it's a
wave, it behaves like every law-abiding wave: when it hits the window, it splits up
into two weaker waves, each one carrying half the energy. But wait --- it's also
a particle. How can you have half a particle? You can't. This is where the randomness
comes from. The half-strength reflected wave represents a 50\% probability that the
particle will be reflected, and likewise for the half-strength wave that gets through.\index{probability}

Now that I've told you what I'd known that you hadn't, let me finish up by telling you a question
that nobody knows the answer to. The wave-particle theory works great, and forms the
theoretical basis for such practical devices as the laser that makes your CD player
work. Relativity is also a highly successful theory. Special relativity passed a vast
number of experimental tests, and in recent decades, so has general relativity, the
version of the theory that includes gravity. General relativity is programmed into
GPS (the global positioning system) for example. Now here's the problem: as far as
we can tell, the wave-particle theory (called quantum mechanics) is logically
inconsistent with relativity. Nobody knows how to reconcile them. This presumably
means that they're both only approximations to some deeper, underlying theory, but
we don't know what that theory is. If we can find it, we'll probably also learn the
answers to some intriguing questions. What did the universe look like a gazillionth
of a second after the big bang, and how did that give rise to the universe we inhabit
today, with its clusters of galaxies separated by vast oceans of emptiness? Is time
travel possible? What happens if you fall into a black hole?

\begin{hwsection}

\begin{hw}{muonic}
	A hydrogen atom consists of an electron and a proton. For our
	present purposes, we'll think of the electron as orbiting in
	a circle around the proton.
	
	The subatomic particles called muons behave exactly like
	electrons, except that a muon's mass is greater by a factor
	of 206.77.  Muons are continually bombarding the Earth as
	part of the stream of particles from space known as cosmic
	rays.  When a muon strikes an atom, it can displace one of
	its electrons.  If the atom happens to be a hydrogen atom,
	then the muon takes up an orbit that is on the average
	206.77 times closer to the proton than the orbit of the
	ejected electron.  How many times greater is the electric
	force experienced by the muon than that previously
	felt by the electron?
\end{hw}


\marg{\fig{hw-many-measurements}{Problems \ref{hw:many-v-measurements} and \ref{hw:many-i-measurements}.}}
\begin{hw}{many-v-measurements}
(a) Consider the waterfall metaphor introduced in figure \ref{fig:waterwheels}
on page \pageref{fig:waterwheels}, in which voltage differences are represented by
height differences. In this metaphor, how would you represent a piece of wire?\\
(b) The figure shows a circuit containing five lightbulbs connected to a battery.
Suppose you're going to connect one probe of a voltmeter to the circuit at
the point marked with a dot. How many unique, nonzero voltage differences
could you measure by connecting the other probe to other wires in the circuit?
Visualize the circuit using the same waterfall metaphor.
\end{hw}

\hwnote{Problem \ref{hw:many-i-measurements} is meant to be done after lab
\ref{ch:em}\ref{lab:is-charge-conserved}.
}

\begin{hw}{many-i-measurements}
The lightbulbs in the figure are all identical. If you were inserting an ammeter
at various places in the circuit, how many unique currents could you measure?
If you know that the current measurement will give the same number in more than
one place, only count that as one unique current.
\end{hw}

\begin{hw}{magtimereversal}
Albert Einstein wrote, ``What really interests me is whether God had any
choice in the creation of the world.'' What he meant by this is that if you randomly
try to imagine a set of rules --- the laws of physics --- by which the universe
works, you'll almost certainly come up with rules that don't make sense. For instance,
we've seen that if you tried to omit magnetism from the laws of physics, electrical
interactions wouldn't make sense as seen by observers in different frames of reference;
magnetism is required by relativity.

The magnetic interaction rules in figure
\figref{magtwobody} are consistent with the time-reversal symmetry of the laws of
physics. In other words, the rules still work correctly if you reverse the
particles' directions of motion. Now you get to play God (and fail).
Suppose you're going to make an alternative version of the laws of physics
by reversing the direction of motion of only \emph{one} of the eight particles.
You have eight choices, and each of these eight choices would result in a new
set of physical laws. We can imagine eight alternate universes, each governed
by one of these eight sets. Prove that \emph{all} of these modified sets of
physical laws are impossible, either because the are self-contradictory, or
because they violate time-reversal symmetry.
\end{hw}

\begin{hw}{alienmagnetism}\index{Ozma problem!for magnetism}
Discussion question \ref{dq:ozma} on page \pageref{dq:ozma} introduced the general
concept of an Ozma problem. Here is an Ozma problem for magnetism. Suppose we establish
communication with aliens, and we want to tell them how we define the direction of the
magnetic field. Can we explain to them how to eliminate the ambiguities 
described on page \pageref{arbitrarydirectionofmag}? How is this related to the
Ozma problems for charge and for left and right?
\end{hw}

\begin{hw}{forcebetweencurrents}
The purpose of this problem is to show that the magnetic interaction rules shown in figure
\figref{magtwobody} can be simplified by stating them in terms of current. 
Recall that, as discussed in discussion question \ref{dq:signsofcurrent} on page
\pageref{dq:signsofcurrent}, one type of charge moving in a particular direction produces
the same current as the other type of charge moving in the opposite direction. 
Let's say arbitrarily that the current made by the dark type of charged particle is
in the direction it's moving, while a light-colored particle produces a current in the
direction opposite to its motion. Redraw all four panels of figure \figref{magtwobody},
replacing each picture of a moving light or dark particle with an
arrow showing the direction of the current it makes. Show that the rules for attraction and
repulsion can now be made much simpler, and state the simplified rules explicitly.
\end{hw}

\begin{hw}{feynmanantimatterbackintime}
Physicist Richard Feynman\index{Feynman, Richard} originated a new way of thinking about charge: a charge of a certain
type is equivalent to a charge of the opposite type that happens to be moving backward
in time! An electron moving backward in time is an antielectron --- a particle that has the same
mass as an electron, but whose charge is opposite. Likewise we have antiprotons, and antimatter
made from antiprotons and antielectrons. Antielectrons occur naturally everywhere around you due to
natural radiactive decay and radiation from outer space. A small number of antihydrogen atoms has even
been created in particle accelerators!

Show that, for each rule for magnetic interactions shown in \figref{magtwobody}, the rule is
still valid if you replace one of the charges with an opposite charge moving in the 
opposite direction (i.e., backward in time).
\end{hw}

\begin{hw}{analyzecircularorbit}
Refer to figure \figref{circularorbit} on page  \pageref{fig:circularorbit}. Electrons have the
type of charge I've been representing with light-colored spheres.\\
(a) As the electrons in the beam pass over the top of the circle,
what is the direction of the force on them? Use what you know about circular motion.\\
(b) From this information, use figure \figref{righthandrule}
on page \pageref{fig:righthandrule} to determine the direction of the magnetic field
(left, right, up, down, into the page, or out of the page).
\end{hw}

\begin{hw}{microscopy}
You can't use a light wave to see things that are smaller than the wavelength of the light.\\
(a) Referring to figure \figref{emspectrum} on
page \pageref{fig:emspectrum}, what color of light do you think would be the best to use
for microscopy?\hwendpart
(b) The size of an atom is about $10^{-10}$ meters. Can visible light be used to make images
of individual atoms?
\end{hw}

\marg{\fig{standing-waves}{Stationary wave patterns on a clothesline (problem \ref{hw:microwave-oven}).}}
\begin{hw}{microwave-oven}
You know how a microwave gets some parts of your food hot, but leaves other parts cold?
Suppose someone is trying to convince you of the following explanation for this fact:
\emph{The microwaves inside the oven form a stationary wave pattern, like the vibrations
of a clothesline or a guitar string. The food is heated unevenly because
the wave crests are a certain
distance apart, and the parts of the food that get heated the most are the ones where there's
a crest in the wave pattern.} Use the wavelength scale in figure \figref{emspectrum} on
page \pageref{fig:emspectrum} as a way of checking numerically
whether this is a reasonable explanation.
\end{hw}

\begin{hw}{correspondence-principle-with-light}
This book begins and ends with the topic of light. Give an example
of how the correspondence principle applies here, referring to
a concrete observation from a lab.
\end{hw}
\end{hwsection}

%========================================== labs ===============================================
%--------------------------------- charge lab --------------------------------------
\begin{lab}{Charge}\index{charge!number of types}\index{charge!conservation of}\label{lab:charge}

\apparatus
\equip{scotch tape}\\
\equip{rubber rod}\\
\equip{heat lamp}\\
\equip{fur}\\
\equip{bits of paper}

\goal{Determine the qualitative rules governing electrical charge and forces.}

\labintroduction

Newton's law of gravity gave a mathematical formula for the
gravitational force, but his theory also made several
important non-mathematical statements about gravity:

\begin{itemize}
\item[] Every mass in the universe attracts every other mass in the universe.

\item[] Gravity works the same for earthly objects as for heavenly bodies.

\item[] The force acts at a distance, without any need for physical contact.

\item[] Mass is always positive, and gravity is always attractive, not repulsive.
\end{itemize}

The last statement is interesting, especially because it
would be fun and useful to have access to some negative
mass, which would fall up instead of down (like the
``upsydaisium'' of Rocky and Bullwinkle fame).

Although it has never been found, there is no theoretical
reason why a second, negative type of mass can't exist. 
Indeed, it is believed that the nuclear force, which holds
quarks together to form protons and neutrons, involves three
qualities analogous to mass. These are facetiously referred
to as ``red,'' ``green,'' and ``blue,'' although they have
nothing to do with the actual colors. The force between two
of the same ``colors'' is repulsive: red repels red, green
repels green, and blue repels blue. The force between two
different ``colors'' is attractive: red and green attract
each other, as do green and blue, and red and blue.

When your freshly laundered socks cling together, that is an
example of an electrical force. If the gravitational force
involves one type of mass, and the nuclear force involves
three colors, how many types of electrical ``stuff'' are
there? In the days of Benjamin Franklin, some scientists\index{Franklin, Benjamin}
thought there were two types of electrical ``charge'' or
``fluid,'' while others thought there was only a single
type. In this lab, you will try to find out experimentally
how many types of electrical charge there are.

\labobservations

Stick a piece of scotch tape on a table, and then lay
another piece on top of it. Pull both pieces off the table,
and then separate them. If you now bring them close
together, you will observe them exerting a force on each
other. Electrical effects can also be created by rubbing the
fur against the rubber rod.

Your job in this lab is to use these techniques to test
various hypotheses about electric charge. The most common
difficulty students encounter is that the charge tends to
leak off, especially if the weather is humid. If you have
charged an object up, you should not wait any longer than
necessary before making your measurements. It helps if you
keep your hands dry.

\labpart{Repulsion and/or attraction}

Test the following hypotheses. Note that they are
mutually exclusive, i.e. only one of them can be true.

A1) Electrical forces are always attractive.

A2) Electrical forces are always repulsive.

A3) Electrical forces are sometimes attractive and
sometimes repulsive.

Interpretation: Once you think you have tested these
hypotheses fairly well, discuss with your instructor what
this implies about how many different types of charge there might be.

\labpart{Are there forces on objects that have not been specially prepared?}

So far, special preparations have been necessary in order to
get objects to exhibit electrical forces. These preparations
involved either rubbing objects against each other (against
resistance from friction) or pulling objects apart (e.g.
overcoming the sticky force that holds the tape together).
In everyday life, we do not seem to notice electrical forces
in objects that have not been prepared this way.

Now try to test the following hypotheses. Bits of paper are
a good thing to use as unprepared objects, since they are
light and therefore would be easily moved by any force.
\emph{Do not} use tape as an uncharged object, since it can
become charged a little bit just by pulling off the roll.

B1) Objects that have not been specially prepared are immune
to electrical forces.

B2) Unprepared objects can participate in electrical forces
with prepared objects, and the forces involved are always attractive.

B3) Unprepared objects can participate in electrical forces
with prepared objects, and the forces involved are always repulsive.

B4) Unprepared objects can participate in electrical forces
with prepared objects, and the forces involved can be either
repulsive of attractive.

Hypotheses B1 through B4 are mutually exclusive.

Interpretation: If you think your observations support a
hypothesis other than B1, discuss with your instructor
whether the forces seem to obey the rule given on page
\pageref{thirdlaw} about forces occurring in equal-strength
pairs, and
discuss why an unprepared object might participate
in electrical forces.

\labpart{Rules of repulsion and/or attraction and the
number of types of charge}

Test the following mutually exclusive hypotheses:

C1) There is only one type of electric charge, and the force
is always attractive.

C2) There is only one type of electric charge, and the force
is always repulsive.

C3) There are two types of electric charge, call them X
and Y. Like charges repel (X repels X and Y repels Y)
and opposite charges attract (X and Y attract each other).

C4) There are two types of electric charge. Like charges
attract and opposite charges repel.

C5) There are three types of electric charge, X, Y and
Z. Like charges repel and unlike charges attract.

The only way to keep all your observations straight is
to make a table, in which the rows and columns correspond
to the different objects you're testing against each other
for attraction and repulsion. To test C3 versus C5, you'll
need to see if you can successfully explain your whole
table by labeling the objects with only two labels, X and Y.

Discuss your conclusions with your instructor.

\labpart{Creation, transfer, and/or conservation of charge}

Test the following mutually exclusive hypotheses:

D1) Charge can be created, destroyed, or transferred without
any particular restrictions. 

D2) Putting a certain type of charge on one object always
involves putting equal amounts of the other type(s) of
charge on some other object.

Discuss with your instructor whether your conclusion can be
put in the form of a conservation law. Conservation laws in
physics state that if you add up how much there is of
something in a closed system, then that total amount can't
change as long as the system stays closed.

You will revisit this issue, using a much more accurate
technique, in lab \ref{ch:em}\ref{lab:is-charge-conserved}.

\labselfcheck

The following are examples of incorrect reasoning about this
lab. As a self-check, it would be a very good idea to figure
out for yourself in each case why the reasoning is logically
incorrect or inconsistent with Newton's laws. You do not
need to do this in writing --- it is just to help you
understand what's going on. If you can't figure some of them
out, ask your instructor before leaving lab.

(1) ``The first piece of tape exerted a force on the second,
but the second did not exert a force on the first.''

(2) ``The first piece of tape repelled the second, and the
second attracted the first.''

(3) ``I observed three types of charge: two that exert
forces, and a third, neutral type.''

(4) ``The piece of tape that came from the top was positive,
and the piece from the bottom was negative.''

(5) ``One piece of tape had electrons on it, and the other
had protons on it.''

(6) ``I know there were two types of charge, not three,
because we observed two types of interactions, attraction
and repulsion.''

\labwriteup

Explain what you have concluded about electrical charge and
forces. Base your conclusions on your data!

\end{lab}

%--------------------------------- electrical measurements lab --------------------------------------
\begin{lab}{Electrical Measurements}

\apparatus
\equip{banana-plug cables}\\
\equip{alligator clips}\\
\equip{DC power supplies}\\
\equip{batteries}\\
\equip{HP multimeters}\\
\equip{2-amp fuses}
\equip{lightbulbs and sockets}

\goal{Learn how to measure current and voltage.}

\labsection{Starting Out}

Let's start out by taking a battery, a lightbulb, and two wires, and trying to make
the bulb light up. Note that the bulb has two metal contacts: one at the tip, and another
consisting of the metal screw threads. Once you get it to work, draw a circuit diagram.

\newcommand{\spacefordiagram}{\vspace{40mm}}
\spacefordiagram

See if you can get it to work by hooking things up in different ways, and see if you can
come up with a statement about what conditions are necessary in order to make it work:

\spacefordiagram

In the rest of the lab, you'll think about a circuit, predict how it will behave, and
then test your prediction. Your prediction should say whether the lightbulbs light up,
and if you expect that a bulb will be brighter or dimmer than normal, you should also
say that.

\labsection{Measuring Voltage and Current}

From now on, it will be more convenient to use the DC power supply instead of the battery.
While you're hooking up the circuit, turn the knob all the way down. You can stick the
banana-plug cables directly into the top two terminals of the power supply. (Don't use
the ground terminal at the bottom, which isn't meant to be a current-carrying connection.)
To connect them to the screw heads on the lightbulb socket, use the alligator clips.

Turn up the power supply until you can just barely see the lightbulb starting to glow.
Use the voltmeter to measure the voltage difference across the lightbulb. A multimeter
can be used to measure either current or voltage. To measure voltage, put the switch
on a voltage scale, and connect wires to the V and COM (common) plug. The common plug
is the one that's always used for every type of measurement, hence the name.
Figure
\figref{voltmeteruseschematic} on page \pageref{fig:voltmeteruseschematic} shows
the right way to connect the meter to the circuit.
Record your data in the table on the next page.

Does it make any difference if you touch the voltmeter's probes to the terminals of
the power supply rather than the screwheads on the lightbulb socket?

Now disconnect the multimeter from the circuit, and change the switch so it's
on a current (amps) scale. 
Use it to measure the current, as shown in figure
\figref{ammeteruseschematic} on page \pageref{fig:ammeteruseschematic}.
If you mess up, you may blow a fuse in the meter. To avoid the hassle of replacing
the fuse, you may want to turn off the power supply while you set up for the
measurement. When you think you're ready to go, look carefull at what would happen
to an electron that came out of the power supply. Would it ever come to a fork
in the road and have a choice of whether to go through the meter or the bulb? If
so, then you've hooking things in a way that won't work, and that will blow the fuse.

Now repeat the same set of measurements with the voltage turned up higher, so the
lightbulb glows more brightly. 

\begin{tabular}{|p{20mm}|p{20mm}|p{20mm}|}
\hline
\emph{voltage (volts)} & \emph{current (amps)} & \emph{resistance (ohms)}\\
\hline
\raisebox{0mm}[0mm][12mm]{} & & \\
\hline
\raisebox{0mm}[0mm][12mm]{} & &\\
\hline
\end{tabular}

Is it possible to find a single, consistent value
for the resistance of the lightbulb?


\end{lab}
%--------------------------------- is charge conserved? lab --------------------------------------
\begin{lab}[0]{Is Charge Conserved?}\index{charge!conservation of}\label{lab:is-charge-conserved}

\apparatus
\equip{wires}\\
\equip{banana-plug cables}\\
\equip{alligator clips}\\
\equip{DC power supplies}\\
\equip{multimeters}\\
\equip{resistors}

\goal{Find out whether charge is conserved.}

In lab \ref{ch:em}\ref{lab:charge}, you made a crude test of whether charge was conserved. In this
lab, you'll make an accurate numerical test.

In the circuit diagram below, the zigzag lines represent resistors. Get two different
resistors with two different values, both in the kiloohm range, and assemble the
circuit.

\labfig{lab-junction}

At how many places in the circuit is it possible to measure the current? Are any of the
possibilities redundant? Now go ahead and measure all these currents.

Do your results support conservation of charge, or not?

Notes: (1) The plus and minus signs of the current readings on the meter are only
meaningful if you take into account which way the meter is hooked into the circuit ---
if you reverse the meter's two connections, you'll get the opposite sign.
(2) Make sure to record the units of the currents. Note that the meter may read in
units of $\mu\zu{A}$ (microamps), mA (milliamps), or A (amps), depending on the
scale you're using.
\end{lab}
%--------------------------------- circuits lab --------------------------------------
\begin{lab}{Circuits}

This lab is based on one created by Virginia Roundy.

\apparatus
\equip{batteries}\\
\equip{lightbulbs and holders}\\
\equip{wire}\\
\equip{highlighting pens, 3 colors}

\goal{Apply four methods of thinking about circuits.}

\labintroduction

When you first glance at this lab, it may look scary and intimidating --- all those
circuits! It's not that bad once you understand the symbols:

\labfig{lab-circuit-symbols}

Also, all those wild-looking circuits can be analyzed using the following four
guides to thinking:

1. \emph{A circuit has to be complete\/}, i.e., it must be possible for charge to get recycled
as it goes around the circuit. If it's not complete, then charge will build up at a dead end. This built-up
charge will repel any other charge that tries to get in, and everything will rapidly grind to a stop.

2. \emph{There is constant voltage everywhere along a piece of wire.} To apply this rule during
this lab, I suggest you use the colored highlighting pens to mark the circuit. For instance, if there's
one whole piece of the circuit that's all at the same voltage, you could highlight it in yellow.
A second piece of the circuit, at some other voltage, could be highlighted in blue.

3. \emph{Charge is conserved,} so charge can't ``get used up.''

4. When in doubt, use a \emph{rollercoaster diagram}, like the one shown below. On this
kind of diagram, height corresponds to voltage --- that's why the wires are drawn as horizontal
tracks.

\labfig{lab-circuit-rollercoaster}

\vfill\pagebreak[4]


\labsection{A Bulb and a Switch}

Look at circuit 1, and try to predict what will happen when the switch is open, and what
will happen when it's closed. Write both your predictions below before you build the circuit.
When you build the circuit, you don't need an actual switch like a light switch; just 
connect and disconnect the banana plugs.


\newcommand{\circuittablestrut}{\raisebox{0mm}[0mm][25mm]{}}
\newcommand{\circuittabletwocols}[2]{
\begin{tabular}{|p{20mm}|p{42mm}|}
\hline
         & \emph{#1} \\
\hline
prediction & \\
\hline
explanation  \circuittablestrut &\\
\hline
observation & \\
\hline
explanation  & \\
(if different) \circuittablestrut & \\
\hline
\end{tabular}

\begin{tabular}{|p{20mm}|p{42mm}|}
\hline
         & \emph{#2} \\
\hline
prediction & \\
\hline
explanation \circuittablestrut  &\\
\hline
observation  & \\
\hline
explanation  & \\
(if different) \circuittablestrut & \\
\hline
\end{tabular}
}
\newcommand{\circuittableonecol}{
\begin{tabular}{|l|p{50mm}|}
\hline
prediction  &\\
\hline
explanation \circuittablestrut   &\\
\hline
observation &\\
\hline
explanation  & \\
(if different) \circuittablestrut &\\
\hline
\end{tabular}
}

\labfigcaption{lab-circuit-simple}{Circuit 1}

\vfill\pagebreak[4]

\circuittabletwocols{switch open}{switch closed}

Did it work the way you expected? If not, try to figure it out with the benefit of
hindsight, and write your explanation in the table above.

\vfill\pagebreak[4]

\labfigcaption{lab-circuit-switch-shorts}{Circuit 2 (Don't leave the switched closed for a long time!)}

\circuittabletwocols{switch open}{switch closed}

\vfill\pagebreak[4]

\labfigcaption{lab-circuit-irrelevant}{Circuit 3}

\circuittabletwocols{switch open}{switch closed}

\vfill\pagebreak[4]

\labfigcaption{lab-circuit-far-side}{Circuit 4}

\circuittabletwocols{switch open}{switch closed}

\vfill\pagebreak[4]

\labsection{Two Bulbs}

Try a rollercoaster diagram on this one!

\labfigcaption{lab-circuit-series}{Circuit 5}

\circuittabletwocols{bulb a}{bulb b}

\vfill\pagebreak[4]

\labfigcaption{lab-circuit-corner}{Circuit 6}

\circuittabletwocols{bulb a}{bulb b}

\vfill\pagebreak[4]

\labsection{Two Batteries}

Circuits 7 and 8 are both good candidates for rollercoaster
diagrams.

\labfigcaption{lab-circuit-two-batteries}{Circuit 7}

\circuittableonecol

\labfigcaption{lab-circuit-reversed}{Circuit 8}

\circuittableonecol

\labsection{A Final Challenge}

\labfigcaption{lab-circuit-inside}{Circuit 9}

\circuittabletwocols{bulb a}{bulb b}

\end{lab}


%--------------------------------- electric fields lab --------------------------------------
\begin{lab}{Electric Fields}\index{electric field!lab}
\apparatus
\equip{board and U-shaped probe   ruler}\\
\equip{DC power supply (Thornton)}\\
\equip{multimeter}\\
\equip{scissors}\\
\equip{stencils for drawing electrode shapes on paper}

\goal{To be better able to visualize electric fields and
understand their meaning.}

\labintroduction

The gravitational field is something we experience every day, but the electric field
isn't usually quite as dramatic, except if you happen to get caught outside in a thunderstorm!
Visualizing the electric field is more of a challenge, because we don't feel it physically,
and it's also not usually uniform, as the gravitational field approximately is.

Let's imagine a method for measuring the gravitational field. First you pick a certain point
in space, let's say a point on the ceiling. Then you try to locate all the other points where an
object would have the same gravitational energy. You'll find out that the all the other points
on the ceiling have this property; a mass can be moved from any point on the ceiling to any
other point without having to work against gravity. We call this an equal-energy surface.

Next, we drop a 1-kg mass from the
ceiling, and watch how far it has to fall before it's converted one joule worth of its
gravitational energy into kinetic energy. Since the earth's gravitational field is about
10 (in units of joules per kilogram per meter), this will happen when the mass has fallen
about 1/10 of a meter. This new point is part of a new equal-energy surface 1/10 of a meter
below the ceiling. We could continue this way until we'd constructed enough equal-energy
surfaces to reach the floor; they'd be like the layers of a cake.

\enlargethispage{-\baselineskip}

Note how the field's strength is related to the distance between the equal-energy surfaces.
Since the field's strength is a relatively big number, 10, the distance between the
equal-energy surfaces is a relatively small number, 1/10. In general, the greater the field
strength, the closer the spacing between the surfaces. If you've ever gone hiking and
used a topographical map, the concept is similar: the closer together the contour lines
are, the steeper the slope.

\enlargethispage{-\baselineskip}

\labfigcaption{lab-topo-map}{Each contour line on the map represents a set of points that are
all at the same elevation. Where the contour lines are close together, the slope is steep.
Notice how the streams run perpendicular to the contour lines. (19th century USGS map)}

That tells us the strength of the field, but what about the direction? As suggested
by the streams in the figure, the direction of the field is perpendicular to the equal-energy
surfaces.

This is essentially what you're going to do in this lab using electrical fields, with
a few differences. One difference is that rather than releasing a charged particle from
point A and watching it accelerate to point B --- not very practical! --- you'll 
send it from point A through a voltmeter to point B. The other difference is that
the experiment will be two-dimensional, not three-dimensional, so you'll end up with
a flat map very much like the figure above.


\labsection{Method}

Turn your  board upside down. Find the board
with pattern 1 on it, and screw
it to the underside of the board, with the
black side facing outward. Now connect the voltage source
(using the provided wires) to the two large screws on either
side of the board. Adjust the voltage source to give 8 volts.

\labfigcaption{lab-efield-patterns}{You'll use pattern 1 plus
one other pattern. The dark areas represent parts of the board
that are conductive.}

Once you turn this voltage on, charges flow between the
connections on the field plate under the
board. Two of the conductors in your pattern are connected
directly to the voltage source, so these will be two of your
constant-voltage curves, differing from each other by 8 volts. You
can select one of these as your reference voltage level, so
it is by definition at $V=0$ V, and other is at $V=8$ V. One
of the probes of your voltmeter can be connected to the 0-V
conductor indirectly, simply by connecting it to the
appropriate terminal of the voltage supply.

Now look at your U-probe. It has a conductor at the end of
the bottom part and a wire going through the bottom part
that connects to the screw at the back end of it. It also
has a hole in the end of the top part that is directly above
the end conductor on the bottom. You will be connecting one
side of the voltmeter to the screw on the U-probe and the
other to a fixed reference point of your choice. 

Place a sheet of paper on the board.
If you press down on the board, you can slip the paper
between the board and the four buttons you see at the
corners of the board. Now put the U-probe in place so that
the top is above the board and the bottom of
it is below the board. You will first be looking for places
on the pattern board where the voltage is one volt --- look
for places where the meter reads 1.0 and mark them through
the hole on the top of your U-probe with a pencil or pen.
You should find a whole bunch of places there the voltage
equals one volt, so that you can draw a nice equal-energy
curve connecting them. (If the line goes very far or curves
strangely, you may have to do more.) You can then repeat the
procedure for 2 V, 3 V, and so on.  Label each
constant-voltage curve.

Draw the electric field using arrows, with longer arrows
to represent stronger fields.

Repeat this procedure with another pattern.

\end{lab}

%--------------------------------- magnetic fields lab --------------------------------------
\begin{lab}{Magnetic Fields}\index{magnetic field!lab}
\apparatus
\equip{bar magnet}\\
\equip{old computer monitor}\\
\equip{coil of wire (solenoid)}\\
\equip{DC power supply}\\
\equip{magnetic compass}\\
\equip{2 neodymium magnets}\\
\equip{1 tiny neodymium magnet}\\
\equip{index cards}\\
\equip{vacuum tube (Nakamura EM-1N)}

\begin{goals}

\item[] Observe magnetic field patterns.

\item[] Discover the laws of nature governing sources and sinks in the magnetic field.

\end{goals}

\labpart{Deflecting a beam of electrons}

Position the bar magnet as suggested in the figure, to the left of some
identifying point on the monitor such as the mouse cursor. Use a monitor
that your instructor has designated --- some types of monitors may be permanently
damaged by this experiment! Mark the top end of the magnet with some masking tape
so you won't forget which way you held it.

\labfig{lab-fields-crt}

\newcommand{\fieldslabshortanswer}{\_\_\_\_\_\_\_\_\_\_\_\_\_\_\_\_}

The first thing you'll notice is that the screen breaks out in psychedelic colors.
The moving charges inside the magnet are interacting with the moving charges in the
beam of electrons shooting from the back of the tube to the front.
The phosphor coating inside a color monitor consists of red, green, and blue dots,
and the beam isn't hitting the dots it normally would. Once you get done admiring
the pretty colors, the point here is to observe the direction in which the beam
is deflected. Is it attracted to the magnet, repelled by it, or deflected up or
down? \fieldslabshortanswer

Now think about --- but don't yet do! --- the following experiment. What do you
think will happen if you bring the magnet over to the right side? \fieldslabshortanswer

OK, now try it. What really happened? \fieldslabshortanswer

\labpart{Two more examples}
The setup with the computer monitor can be a little awkward, because you can't
stick your hand inside the tube, near the beam. I'll keep on drawing that
setup, but there's another one that you may find more convenient. In this
alternative setup, we have a smaller bulb-shaped vacuum that isn't hidden
inside a plastic box like the computer monitor. The beam goes up rather than
horizontally as it comes out of the gun, but other than that it's very similar
to the monitor. 
(You'll have to imagine all of the figures in the lab manual
as views from above the tube.)
With this setup, you need to be careful not to magnetize
the metal parts inside the tube, so use one of the tiny neodymium magnets,
the size of a ladybug. Even though these magnets are relatively small and weak,
you can deflect the beam a lot with them, because the setup allows you to
place the magnet close to the beam. Tape it on the end of a pencil like this:

\labfig{lab-magnet-on-pencil}

Now bring the magnet in from above and below:\\
above: \fieldslabshortanswer\\
below: \fieldslabshortanswer\\

Confusing, isn't it?

\labpart{The left-hand rule}

Now let's try to make some sense out of your data. As suggested by the figure below,
you've probed the magnetic field in four regions near the magnet. For instance, when
you held the magnet to the left of the beam, you were finding out about the magnetic
field to the magnet's right.

\labfigcaption{lab-fields-cross}{Record the directions of the forces you observed.}

Now a magnetic field, unlike an electric or gravitational field, doesn't lie along
the same line as the force it creates. First, in the diagram above, organize your
information about the directions of the \emph{forces}.

\labfigcaption{lab-fields-cross}{Infer the directions of the magnetic fields.}\index{magnetic field!of a bar magnet}

Now use figure \figref{righthandrule} on page \pageref{fig:righthandrule} to figure
out the direction of the bar magnet's \emph{field} in each of these regions, and draw arrows
in the boxes to represent those directions. Note that there are two different geometric
rules, one for each type of charge. For our purposes it doesn't really matter which
type of charge we assume an electron has, but to make life easier for your instructor,
let's all be consistent: assume that the electrons have a charge corresponding to the
lighter-colored particle in the diagram, so that you're using the left-hand rule, not
the right-hand one.

Suppose someone tells you that this supposed magnet really isn't a magnet at all --- it's
simply a piece of metal with some electric charge on it. This person claims that the
forces you're observing are really electrical, not magnetic. Try to evaluate this claim
based on the data you already have:

\newcommand{\fieldslabonelineforwriting}{\_\_\_\_\_\_\_\_\_\_\_\_\_\_\_\_\_\_\_\_\_\_\_\_\_\_\_\_\_\_\_\_\_\_\_\_\_\_\_\_\_\_\_\_\_\_\_\_\_\_\_\_\_\_}
\newcommand{\fieldslabspaceforwriting}{\fieldslabonelineforwriting\\ \fieldslabonelineforwriting}
\fieldslabspaceforwriting

\labpart{Checking with a compass}

Use the magnetic compass to check the field pattern you've inferred above. You may also be
interested in finding out what the magnetic field does in regions you haven't mapped.
For instance, what do you think the field would be like in the region diagonally
above the magnet and to its right?

\labpart{Charge going in circles}

If you believe figure \figref{barmagnethanging}/2 on page \pageref{fig:barmagnethanging}, then
the bar magnet has little charges inside it going around in circles. How do you know this isn't
just a fairy tale? One good way to test this claim would be to observe the magnetic field made
by an electric current going around a circular loop of wire, and see if it resembles the field
pattern of the bar magnet. Another interesting possibility is to investigate the field pattern
inside the loop --- there was no way to probe the magnetic field inside the bar magnet!

\labfig{lab-fields-solenoid}

It turns out that you need quite a large amount of current to get a measurable field from a single
loop of wire. Rather than using large currents, and risking killing off too many students,
we'll make a stronger field by using a spool of wire with hundreds of turns on it. This is known
as a solenoid.\index{solenoid}\index{magnetic field!of a solenoid} Hook up the solenoid to
the power supply to form a complete circuit. (Don't use the ground plug at the bottom of the
power supply --- it's not meant to be a current-carrying connection.) You can crank up the
current all the way.
\footnote{To start out with, you may enjoy playing with the neodymium magnet in the space in and around
the solenoid. Fun!}

The blank figure above is for you to record your observations.
Lay the coil on the desk so that it's oriented like a tunnel (not like a coffee mug).
Note that the compass can only respond to horizontal magnetic fields. Therefore
you can only probe the magnetic field in the horizontal plane cutting through the center
of the coil, where we know by symmetry that the magnet's field is purely horizontal.
(Since the coil has symmetry with respect to rotation about its central axis, determining
the field in this plane also suffices to determine its field everywhere in space.)


Map out the field. Does the field outside the coil make the same kind of pattern as the
one you observed with the bar magnet?

\labpart{Sources and sinks?}

An important feature of any field is its sources and sinks. A sink is where all the field
arrows converge on one point, like water going down the drain --- the earth is a sink of
the gravitational field. A source is the same thing, but in reverse. The two types of
charge form the sources and sinks of the electric field.

We can now imagine two possible hypotheses:

1. The magnetic field has sources and sinks. For example, one pole of a bar magnet
is a source, and the other is a sink.

2. The magnetic field has no sources or sinks.

Think about how your data from the bar magnet and the coil relate to this.
You'll see that there's a bit of ambiguity, since you can't probe the field inside
the bar magnet, so we don't know how analogous it is to the solenoid. 

Here's one way to get at this issue. Take two neodymium magnets, and, being careful
not to pinch your skin or chip the magnets, let them come together with a small
scrap of cardboard between them. The cardboard helps to avoid chipping the magnets,
and also makes it easier to get them apart afterward. Use the compass to map the
external field of this double magnet. Does it look like the field of the bar magnet?
By separating them again, do you get one sink and one source?

What do you think would happen if you broke the bar magnet in half?

From these observations, what do you conclude about the sources and sinks
of the magnetic field?


\end{lab}
%--------------------------------- induction lab --------------------------------------

\begin{lab}{Induction}

\apparatus
\equipn{solenoid (Heath)}{1}\\
\equipn{oscilloscope (HP1222A in rm. 418)}{1}\\
\equipn{2-meter wire with banana plugs}{1}\\
\equip{neodymium magnets}\\
\equip{masking tape}

\begin{goals}

\item[] Observe electric fields induced by changing magnetic fields.

\item[] Build a generator.

\item[] Discover Lenz's law.
\end{goals}

\labintroduction\index{induction}\index{Faraday, Michael}\index{generator}\index{Lenz's law}

Physicists hate complication, and when physicist Michael
Faraday was first learning physics in the early 19th
century, an embarrassingly complex aspect of the science was
the multiplicity of types of forces. Friction, normal
forces, gravity, electric forces, magnetic forces, surface
tension --- the list went on and on. Today, 200 years later,
ask a physicist to enumerate the fundamental forces of
nature and the most likely response will be ``four: gravity,
electromagnetism, the strong nuclear force and the weak
nuclear force.'' Part of the simplification came from the
study of matter at the atomic level, which showed that
apparently unrelated forces such as friction, normal forces,
and surface tension were all manifestations of electrical
forces among atoms. The other big simplification came from
Faraday's experimental work showing that electric and
magnetic forces were intimately related in previously
unexpected ways, so intimately related in fact that we now
refer to the two sets of force-phenomena under a single
term, ``electromagnetism.''

Even before Faraday, Oersted had shown that there was at
least some relationship between electric and magnetic
forces. An electrical current creates a magnetic field, and
magnetic fields exert forces on an electrical current. In
other words, electric forces are forces of charges acting on
charges, and magnetic forces are forces of moving charges on
moving charges. (Even the magnetic field of a bar magnet is
due to currents, the currents created by the orbiting
electrons in its atoms.)

Faraday took Oersted's work a step further, and showed that
the relationship between electricity and magnetism was even
deeper. He showed that a changing electric field produces a
magnetic field, and a changing magnetic field produces an
electric field. Faraday's work forms the basis for such
technologies as the transformer, the electric guitar, the
amplifier, and generator, and the electric motor.

\labsection{Qualitative Observations}

In this lab you will use a permanent magnet to produce
changing magnetic fields. This causes an electric field to
be induced, which you will detect using a solenoid (spool of
wire) connected to an oscilloscope. The electric field
drives electrons around the solenoid, producing a current
which is detected by the oscilloscope.

\labpart{ A changing magnetic field}

Do you detect any signal when you move the magnet or wiggle
it inside the solenoid or near it? What happens if you
change the speed at which you move the magnet?

\labpart{ A constant magnetic field}

Do you detect any signal on the oscilloscope when the magnet
is simply placed at rest inside the solenoid? Try the most
sensitive voltage scale.

\labpart{ Moving the solenoid}

What happens if you hold the magnet still and move the solenoid? 

\labpart{ A generator}

Tape the magnet securely to the eraser end of a pencil so
that its flat face (one of its two poles) is like the head
of a hammer.
Spin the pencil near the
solenoid and observe the induced signal. You have built a
generator. (I have unfortunately not had any luck lighting a
lightbulb with the setup, due to the relatively high
internal resistance of the solenoid.)

\labfig{lab-spin-magnet}

\labsection{Trying Out Your Understanding}

\labpart{ Changing the speed of the generator}

If you change the speed at which you spin the pencil, you
will of course cause the induced signal to oscillate more rapidly
(less time for each oscillation).
Does it also have any effect on the
\emph{strength} of the effect?

\labpart{Dependence on distance}

How does the signal picked up by your generator change with distance?

Try to explain what you have observed, and discuss your
interpretations with your instructor.

\labpart{ A solenoid with fewer loops}

Use the two-meter cable to make a second solenoid with the
same diameter but fewer loops. Compare the strength of the
induced signals.

\end{lab}
%--------------------------------- light waves lab --------------------------------------

\begin{lab}{Light Waves}\label{lab:light-waves}

\apparatus
\equip{helium-neon laser}{1/group}\\
\equipn{optical bench with posts \& holders   }{1/group}\\
\equipn{double slits, 0.05 cm (Klinger)}{1/group}\\
\equip{rulers}\\
\equip{meter sticks}\\
\equip{tape measures}

\begin{goals}

\item[] Observe evidence for the wave nature of light.

\item[] Determine the wavelength of red light (specifically, the
color emitted by the laser), by measuring a double-slit
diffraction pattern.

\end{goals}


Isaac Newton's epitaph, written by Alexander Pope, reads:

   Nature and Nature's laws lay hid in night.

   God said let Newton be, and all was light.

Notwithstanding Newton's stature as the greatest physical
scientist who ever lived, it's a little ironic that Pope
chose light as a metaphor, because it was in the study of
light that Newton made some of his worst mistakes. Newton
was a firm believer in the dogma, then unsupported by
observation, that matter was composed of atoms, and it
seemed logical to him that light as well should be composed
of tiny particles, or ``corpuscles.'' His opinions on the
subject were so strong that he influenced generations of his
successors to discount the arguments of Huygens and Grimaldi
for the wave nature of light. It was not until 150 years
later that Thomas Young demonstrated conclusively that light was a wave.

\labfig[b]{doubleslit}

In this lab, you'll do an experiment similar to Young's, but with
modern equipment to make things easier. To understand how it works,
let's consider an analogy with water waves. Figure \figref{doubleslit} shows
what happens when water waves encounter a barrier with two gaps in it:
beyond the barrier, there are two overlapping sets of ripples,
which form a fan pattern. (This is a real photo, but doctored slightly
in order to make the fan pattern easier to see --- in reality, most of the
wave energy is wasted when the wave pattern hits the barrier, and the amount
of energy that gets through the slit is relatively small.)
Along the center line, the crests of the ripples from the left-hand
hole coincide with the crests of those coming from the one on the right,
making double-height waves. Similarly, the troughs coincide with the troughs,
making troughs of double depth. A bug, standing on the surface of the water 
along this center line, would experience strong up and down motion. The reason
that crests reinforce crests and troughs reinforce troughs along this line is that
at any given point along the line, the waves coming from the two holes have had
to travel an equal number of wavelengths. At the point marked A,
for example, we have a double-height wave crest formed by coinciding waves that have
each had to travel 15 wavelengths from the holes where they originated. It's as though
two soldiers set out marching from the holes, both heading for this spot. Each one has performed
the cycle of ``left-right'' fifteen times, so they're on the same foot when they meet up.

This explains why there is strong wave motion along the center line, but what about
the lines of strong wave motion coming out at other angles? These are the ones
where the soldiers have taken a different number of steps, but are nevertheless on
the same foot. For instance, point C 
lies 16 wavelengths from the left-hand hole, and 15 wavelengths from the right one.

In between the wedge-shaped regions of strong wave motion, we have lines along which the
picture is a uniform gray. There is no up-and-down vibration at a point along these lines.
For instance, point B
lies 15.5 wavelengths from the left-hand hole, and 14 wavelengths from the right one.
Here, the soldiers meet up, and one is on his left foot while the other is on his right.
The crest of one wave coincides with the trough of the other, and they cancel out.

Now let's think about how this would work with light waves. It's obvious that the water
waves are waves, because you can just look at them and see the crests and troughs. As time
goes on, you see the crests and troughs travel across the water. With light waves, however,
this kind of direct observation won't work, and that's why it was possible for Newton
and his successors to be misled about the true wave nature of light. Not only is the
wavelength of a light wave microscopic in scale, but the waves travel through space
at hundreds of thousands of kilometers per second. Not only that, but we don't normally
see light traveling through the air unless there's something in the air to reflect some
of the light. For instance, you can see a car's headlight beams in a fog, but when the
air is clear, all you see is the spots where they hit the road, because the road is the
only thing that can reflect light back to your eyes.

That's why the overlapping-ripples type of experiment is useful here. With light waves, one
can for example let the fan pattern hit a piece of paper; the paper's location would
correspond to the top edge of the picture of the water waves. A point, such as the
one at the center, that experiences strong wave motion will be steadily illuminated,
while the gaps in between will be dark. This will not only confirm the wave nature of
light, but it will also end up giving you a way to determine the wavelength of visible
(red) light.

\labobservations

Set up your laser on your optical bench.
Put the double slit in the beam, and observe the pattern
of dark and bright spots on the wall across the room. You
should see something like the pattern shown in the figure on page \pageref{fig:fringes}.

\labwidefig{fringes}\label{fig:fringes}

Measure the distance from the slits to the wall, and measure the spacing
of the light and dark pattern. Also, write down the center-to-center distance
between the slits, which is printed on them. 

\labsection{Analysis}

\labfig{radians}

In the photo of the water waves, the fan pattern consisted of wedges, each of
which made a certain angle. Your first job is to figure out what that angle
was in the laser experiment. For instance, if the spacing of the pattern on the
wall was 1/100 of the diameter of the imaginary circle, then the angle at the tip of
the long, skinny pie slice would be 1/100 of a full circle, or 3.60 degrees.
In general, you can set this up as a proportionality,
\begin{equation*}
\frac{\text{spacing of pattern}}{\text{circumference of circle}} = \frac{\text{angle}}{360\degunit} \qquad .
\end{equation*}
If you didn't remember that the circumference of a circle equals $2\pi$ times its radius, feel
free to flagellate yourself now.

\labfig{compareslits}\label{fig:compareslits}

Now we have three things that are related: (1) the angle you've just calculated; (2) the distance
between the slits, which you know; and (3) the wavelength of the light, which you want to find out.
As shown in the figure above, a smaller spacing between the slits actually causes a \emph{bigger}
angle. Since all the reasoning is purely geometric, the angle also can't change if we shrink or
enlarge the whole diagram uniformly. For example, if we doubled both the distance between the slits
and the wavelength, the result would be the same picture, just enlarged to twice the scale. Since
the angle decreases with the slit distance, and depends only on the ratio between the slit distance
and the wavelength, we must have a relationship of the form

\begin{equation*}
  \text{angle} = \text{constant}\times\frac{\text{wavelength}}{\text{slit distance}} \qquad ,
\end{equation*}

where the constant out in front only has to be determined once and for all from one example. If you
measure carefully with a ruler and protractor on
the second example in the diagram on page \pageref{fig:compareslits}, you'll find that the angle is $26\degunit$, the wavelength is 2 mm, and the slit distance
is 5 mm, so we find that the constant is about 60 ($26\approx 60\times 2/5$). To solve for the wavelength
of the light, we multiply both sides by the slit distance, and divide both sides by the constant, 60, giving

\begin{equation*}
  \text{wavelength} \approx \frac{\text{angle}\times\text{slit distance}}{60} \qquad .
\end{equation*}


You can now use this equation to determine the wavelength of the red light from the laser.

\end{lab}

%--------------------------------- electron waves lab --------------------------------------

\begin{lab}{Electron Waves}

\apparatus
\emph{(two setups available)}\\
\equip{cathode ray tube (Leybold 555 626)}\\
\equip{high-voltage power supply (new Leybold)}\\
\equip{100-k$\Omega$ resistor with banana-plug connectors}

\begin{goals}

\item[] Observe evidence for the wave nature of electrons.

\item[] Determine whether an increase in an electron's speed
lengths its wavelength, or shortens it.

\end{goals}

The most momentous discovery in physics during the last century was
that matter behaves as both a particle and a wave. Electrons are one
of the basic particles that matter is made of, and in this lab you'll
see evidence that they behave not just as particles, but also as waves.


Conceptually, the experiment is very similar to lab \ref{lab:light-waves}.
As shown in the figure on the following page, what you are working with is basically the same kind of
vacuum tube as the picture tube in your television. As in a
TV, electrons are accelerated through a voltage and shot in
a beam to the front (big end) of the tube, where they hit a
phosphorescent coating and produce a glow. You cannot see
the electron beam itself. There is a very thin carbon foil
(it looks like a tiny piece of soap bubble) near where the
neck joins the spherical part of the tube, and the electrons
must pass through the foil before crossing over to the
phosphorescent screen.

The purpose of the carbon foil is to act sort of like the double slit
in lab  \ref{lab:light-waves}. Because the wavelengths of the electrons
are so short, we need a slit spacing that is on the same order of
magnitude as the size of an atom. In this lab, the slits are the
gaps between the carbon atoms themselves! The atoms in a graphite
crystal are arranged in a complicated hexagonal pattern, and the foil
contains many tiny graphite crystals, each with its hexagonal lattice
oriented randomly in three dimensions.
The resulting pattern of light and dark is therefore not quite the same as the one
you got with a simple double slit, but it's conceptually similar.
You'll see a bright spot at the center of the tube, which corresponds
to the bright central spot you saw with the laser. Surrounding it, you'll
see two somewhat fainter rings; these correspond to the spots of light
on either side of the central spot made with the laser. The angles of these
two rings with respect to the central axis of the tube are related to
the spacings between the atoms labeled $d_1$ and $d_2$ in the figure.

\labfigcaption{graphite}{The carbon atoms in the graphite crystal are
arranged hexagonally. The distances between the atoms can be measured
in units of nanometers (nm), one nanometer being a billionth of a meter.}

\labfigcaption{electrondiffraction}{The vacuum tube.}\label{fig:electrondiffraction}

\end{lab}

%========================================== self-check answers ===============================================

\startscanswers{ch:em}
\scanshdr{foo}{The second river is shallower, but is flowing more
rapidly. Although there is a smaller amount of water in the second picture, it will
take less time to flow ``off stage,'' so the ratio of water divided by time is the
same in the two pictures. Similarly, $\frac{1}{2}$ and $\frac{2}{4}$ represent the same number.}
\scanshdr{alternator} Unless the engine is already turning over, the permanent magnet
isn't spinning, so there is no change in the magnetic field. Only a changing magnetic
field creates an induced electric field.

%========================================== toc decoration ===============================================

\addtocontents{toc}{\protect\figureintoc{sunspot}}
