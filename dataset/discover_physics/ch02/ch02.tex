\mychapter{The Ray Model of Light}\label{ch:raymodel}\index{light!ray model}

\mysection[0]{Rays Don't Rust}

If you look at the winter night sky on a clear, moonless night far from any city
lights, something strange will soon catch your eye. Near the constellation of Andromeda\index{Andromeda galaxy}
is a little white smudge. What is it? You can easily convince yourself that it's
not a cloud, because it moves along with the stars as they rise and set. What you're
seeing is the Andromeda galaxy, a fantastically distant group of stars very similar
to our own Milky Way.%
\footnote{If you're in the
southern hemisphere, you have a more scenic sky than we in the north do, but
unfortunately you can't see any naked-eye objects that are as distant as the
Andromeda galaxy. You can enjoy the Magellanic Clouds and the Omega Centauri
cluster, but they're an order of magnitude closer.}
We can see individual stars within the Milky Way galaxy because
we're inside it, but the Andromeda galaxy looks like a fuzzy patch because we can't
make out its individual stars.
The vast distance to the Andromeda galaxy is hard to fathom, and it won't help
you to imagine it if I tell you the number of kilometers is 2 followed by 
19 zeroes.
Think of it like this: if the stars in our own galaxy were as close together
as the hairs on your skin, the Andromeda galaxy would be thousands of kilometers
away.

\margup{-75mm}{\fig{andromedamap}{How to locate the Andromeda galaxy.}}
The light had a long journey to get to your eyeball! A well-maintained car might
survive long enough to accumulate a million kilometers on its odometer, but
by that time it would be a rickety old rust-bucket, and the distance it had covered would
still only amount to a fraction of a billionth of a billionth of the distance
we're talking about. Light doesn't rust. A car's tracks can't
go on forever, but the trail of a light beam can. We call this
trail a ``ray.''

\mysection[0]{Time-Reversal Symmetry}\index{symmetry!time-reversal}

The neverending motion of a light ray is surprising compared
with the behavior of everyday objects, but in a way it makes sense. A car is
a complex system with hundreds of moving parts. Those parts can break,
or wear down due to friction. Each part is itself made of atoms, which can
do chemical reactions such as rusting. 
Light, however, is fundamental: as far as we know, it isn't
made of anything else. 
My wife's car has a dent in it that preserves the record of the time she
got rear-ended last year. As time goes on, a car accumulates more and more history.
Not so with a light ray. Since a light ray carries no history, there is no way
to distinguish its past from its future. Similarly, some brain-injured people
are unable to form long-term memories. To you and me, yesterday is different
from tomorrow because we can't remember tomorrow, but to them there is no such
distinction.

Experiments --- including some of the experiments you're going to do in this
course --- show that the laws of physics governing light rays are perfectly
symmetric with respect to past and future. If a light ray can go from A to
B, then it's also possible for a ray to go from B to A. I remember as a child
thinking that if I covered my eyes, my mommy couldn't see me. I was almost
right: if I couldn't see her eyes, she couldn't see mine. 

\begin{eg}{Why light rays don't stop}\label{eg:tiredlight}
Once the experimental evidence convinces us of time-reversal symmetry, it's
easy to prove that light rays never get tired and stop moving. Suppose
some light was headed our way from the Andromeda galaxy, but it
stopped somewhere along the way and never went any farther. Its trail,
which we call the ``ray,'' would be a straight line ending at that
point in empty space. Now suppose we send a film crew along in a space ship to document
the voyage, and we ask them to play back the video for us, but backwards.
Time is reversed. The narration is backwards. Clocks on the wall go counterclockwise.
In the reversed documentary, how does the light ray behave? At the beginning
(which is really the end), the light ray doesn't exist. Then, at some random moment
in time, the ray springs into existence, and starts heading back towards the
Andromeda galaxy. In this backwards version of the documentary, the light ray
is not behaving the way light rays are supposed to. Light doesn't just appear
out of nowhere in the middle of empty space for no reason. (If it did, it would
violate rotational symmetry, because there would be no physical reason why this
out-of-nowhere light ray would be moving in one direction rather than another.)
Since the backwards video is impossible, and all our accumulated data have shown that
light's behavior has time-reversal symmetry, we conclude that the forward video is
also impossible. Thus, it is not possible for a light ray to stop in the middle of empty
space.
\end{eg}

\marg{\fig{apollomirror}{The mirror left on the moon by the Apollo 11 astronauts.}}
\begin{eg}{The Apollo lunar ranging experiment}\label{eg:lunarranging}
In 1969, the Apollo 11 astronauts made the first crewed landing on the moon, and
while they were there they placed a mirror on the lunar surface. Astronomers
on earth then directed a laser beam at the landing site. The beam was reflected
by the mirror, and retraced its own path back to the earth, allowing the distance
to the moon to be measured extremely accurately (which turns out to provide important
information about the earth-moon system). Based on time-reversal symmetry,
we know that if the reflection is a 180-degree
turn, the reflected ray will behave in the same way as the outgoing one, and retrace
the same path. (Figure \figref{cornerreflector} on page
\pageref{fig:cornerreflector} explains the clever trick used to make sure the reflection
would be a 180-degree turn, without having to align the mirror perfectly.)
\end{eg}

\pagebreak[4]

\begin{eg}{Looking the wrong way through your glasses}
If you take off your glasses, turn them around, and look through them the other
way, they still work. This is essentially a demonstration of time-reversal
symmetry, although an imperfect one. It's imperfect because you're not
time-reversing the entire path of the rays. Instead of passing first through
the front surface of the lenses, then through the back surface, and then
through the surface of your eye, the rays are now going through the three
surfaces in a different order. For this reason, you'll notice that things look
a little distorted with your glasses reversed. To make a perfect example of
time-reversal, you'd have to have a little lamp inside your eyeball!
\end{eg}

If light never gets tired, why is it that I usually can't see the mountains from
my home in Southern California? They're far away, but if light never stops, why
should that matter? It's not that light just naturally stops after traveling
a certain distance, because I can easily see the sun, moon, and stars from my house,
and they're much farther away than the mountains. The difference is that my line
of sight to the mountains cuts through many miles of pollution and natural haze.
The time-reversal argument in example \ref{eg:tiredlight} depended on the assumption that the
light ray was traveling through empty space. If a light ray starts toward me from the
mountains, but hits a particle of
soot in the air, then the time-reversed story is perfectly reasonable: a particle
of soot emitted a ray of light, which hit the mountains.

\dqheader
\begin{dq}
If you watch a time-reversed soccer game, are the players still obeying the rules?
\end{dq}

\pagebreak[4]

\mysection[0]{Applications}
\begin{envsubsection}{The inverse-square law}\index{inverse-square law}
Yet another objection is that a distant candle appears dim. Why is this, if not
because the light is getting tired on the way to us? Likewise, our sun is just
a star like any other star, but it appears much brighter because it's so much
closer to us. Why are the other stars so dim if not because their light wears out?
It's not that the light rays are stopping, it's that they're
getting spread out more thinly. The light comes out of the source in all directions, and if you're
very far away, only a tiny percentage of the light will go into your eye. (If all the
light from a star went into your eye, you'd be in trouble.)


\widefigsidecaption[c]{inversesquare}{The light is four times dimmer at twice the distance.}

Figure \figref{inversesquare} shows what happens if you double your distance from the source. The
light from the flame spreads out in all directions. We pick four
representative rays from among those
that happen to pass through the nearer square. Of these four,
only one passes through the square of equal area at twice the distance. If the
two equal-area squares were people's eyes, then only one fourth of the light would
go into the more distant person's eye.


Another way
of thinking about it is that the light that passed through the first square spreads
out and makes a bigger square; at double the distance, the square is twice as wide and
twice as tall, so its area is $2\times2=4$ times greater. The same light has been spread
out over four times the area.

In general,
the rule works like this:
\begin{align*}
	\text{distance}\times2 &\Rightarrow \text{brightness}\times\frac{1}{4}\\
	\text{distance}\times3 &\Rightarrow \text{brightness}\times\frac{1}{9}\\
	\text{distance}\times4 &\Rightarrow \text{brightness}\times\frac{1}{16}
\end{align*}
To get the 4, we multiplied 2 by itself, 9 came from multiplying 3 by itself, and so on.
Multiplying a number by itself is called squaring it, and dividing one by a number
is called inverting it, so a relationship like this is known as an inverse square law.
Inverse square laws are very common in physics: they occur whenever something is spreading
out in all directions from a point.

\selfcheck{candlefivetimes}{%
Alice is one meter from the candle, while Bob is at a distance of five meters.
How many times dimmer is the light at Bob's location?%
}

\begin{eg}{An example with sound}
\egquestion Four castaways are adrift in an open boat, and are yelling to try to attract the
attention of passing ships. If all four of them yell at once, how much is their range
increased compared to the range they would have if they took turns yelling one at a time?\\
\eganswer This is an example involving sound. Although sound isn't the same as light,
it does spread out in all directions from a source, so it obeys the inverse-square
law. In the previous examples, we knew the distance and wanted to find the intensity
(brightness). Here, we know about the intensity (loudness), and we want to find out
about the distance. Rather than taking a number and multiplying it by itself to find the
answer, we need to reverse the process, and find the number that, when multiplied by
itself, gives four. In other words, we're computing the square root of four, which is
two. They will double their range, not quadruple it. 
\end{eg}

\begin{eg}{Astronomical distance scales}\label{eg:alphacdistance}
The nearest star, Alpha Centauri,\footnote{Sticklers will note that the nearest star
is really our own sun, and the second nearest is the burned-out cinder
known as Proxima Centauri, which is Alpha Centauri's close companion.}
is about 10,000,000,000,000,000 times dimmer than our sun when viewed from
our planet. If we assume that Alpha Centauri's true brightness is roughly the same
as that of our own sun, then we can find the distance to Alpha Centauri by
taking the square root of this number. Alpha Centauri's distance from us
is equal to about 100,000,000 times our distance from the sun.
\end{eg}

\margup{-100mm}{\fig{diaphragm}{The same lens is shown with its diaphragm set to three different
apertures.}}
\begin{eg}{Pupils and camera diaphragms}
In bright sunlight, your pupils contract to admit less light. At night they dilate,
becoming bigger ``light buckets.''
Your perception of brightness depends not only on the true brightness of the source
and your distance from it, but also on how much area your pupils present to the light.
Cameras have a similar mechanism, which is easy to see if you detach the lens
and its housing from the body of the camera, as shown in the figure. Here, the diameter of
the largest aperture is about ten times greater than that of the smallest aperture.
Making a circle ten times greater in radius increases its area by a factor of 100, so
the light-gathering power of the camera becomes 100 times greater. (Many people expect
that the area would only be ten times greater, but if you start drawing copies of the
small circle inside the large circle, you'll see that ten are not nearly enough to
fill in the entire area of the larger circle. Both the width and the height of the
bigger circle are ten times greater, so its area is 100 times greater.)
\end{eg}
\end{envsubsection}
%
\begin{envsubsection}{Parallax}\index{parallax}
Example \ref{eg:alphacdistance} on page \pageref{eg:alphacdistance} showed how we
can use brightness to determine distance, but your eye-brain system has a different method.
Right now, you can tell how far away this page is from your eyes. This sense of
depth perception comes from the fact that your two eyes show you the same scene
from two different perspectives. If you wink one eye and then the other, the page will
appear to shift back and forth a little.

\widefig{parallax}{At double the distance, the parallax angle is approximately halved.}
If you were looking at a fly on the bridge
of your nose, there would be an angle of nearly $180\degunit$ between the ray that
went into your left eye and the one that went into your right. Your brain would know
that this large angle implied a very small distance. This is called the parallax
angle. Objects at greater distances have smaller parallax angles, and when the angles
are small, it's a good approximation to say that the angle is inversely proportional
to the distance. In figure \figref{parallax}, the parallax angle is almost exactly
cut in half when the person moves twice as far away.

Parallax can be observed in other ways than with a pair of eyeballs.
As a child, you noticed that when you walked around on a moonlit evening,
the moon seemed to follow you. The moon wasn't really following you, and this isn't
even a special property of the moon. It's just that as you walk, you expect to observe
a parallax angle between the same scene viewed from different positions of your whole head.
Very distant objects, including those on the Earth's surface, have parallax
angles too small to notice by walking back and forth. In general, rays coming from
a very distant object are nearly parallel.

If your baseline is long enough, however, the small parallaxes of even very distant objects
may be detectable. In the nineteenth century, nobody knew how tall the Himalayas were,
or exactly where their peaks were on a map, and the Andes were generally believed to be the
tallest mountains in the world. The Himalayas had never been climbed, and could only be viewed
from a distance. From down on the plains of India, there was no way to tell whether they were
very tall mountains very far away, or relatively low ones that were much closer. British
surveyor George Everest finally established their true distance, and astounding height,
by observing the same peaks through a telescope from different locations far apart.

An even more spectacular feat of measurement was carried out by Hipparchus over\index{Hipparchus}\index{moon!distance to}
twenty-one centuries ago. By measuring the parallax of the moon as observed from
Alexandria and the Hellespont, he determined its distance to be about 90 times
the radius of the earth.\footnote{The reason this was
a hard measurement was that accurate clocks hadn't been invented, so there was no
easy way to synchronize the two observations, and the desired effect would be masked
by the apparent motion of the moon across the sky as it rose and set. Hipparchus's trick
was to do the measurement during a solar eclipse, so that people at both locations
would know they were in sync.}\label{hipparchusmoondistance}

The earth circles the sun, \figref{stellarparallax}, and we can therefore determine the distances to a few
hundred of the nearest stars by making observations six months apart, so that the
baseline for the parallax measurement is the diameter of the earth's orbit. For these
stars, the distances derived from parallax can be checked against the ones found by
the method of example \ref{eg:alphacdistance} on page \pageref{eg:alphacdistance}. They
do check out, which verifies the assumption that the stars
are objects analogous to our sun.

\widefigsidecaption{stellarparallax}{The nearer star has a larger parallax angle. By measuring
the parallax angles, we can determine the distances to both stars. (The scale on this
drawing is not realistic. If the earth's orbit was really this size, the nearest stars would
be several kilometers away.)}

\end{envsubsection}

\mysection{The Speed of Light}\index{light!speed of}
How fast does light travel? Does it even take any time to go from one place to another?
If so, is the speed different for light with different colors, or for light with
different brightnesses? Can a particular ray of light speed up or slow down?

\begin{envsubsection}{The principle of inertia}
We can answer the last question based on fundamental principles.
All the experimental evidence supports time-reversal symmetry for light rays.
Suppose that a beam of light traveling through a vacuum slowed down. After all,
a rolling soccer ball starts to slow down immediately after you kick it. Even
a rifle bullet slows down between the muzzle and the target.
Why shouldn't light slow down gradually? It can't slow down, because of
time-reversal symmetry. If the laws of physics said that
a ray of light slowed down while traveling through
a vacuum, then the time-reversed motion of the ray would violate the laws of
physics. In the time-reversed version, the ray is moving the opposite direction
and speeding up. Since all the experimental evidence shows that time-reversal
symmetry is valid for light rays, we conclude that a ray will never speed up
or slow down while traveling through a vacuum.

\margup{-60mm}{\fig{soccerball}{The soccer ball will never slow down.}
\spacebetweenfigs
\fig{galileo}{Galileo Galilei (1564-1642)}}\index{Galileo!inertia}
Why, then, do the ball and the bullet slow down? They wouldn't slow
down at all if they were traveling through interstellar space. It's only due to friction
that they lose speed. The ball slows down because of friction with the grass, and air
friction is what decelerates the bullet. The laws of physics are not complicated, and
in many ways they're not even different for light rays than for material objects. The laws of
physics are simple and consistent. We can now state the following important
principle, first proposed by Florentine physicist Galileo Galilei:\label{weak-principle-of-inertia}

\begin{important}[The principle of inertia]\index{inertia!weak principle of}
A ray of light or a material object continues moving in the same direction and
at the same speed if it is not interacting with anything else.
\end{important}
\end{envsubsection}
%
\begin{envsubsection}{Measuring the speed of light}
Observations also show that in a vacuum, all light moves at the same speed,
regardless of its color, its brightness, or the manner in which it was emitted.
The best evidence comes from supernovae, which are exploding stars. Supernovae
are so bright that we can see them even when they occur in distant galaxies whose
normal stars are too dim to resolve individually. When we observe a supernova,
all the light gets to us at the same time, so it must all have traveled at the
same speed.

Galileo made the first serious attempt to measure the speed of light. In his\index{Galileo!speed of light}
experiment, two people with lanterns stood a mile apart. The first person
opened the shutter of his lantern, and the second person opened the shutter
on his as soon as he saw the light from the first person's. A third observer
stood at an equal distance from both of them, and tried to measure the time
lag between the two. No such time lag was observed, so you could say that the
experiment failed, but in science a failure can still be important. This is
known as a negative experiment. Galileo's results showed that the speed of
light must be at least ten times the speed of sound. It was important that he
published his negative result, both because it convinced people that the problem
was scientifically interesting and because it told later workers that the
speed of light must be very fast, which would help them to design experiments
that might actually work.

\margup{-65mm}{\fig{io}{A modern image of Jupiter and its moon Io (right) from the
Voyager 1 probe.}
\spacebetweenfigs
\fig{roemer}{The earth is moving towards Jupiter and Io. Since the distance
is shrinking, it's taking less and less time for light to
get to us from Io. Io appears to circle Jupiter more quickly than normal. Six
months later, the earth will be on the opposite side of the sun, and receding
from Jupiter and Io, so Io will appear to go around more slowly.}}\index{Jupiter}\index{Io}\index{Roemer, Ole}
The first person to prove that light's speed was finite, and to determine it
numerically, was Ole Roemer, in a series of measurements around the year
1675. Roemer observed Io, one of Jupiter's moons, over a long period.
Since Io presumably took the same amount of time to complete each orbit of
Jupiter, it could be thought of as a very distant, very accurate clock.
A practical and accurate pendulum clock had recently been invented, so
Roemer could check whether the ratio of the two clocks' cycles, about
42.5 hours to one orbit, stayed exactly constant or changed a little. If the
process of seeing the distant moon was instantaneous, there would be no reason
for the two to get out of step. Even if the speed of light was finite, you might
expect that the result would be only to offset one cycle relative to the other.
The earth does not, however, stay at a constant distance from Jupiter and its
moons. Since the distance is changing gradually due to the two planets' orbital
motions, a finite speed of light would make the ``Io clock'' appear to run
faster as the planets drew near each other, and more slowly as their
separation increased. Roemer did find a variation in the apparent speed of
Io's orbits, which caused Io's eclipses by Jupiter (the moments when Io passed
in front of or behind Jupiter) to occur about 7 minutes early when the
earth was closest to Jupiter, and 7 minutes late when it was farthest. Based on
these measurements, Roemer estimated the speed of light to be approximately
200,000 kilometers per second, which is in the right ballpark compared to modern measurements
of 300,000 km/s.
\end{envsubsection}

\dqheader
\begin{dq}
When phenomena like X-rays and cosmic rays were first discovered, nobody knew what they
were. Suggest one way of testing the hypothesis that they were forms of light.\index{x-rays}\index{cosmic rays}
\end{dq}

\mysection[4]{Reflection}
\begin{envsubsection}{Seeing by reflection}\index{light!reflection}\index{reflection}
So far we've only talked about how you see things that emit light: stars, candles,
and so on. If you're reading this book on a computer screen, that's how you're seeing
it right now. But what if you're reading this book on paper? 
The paper doesn't emit light, and it would be invisible if
you turned out the lights in the room. The light from
the lamp is hitting the paper and being reflected to your eyes.

\margup{-47mm}{\fig{selfportraits}{Two self-portraits of the author, one taken in a mirror
and one with a piece of aluminum foil.}
\spacebetweenfigs
\fig{specularzero}{The incident and reflected rays are both perpendicular to the surface.}
\spacebetweenfigs
\fig{specularwrong}{This doesn't happen.}}
Most people only
think of reflection as something that happens with mirrors or other shiny, smooth
surfaces, but it happens with all surfaces. Consider figure \figref{selfportraits}.
The aluminum foil isn't as smooth as the mirror, so my reflection is blurry and jumbled.
If I hadn't told you, you probably wouldn't have known that it was a reflection of a
person at all. If the paper you're reading from was as smooth as a mirror, you would
see a reflection of the room in it, and the brightest object in the reflection would
probably be the lamp that's lighting the room. Paper, however, is not that smooth. It's
made of wood pulp. The reflection of the room is so blurry and jumbled that it all looks
like one big, washed-out, white blur. That white blur is what you see when you see the
paper. This is called diffuse reflection. In diffuse reflection, the reflected rays come
back out at random angles.\index{reflection!diffuse}
\end{envsubsection}
%
\begin{envsubsection}{Specular reflection}\index{reflection!specular}
Reflection from a smooth surface is called specular reflection, from the Latin word
for mirror. (The root, a verb meaning ``to look at,'' is the same as the root of ``spectacular''
and ``spectacle.'') When a light ray is reflected, we get a new ray at some new angle, which
depends on the angle at which the incident (original) ray came in. What's the rule that determines
the direction of the reflected ray? We can determine the answer by symmetry.

First, if
the incident ray comes in perpendicular to the surface, \figref{specularzero},
then there is perfect left-right
reflection symmetry. (It's just a coincidence that we have \emph{reflection} symmetry
occurring in our analysis of \emph{reflection}.) If the reflected ray came back at some
angle to the left or right, it would violate this symmetry. Therefore the reflected
ray must be right on top of the incident ray, straight back up.
Because this is the simplest possible
specular reflection, we define these angles as zero: all rays have their angles
measured with respect to perpendicular, not with respect to the surface itself.
Typically the rays themselves will not be perpendicular to the surface, but we still
measure their angles with respect to an imaginary line perpendicular to the surface,
which we call the normal. (``Normal'' is simply another word for perpendicular.)\index{normal}

Now what if the incident angle isn't zero? Figure \figref{specularwrong} shows
what doesn't happen. It's not possible for the reflected angle $r$ to be unequal to
the incident angle $i$, because of symmetry. Suppose we lived in a goofy universe, where
the laws of physics gave the result shown in the figure: $r$ is always less than
$i$. What would happen if we did a time-reversal on the diagram? Oops --- then
we'd have $r$ greater than $i\,$! Since experiments support time-reversal symmetry
for light rays, we conclude that this is impossible.\footnote{There are a couple
of oversimplifications in this argument, which shows how debased a physicist's conception
of mathematical proof can be. First, we could imagine a rule like $r=90\degunit-i$,
which would satisfy time-reversal symmetry, since $i=90\degunit-r$; however, such a
rule would not give $r=0$ when $i=0$, which we require based on reflection symmetry.
Another grotesque possibility is $r=i$, but with the reflected ray on the same side
of the normal as the incident ray. This satisfies both time-reversal symmetry and
reflection symmetry, but experiments show that it isn't what really happens in our
universe. It can also be ruled out based on another type of symmetry which we haven't
discussed yet (section \ref{sec:strong-inertia}).} The actual laws of physics
give equal angles of incidence and reflection,
\begin{equation*}
	r = i \qquad .
\end{equation*}

\margup{-25mm}{\fig{specularright}{This does happen.}
\spacebetweenfigs
\fig{poolball}{example \ref{eg:reflectpoolball}}
\spacebetweenfigs
\fig{cornerreflector}{A corner reflector}%
\spacebetweenfigs
\fig{flatmirrorimage}{example \ref{eg:flatmirrorimage}}
}
\begin{eg}{Reflecting a pool ball}\label{eg:reflectpoolball}
The proof of $r=i$ for light rays works equally well for pool balls, provided that
the effects that violate symmetry are small. For instance, we assume that the ball
doesn't have lots of spin put on it, because that would break the left-right
reflection symmetry.
\end{eg}
\selfcheck{cornerreflector}{%
Continue the ray in figure \figref{cornerreflector} through its second reflection. In what
direction is the returning ray? How does this relate to example
\ref{eg:lunarranging} on page \pageref{eg:lunarranging}?%
}
\begin{eg}{An image}\label{eg:flatmirrorimage}
Figure \figref{flatmirrorimage} shows some representative rays spreading out 
from one point on the flame. These rays strike the mirror and are reflected. To the observer on
the left, the reflected rays are indistinguishable from the ones that would have originated from
an actual flame on the far side of the mirror. Rays don't carry any history, so there is no
way for the eye to know that the rays underwent reflection along the way. (The rays shown in the diagram
form an image of one point on the flame, but every other point on the flame sends out a similar
bundle of rays, and has its own image formed.)
\end{eg}
\selfcheck{diffuseimage}{What happens in figure 
\figref{flatmirrorimage} if you replace the flame with an object that doesn't
emit light, and can only be seen by diffuse reflection?
}
\end{envsubsection}

\dqheader
\begin{dq}
Laser beams are made of light. In science fiction movies, laser beams are often shown
as bright lines shooting out of a laser gun on a spaceship. Why is this scientifically
incorrect?
\end{dq}

%===============================================================================
%===============================================================================

\begin{hwsection}

\begin{hw}{alien-pool}
The natives of planet Wumpus play pool using light rays on an eleven-sided
table with mirrors for bumpers. Trace this shot accurately with a ruler and protractor
to reveal the hidden message.
\end{hw}

\widefig{alienpool}{Problem \ref{hw:alien-pool}.}

\begin{hw}{where-image-is-visible}
Sketch a copy of figure \figref{flatmirrorimage} on page \pageref{fig:flatmirrorimage}.
There are some places from which the image is visible, and some
from which it isn't. Show these regions on your sketch by outlining their borders and
filling them with two different kinds of shading.
\end{hw}

\margup{-126mm}{\fig{astronautshadows}{Problem \ref{hw:diffuse-shadows}.}
\spacebetweenfigs
\fig{glossy}{Problem \ref{hw:glossy}a.}
\spacebetweenfigs
\fig{onion}{Problem \ref{hw:glossy}b.}
}
\begin{hw}{diffuse-shadows}
(a) Draw a ray diagram showing why a small light source (a
candle, say) produces sharper shadows than a large one (e.g.
a long fluorescent bulb). Draw a cross-section --- don't try to draw in three dimensions.
Your diagram needs to show rays spreading in many directions from each point on the
light source, and you need to track the rays until they hit the surface on which
the shadow is being cast. \hwendpart
(b) Astronaut Mary goes to Mercury,
while Gary visits Jupiter's moon Ganymede. Unfortunately it's hard to
tell whose vacation pictures are whose, because everybody looks the same
in a space suit. Which picture is which? (Note that the brightness of the light
is irrelevant. As you can see, the pictures look equally bright, because they
took longer or shorter exposures to compensate for the amount of sunlight.)
\end{hw}

\begin{hw}{glossy}
(a) The first figure shows a surface that is mostly smooth, but has a few irregularities in it.
Use a ray diagram to show how reflection from this surface would work.\hwendpart
(b) The second figure shows an onion on an old chair. What evidence do you see in this
picture that there are surfaces like the one in part a?
\end{hw}

\begin{hw}{nostellarparallax}
Many astronomers made attempts to detect the parallax
of the stars before anyone finally measured their very
small parallax angles. The early results were used as an argument
against models of the universe in which the earth
orbited the sun. Were all these efforts a waste? Should we criticize the
astronomers who made them for producing incorrect results? How does this
resemble the story of Galileo's attempt to measure the speed of light?
Galileo's result could be stated as a lower limit on the speed of light, i.e.,
a mathematical inequality rather than an equality; could you do something
similar with the early parallax measurements?
\end{hw}

\begin{hw}{see-below-waist}
If a mirror on a wall is only big enough for you to see
yourself from your head down to your waist, can you see your
entire body by backing up?  Test this experimentally and
come up with an explanation for your observations using ray diagrams. Note that
it's easy to confuse yourself if the mirror is even a tiny
bit off of vertical; check whether you are able to see more
of yourself both above \emph{and} below. (To make this test work,
you may need to lower the mirror so that you can't see the top of
your head, or put a piece of tape on the mirror, and pretend that's
the top of it.)
\end{hw}


\margup{-40mm}{
\fig{see-below-waist}{Problem \ref{hw:see-below-waist}}
\spacebetweenfigs
\fig{moonphases}{Problem \ref{hw:moonphases}.}
}
\begin{hw}{moonphases}
The diagram shows the moon orbiting the earth (not to scale) with
sunlight coming in from the right.\\
(a) Why are the sun's rays shown coming in parallel? Explain.\hwendpart
(b) Figure out the phase of the moon when the moon is at each point
in its orbit. In other words, when is it a new moon, when is it 
a crescent, when is it a half moon, when is it gibbous, and
when is it full?
\end{hw}

\begin{hw}{campfire}
(a) You're photographing some people around a campfire. If you step back
three times farther from the fire to frame the shot differently, 
how many times longer will the exposure have to be? Explain.\hwendpart
(b) You're worried that with the longer exposure, the dancing flames will
look blurry. Rather than compensating for the greater distance with a longer
exposure, you decide to open the diaphragm of the camera wider. How many
times greater will the diameter of the aperture have to be? Explain.
\end{hw}

\begin{hw}{roemer}
Why did Roemer only need to know the radius of the earth's
orbit, not Jupiter's?
\end{hw}

\begin{hw}{lightversussound}
Suggest a simple experiment or observation, without any special equipment,
to show that light isn't a form of sound. (Note that there are invisible
forms of light such as ultraviolet and infrared, so the invisibility of
sound doesn't prove anything. Likewise, you can't conclude anything from
the inaudibility of light.)
\end{hw}

\hwnote{%
In problems \ref{hw:stealth-bomber} and \ref{hw:gps}, you need
to know that radio waves are fundamentally the same phenomenon as light,
and travel at the same speed.
}

\begin{hw}{stealth-bomber}
The Stealth bomber is designed with flat, smooth
surfaces. Why would this make it difficult to detect via radar?
Explain using a ray diagram.
\end{hw}

\begin{hw}{gps}
A Global Positioning System (GPS) receiver is a device
that lets you figure out where you are by receiving radio
signals from satellites. It's accurate to within a few
meters. The details are a little complicated,
but for our present purposes, let's imagine a simplified version
of the system in which the satellite sends a signal at a known
time, and your handheld unit receives it at a time that is also
very accurately measured. The time delay indicates how far you
are from the satellite. As a further simplification, let's assume
that everything is one-dimensional: the satellite is low on the
eastern horizon, and we're only interested in determining your
east-west position (longitude).\footnote{If you're curious,
here's a brief explanation of how the real system works, without
the oversimplifications. There are currently about 24 GPS satellites
in orbit, and to get your location, you need to get signals from four
of them simultaneously. The basic idea is that by knowing your distance
from three points in space, you can find your location in three dimensions.
Why, then, do you need to get four signals? The satellites all have atomic
clocks on board, but it's not practical to put an atomic
clock in your handheld unit. You can think of the fourth satellite as
a replacement for the atomic clock you wish you had in your receiver.}
How accurate does the measurement of the time delay
have to be, to determine your position to this accuracy of a few meters?
\end{hw}

\end{hwsection}

%========================================== labs ===============================================
%--------------------------------- reflection and time-reversal lab --------------------------------------

\begin{lab}{Time-Reversal and Reflection Sym\-met\-ry}

\apparatus
\equip{laser}\\
\equip{plastic box}\\
\equip{protractor}

\begin{goals}

\item[] Observe the phenomenon of refraction.

\item[] Test whether refraction obeys time-reversal and reflection symmetry.

\end{goals}

\labpart{Refraction}\index{refraction}\index{light!refraction}

Put water in the box, and shine the laser into it at an angle. You should be able
to see that there is a beam that is reflected back from the surface of the box --- although the beam
is invisible in air, you can see a dot where it hits things like your hand or the box.

So far you're just seeing things that you've already read about in the book. But now
look \emph{inside} the water. Part of the light is reflected, but part of it is
\emph{transmitted}, i.e., passes into the water rather than bouncing back. 
We now have three rays: incident, reflected, and transmitted, which form the
angles $i$, $r$, and $t$ with respect to the normal.
It's easiest if you keep everything in a horizontal plane, because angles in three dimensions are
hard to measure. You may want to put a piece of paper under the box to mark the rays.

\labfigcaption{lab-refraction-angles}{The angles of the three rays are measured with respect
to the normal.}

Note that the direction of the transmitted ray isn't the same as the direction of the
incident ray; it's been knocked off course. This bending phenomenon is called refraction.
(Think ``fracture,'' like a broken bone.) 

Two simplifications: (1) From now on I'll stop drawing all the reflected rays.
(2) Let's think of the plastic box as if it didn't exist. In other words, the
light is cruising through air when suddenly it hits some water. A justification for this is that
none of the observations you're going to make depend on the thickness of the plastic, so we could
get the same results even with a box that was infinitely thin, i.e., nonexistent. 

\labpart{Time-reversal symmetry}\index{symmetry!time-reversal!lab}

Try sending the beam through a corner as suggested by the figure. Make sure that the incident
angle of the incoming ray, marked with the dashed arc in the figure, is nice and big. If it's
less than about 60 degrees, you won't get a ray
emerging on the other side of the corner at all.\footnote{This is a phenomenon known as total
internal reflection.\index{total internal reflection}\index{reflection!total internal}
When a ray in a denser medium hits a boundary with a less dense medium,
it may be 100\% reflected, depending on the angles. You can think of it as happening when the
angle of the emerging ray with respect to the normal would have been greater than 90 degrees.
Total internal reflection is the basis for fiber optics, the technology used in modern long-distance
telephone lines.}

\labfig{lab-through-corner}

You can now test whether refraction
obeys time-reversal symmetry. Measure the angles with a protractor, and then redo the experiment with
the ray coming back toward the box along the original ray's exit line. Are your results time-reversal
symmetric, or not?

Incidentally, you may have been wondering why time-reversal symmetry seems to be violated in everyday life.
For instance, if you see a video of Humpty Dumpty assembling himself out of pieces and levitating back
to the top of the wall, you know the video has been reversed. Actually this isn't evidence that the laws
of physics are asymmetric; it's just that it would be extremely difficult to start all of Humpty Dumpty's
pieces moving in precisely the right direction at the the right speed so that he would reassemble himself.
Similarly, there are many reflected rays left out of the figure above. If every possible reflection and
refraction had been included, it would have looked like a pitchfork or a complicated bush. To time-reverse
the diagram exactly is difficult --- you'd have to arrange many different lasers so that their beams came
together perfectly and joined into one beam. Again, it's a practical issue, not an asymmetry in the
laws of physics.

\labpart{Reflection symmetry}\index{symmetry!reflection!lab}

Now we want to see if refraction obeys reflection symmetry.
That sounds confusing, doesn't it? 
The word ``reflection'' here refers to the type of symmetry (i.e., mirror symmetry), not to
the thing that's happening to the light ray.
In other words, suppose you
do a bunch of experiments and measurements involving refraction. Someone videotapes you, and then alters
the videotape so that left and right are reversed. If the laws of physics are
reflection-symmetric, then there is no way to tell that there's anything wrong with the video.

\labfig{lab-time-reversal}

Remember, this whole lab is about \emph{refraction}. That means you're looking at the ray that is
passing on into the water, not the ray that comes back out into the air.

One very simple test is to measure the angle $t$ of the transmitted ray in the case where the incident
angle $i$ is zero. In this situation, what value of $t$ is required by reflection symmetry? Try it.

Now try a few measurements of $i$ and $t$ where $i$ isn't zero, and then redo the measurents with $i$ on
the other side of the normal. Do the results support reflection symmetry?

\pagebreak[4]

\labsection{To Think About Before Lab}

Criticize the following statements:

``The angle of refraction equals the angle of incidence.''

``In part C, we found that there was symmetry, because in every case, the ray bounced back at the
same angle it came in at.''

\end{lab}

%--------------------------------- models of light lab --------------------------------------

\begin{lab}{Models of Light}

\apparatus
\equip{laser}\\
\equip{plastic box}\\
\equip{protractor}

\goal{Test a particle model and a wave model of light.}

\labintroduction\index{light!particle model}\index{light!wave model}

This chapter is called ``The Ray Model of Light,'' but the ray model is obviously a very simplified
one.
What is a light ray, really? We know it bounces off of mirrors, which is like a pool ball bouncing off of
a bumper. It might therefore be natural to guess that a beam of light really consists of a stream of
tiny \emph{particles}, just as the water coming out of a fire hose is really made out of atoms.
On the other hand, \emph{waves} can also bounce off of things --- that's what an echo is. Let's see if we can
figure out anything about this, while keeping in mind that the particle and wave explanations are
only \emph{models}.

\labfigcaption{lab-ramp}{1. A particle model of refraction. As the ball slows down, it turns to the right.}

It's not hard to construct a mechanical model of refraction using particles, as shown in figure 1. The
ball goes straight when it's in the first flat area, curves and decelerates as it goes up the ramp,
and then goes straight again when it's in the other flat area. Note that the ball has different
speeds in the two regions: fast on the right and slow on the left. One of these regions represents air, one water --- we
haven't yet established which is which.

\labfigcaption{lab-water}{2. Water waves are refracted at the boundary between regions having two different depths. As
the waves move toward the top of the page, they encounter the boundary, speed up, and turn to the right.}\label{lab-fig-water}

However, a wave model is also capable of explaining refraction, as in figure 2.
Water waves have different speeds in shallow and deep water. The waves in the figure come up from the bottom,
and encounter the diagonal boundary between the two regions. Note that the distance between one crest and
the next, called the wavelength,\index{wavelength}
changes when the wave speed changes. This is similar to the way that the spacing in a stream of traffic would get farther apart
when the road changed from dirt to pavement: the cars in the front are the first to speed up, so they pull away a little
before the cars following them speed up, too.

The waves hit the boundary at an angle. The only way the waves in the two regions can connect up with each
other is if the crests twist around. This is just like the change of direction we observe when light rays are refracted.

As with the particle model, the wave model involves two regions in which the speeds are different.
It's only a coincidence that the photo in figure 2 was created using water waves. One of the two regions does
represent the water you'll use in the lab, but the other region represents the air! The photo could have been
made using waves in some other medium, e.g., the two regions could have been two sheets of rubber. We can also
easily establish that light is not a mechanical vibration of matter. For instance, we know that sunlight gets
to us through the vacuum of outer space. 

\pagebreak[4]

\labsection{Models of Refraction}
Before we start worrying about which model is correct, let's just see what consequences each one has for
refraction. This part of the lab is just thinking, not observing. You're not taking any data yet.

In figure 1, the incident ``ray'' is on the near side of the normal, and the result is that
the ray makes a right turn. Suppose instead that the incident ray was on the far side of the normal. Which way
would it turn? Also, the incident ray could have come to the ramp from
the high side, and then moved down the ramp to the lower area. If you imagine dividing the diagram into four
quadrants, like a pizza cut into four slices, we have a total of four possibilities for the incident ray.

\labfig{lab-quadrants-idea}

Predict the results for all four possibilities, using the particle model:

\labfig{lab-particle-quadrants}

Can you come up with a simple rule that describes all four results?

Now do the same for the wave model. Remember that the crests will always be closer together in the region where the
wave's speed is lower.

If you have a hard time visualizing this, try making a model using four rulers. First lay down two
rulers to represent two of the parallel wave crests of the incoming wave. Although the rulers are
parallel, they form a parallelogram rather than a rectangle.
 Now lay down two more rulers to represent the wave crests on the
other side of the boundary that connect onto these. Swivel them in order to make the distance between
crests correct in relation to the distance between the two original crests.

\labfig{lab-wave-quadrants}

\labpart{Observing Refraction With the Laser}
Now observe the refraction of the laser beam as it passes into and out of the tub of water, and observe how
it bends when the incident ray is in each of the four possible quadrants. Can your observations be
interpreted successfully with the particle model? If so, does the particle model require that light go
faster in air, or in water? Similarly, see if you can interpret your results with the wave model.

\labpart{Reflection}
Now repeat part A, but observe the reflected ray instead of the refracted one. The main issue here
is simply whether reflection can occur at all in the different cases. The wave model allows both types
of reflection (back into the fast medium, and back into the slow medium). You should be able to figure
out which types of reflection exist in the particle model.

\labsection{Analysis}
You should now have data from a total of eight different setups: four with refraction, and
four with reflection.
Is one model more successful than the other in describing all these data? You need to compare all
eight observations with each model.

\end{lab}

%--------------------------------- speed of light in water lab --------------------------------------

\begin{lab}[0]{The Speed of Light in Matter}

\apparatus
\equip{laser}\\
\equip{plastic box}\\
\equip{protractor}

\goal{Measure the speed of light in a substance such as water, glass, or plastic.}

A picture like figure 2 on page \pageref{lab-fig-water} has two types of information on it. 
First, we can tell that the incident and transmitted angles are about $i=30\degunit$ and $t=60\degunit$.
We can also tell that the wave's speed in the upper-left region is about double what it is in the lower-right region, since
the wavelength (crest-to-crest distance) is about twice as long. However, we can only tell the ratio of
the two speeds, not the absolute speeds in units of meters per second.

There's speed information and angle information, and the two are related. If we knew either one, we could find
the other. For instance, if I gave you only the angle information, and asked you to make a diagram like
the figure, you'd be forced to draw the wavelength of the transmitted wave twice as long as that of the
incident wave.

Your goal in this lab is to use this technique to measure the speed of light in some substance.
Your answer will be a number: the ratio of light's speed in your substance to its speed in air.
All you have to do is measure a pair of angles $i$ and $t$, and then draw an accurate diagram.
Because of the inherent limitations of the technique, you can only find the speed of light in this
substance relative to the speed of light in air, not its absolute speed in units of
meters per second. For instance, you might find that speed of light in weasel sweat
is 0.71 times the speed of light in air.

It's up to you to decide what substance you want to use.
You can bring something from home if you like. If you're adventurous, one interesting possibility
is to measure the speed of light of a solution, like salt in water, and you change the concentration.
Another challenge would be to measure the speed of light in a vacuum --- we have a vacuum pump and
some vacuum flasks.

Make sure to use the largest possible angles with respect to the normal. When the angles are
small, you get a low-precision result. In the extreme case, measuring $i=0\degunit$ and
$t=0\degunit$ tells you absolutely nothing about the speed of light.
\end{lab}

%========================================== self-check answers ===============================================

\startscanswers{ch:raymodel}
\scanshdr{candlefivetimes}{He's five times farther away than she is, so the light he sees
is 1/25 the brightness.}
\scanshdr{cornerreflector}{After the second reflection, the ray is going back parallel to the
original incident ray. This is how the lunar ranging reflector in example
\ref{eg:lunarranging} on page \pageref{eg:lunarranging} worked, except in three dimensions
rather than two. Can you imagine how to make it work in three dimensions?}
\scanshdr{diffuseimage}{It wouldn't matter. The rough surface sends rays out in all directions,
which is no different from what happens with the flame.}

%========================================== toc decoration ===============================================

\addtocontents{toc}{\protect\figureintoc{breakfast}}
