\addtocontents{toc}{\protect\figureintocnoresize{pool}}
%\addtocontents{toc}{\protect\figureintoc{rhic}}
\mychapter{Relativity}\label{ch:relativity}

Complaining about the educational system is a national sport
among professors in the U.S., and I, like my colleagues, am
often tempted to imagine a golden age of education in our
country's past, or to compare our system unfavorably with
foreign ones. Reality intrudes, however, when my immigrant
students recount the overemphasis on rote memorization in
their native countries, and the philosophy that what the
teacher says is always right, even when it's wrong.

\marg{\fig{einstein}{Albert Einstein.}}\index{Einstein, Albert}
Albert Einstein's education in late-nineteenth-century
Germany was neither modern nor liberal. He did well in the
early grades,\footnote{The myth that he failed his elementary-school
classes comes from a misunderstanding based on a reversal of
the German numerical grading scale.} but in high school and
college he began to get in trouble for what today's edspeak
calls ``critical thinking.''\index{Einstein, Albert}

Indeed, there was much that deserved criticism in the state
of physics at that time. There was a subtle contradiction
between the theory of light as a wave and Galileo's
principle that all motion is relative. As a teenager, Einstein began
thinking about this on an intuitive basis,
trying to imagine what a light beam would look like if you
could ride along beside it on a motorcycle at the speed of
light. Today we remember him most of all for his radical and
far-reaching solution to this contradiction, his theory of
relativity, but in his student years his insights were
greeted with derision from his professors. One called him a
``lazy dog.'' Einstein's distaste for authority was typified
by his decision as a teenager to renounce his German
citizenship and become a stateless person, based purely on
his opposition to the militarism and repressiveness of
German society. He spent his most productive scientific
years in Switzerland and Berlin, first as a patent clerk but
later as a university professor. He was an outspoken
pacifist and a stubborn opponent of World War I, shielded
from retribution by his eventual acquisition of Swiss
citizenship.

As the epochal nature of his work became evident,
some liberal Germans began to point to him as a model of the
``new German,'' but after the Nazi coup d'etat, staged public
meetings began, at which Nazi scientists
criticized the work of this ethnically Jewish (but
spiritually nonconformist) giant of science. 
When Hitler was appointed chancellor, 
Einstein was on a stint as a visiting professor at Caltech,
and he never returned to the Nazi state. World War II convinced Einstein to soften his
strict pacifist stance, and he signed a secret letter to
President Roosevelt urging research into the building of a
nuclear bomb, a device that could not have been imagined
without his theory of relativity. He later wrote, however,
that when Hiroshima and Nagasaki were bombed, it made him
wish he could burn off his own fingers for having signed the
letter.

Einstein has become a kind of scientific Santa Claus figure
in popular culture, which is presumably why the public is always
so titillated by his well-documented career as a skirt-chaser
and unfaithful husband. Many are also surprised by his lifelong
commitment to socialism. A favorite target of J. Edgar Hoover's
paranoia, Einstein had his phone tapped, his garbage searched, and
his mail illegally opened. A censored version of his 1800-page FBI
file was obtained in 1983 under the Freedom of Information Act,
and a more complete version was disclosed recently.\footnote{Fred Jerome,
\emph{The Einstein File}, St. Martin's Press, 2002}. It includes comments solicited from
anti-Semitic and pro-Nazi informants, as well as statements, from sources
who turned out to be mental patients, that Einstein had invented a death ray
and a robot that could control the human mind. Even today, an FBI
web page\footnote{foia.fbi.gov/einstein.htm} accuses him of working for or belonging to 34 ``communist-front''
organizations, apparently including the American Crusade Against Lynching.
At the height of the McCarthy witch hunt, Einstein bravely denounced McCarthy,
and publicly urged its targets to refuse to testify before the House Unamerican
Activities Committee.
Belying his other-worldly and absent-minded image, his political positions
seem in retrospect not to have been at all clouded by naivete or the more fuzzy-minded variety of idealism.
He worked against racism in the U.S. long before the civil rights movement got under way.
In an era when many leftists were only too eager to
apologize for Stalinism, he opposed it consistently.

This chapter is specifically about Einstein's
theory of relativity, but Einstein also began a second,
parallel revolution in physics known as the quantum theory,
which stated, among other things, that certain processes in
nature are inescapably random. Ironically, Einstein was an
outspoken doubter of the new quantum ideas that were
built on his foundations, being convinced
that ``the Old One [God] does not play dice with the
universe,'' but quantum and relativistic concepts are now
thoroughly intertwined in physics.

\mysection{The Principle of Relativity}\index{aether}
By the time Einstein was born, Galileo's principle of
inertia had been accepted for two centuries.
The teenage Einstein was
suspicious because his professors said light waves obeyed an
entirely different set of rules than material objects, and in particular
that light did not obey the principle of inertia. They
believed that light waves were a vibration of a mysterious
substance called the aether, and that the speed of light should
be interpreted as a speed relative to this aether.  Thus although the
cornerstone of the study of matter had for two centuries
been the idea that motion is relative, the science of light
seemed to contain a concept that a certain frame of reference
was in an absolute state of rest with respect to the aether,
and was therefore to be preferred over moving frames.

Experiments, however, failed to detect this mysterious aether. Apparently
it surrounded everything, and even penetrated inside physical objects; if light was
a wave vibrating through the aether, then apparently there was aether inside window
glass or the human eye. It was also surprisingly difficult to get a grip on this
aether. Light can also travel through a vacuum (as when sunlight comes to the
earth through outer space), so aether,
it seemed, was immune to vacuum pumps.

Einstein decided that none of this made sense. If the aether was impossible to
detect or manipulate, one might as well say it didn't exist at all. If the aether
doesn't exist, then what does it mean when our experiments show that light has
a certain speed, $3\times10^8$ meters per second? What is this speed relative to?
Could we, at least in theory, get on the motorcycle of Einstein's teenage daydreams,
and travel alongside a beam of light? In this frame of reference, the beam's speed
would be zero, but all experiments seemed to show that
the speed of light always came out the same, $3\times10^8$ m/s. Einstein decided
that the speed of light was dictated by a fundamental law of physics, so it must
be the same in all frames of reference. This put both light and matter on the same
footing: both obeyed laws of physics that were the same in all frames of reference.
\begin{important}[The principle of relativity]
Experiments don't come out different due to the straight-line,
constant-speed motion of the apparatus. This includes both light and
matter.
\end{important}
\noindent This is almost the same as Galileo's principle of inertia, except that we
explicitly state that it applies to light.
\index{light!speed of}\index{relativity!principle of}\index{inertia!principle of relativity}

This is hard to swallow. If a dog is running away from me at
5 m/s relative to the sidewalk, and I run after it at 3 m/s,
the dog's velocity in my frame of reference is 2 m/s.
According to everything we have learned about motion, the
dog must have different speeds in the two frames: 5 m/s in
the sidewalk's frame and 2 m/s in mine. How, then, can a
beam of light have the same speed as seen by someone who is
chasing the beam?\index{velocity!addition of}

In fact the strange constancy of the speed of light had already
shown up in the now-famous Michelson-Morley experiment of
1887. Michelson and Morley set up a clever apparatus to
measure any difference in the speed of light beams traveling
east-west and north-south. The motion of the earth around
the sun at 110,000 km/hour (about 0.01\% of the speed of
light) is to our west during the day. Michelson and Morley
believed in the aether hypothesis, so they expected that the
speed of light would be a fixed value relative to the aether.
As the earth moved through the aether, they thought they
would observe an effect on the velocity of light along an
east-west line. For instance, if they released a beam of
light in a westward direction during the day, they expected
that it would move away from them at less than the normal
speed because the earth was chasing it through the aether.
They were surprised when they found that the expected 0.01\%
change in the speed of light did not occur.\index{Michelson-Morley experiment}\index{aether}

Although the Michelson-Morley experiment was nearly two
dec\-ades in the past by the time Einstein published his first
paper on relativity in 1905, he did not even know of the
experiment until after submitting the paper.\footnote{Actually there is
some controversy on this historical point.} At this time he
was still working at the Swiss patent office, and was
isolated from the mainstream of physics.

How did Einstein explain this strange refusal of light waves
to obey the usual rules of addition and subtraction of
velocities due to relative motion? He had the originality
and bravery to suggest a radical solution. He decided that
space and time must be stretched and compressed as seen by
observers in different frames of reference. Since velocity
equals distance divided by time, an appropriate distortion
of time and space could cause the speed of light to come out
the same in a moving frame. This conclusion could have been
reached by the physicists of two generations before,
but the
attitudes about absolute space and time stated by Newton
were so strongly ingrained that such a radical approach didn't
occur to anyone before Einstein.

\mysection{Distortion of Time and Space}\index{time!relativistic effects}
\begin{envsubsection}{Time}
Consider the situation shown in figure  \figref{zigzag}. Aboard
a rocket ship we have a tube with mirrors at the ends. If we
let off a flash of light at the bottom of the tube, it will
be reflected back and forth between the top and bottom. It
can be used as a clock; by counting the number of times the
light goes back and forth we get an indication of how much
time has passed: up-down up-down, tick-tock tick-tock. (This may not seem very practical, but a
real atomic clock works on essentially the same
principle.) Now imagine that the rocket is cruising at a
significant fraction of the speed of light relative to the
earth. Motion is relative, so for a person inside the
rocket,  \subfigref{zigzag}{1}, there is no detectable change in the behavior
of the clock, just as a person on a jet plane can toss a
ball up and down without noticing anything unusual. But to
an observer in the earth's frame of reference, the light
appears to take a zigzag path through space, \subfigref{zigzag}{2}, increasing
the distance the light has to travel.

\widefigsidecaption{zigzag}{A light beam bounces between two mirrors in a
spaceship.}

If we didn't believe in the principle of relativity, we
could say that the light just goes faster according to the
earthbound observer. Indeed, this would be correct if the
speeds were much less than the speed of light, and if the
thing traveling back and forth was, say, a ping-pong ball.
But according to the principle of relativity, the speed of
light must be the same in both frames of reference. We are
forced to conclude that time is distorted, and the
light-clock appears to run more slowly than normal as seen
by the earthbound observer. In general, a clock appears to
run most quickly for observers who are in the same state of
motion as the clock, and runs more slowly as perceived by
observers who are moving relative to the clock.


We can easily calculate the size of this time-distortion effect.
In the frame of reference shown in figure \subfigref{zigzag}{1}, moving
with the spaceship, let $t$ be the time required for the beam
of light to move from the bottom to the top. An observer on the
earth, who sees the situation shown in figure \subfigref{zigzag}{2},
disagrees, and says this motion took a longer time $T$ (a bigger
letter for the bigger time).
Let $v$ be the velocity of the spaceship relative to the earth.
In frame 2, the light beam travels along the hypotenuse of
a right triangle, figure \figref{lightclocktriangle}, whose base has length
\begin{equation*}
	\text{base} = vT \qquad .
\end{equation*}
 Observers in the
two frames of reference agree on the vertical distance traveled by
the beam, i.e. the height of the triangle perceived in frame 2,
and an observer in frame 1 says that this height is the distance
covered by a light beam in time $t$, so the height is
\begin{equation*}
	\text{height} = ct \qquad ,
\end{equation*}
where $c$ is the speed of light.
The hypotenuse of this triangle is the distance the light travels
in frame 2,
\begin{equation*}
	\text{hypotenuse} = cT \qquad .
\end{equation*}
Using the Pythagorean theorem, we can relate these three quantities,
\begin{equation*}
	(cT)^2 = (vT)^2+(ct)^2 \qquad ,
\end{equation*}
and solving for $T$, we find
\begin{equation*}
	T =  \frac{t}{\sqrt{1-\left(v/c\right)^2}} \qquad .
\end{equation*}
\margup{-60mm}{
\fig{lightclocktriangle}{One observer says the light went a
distance $cT$, while the other says it only had to travel $ct$.}}


The amount of distortion is given by the factor
$1/\sqrt{1-\left(v/c\right)^2}$, and this quantity appears so often that we
give it a special name, $\gamma$ (Greek letter gamma),
\begin{equation*}
	\gamma =  \frac{1}{\sqrt{1-\left(v/c\right)^2}}\qquad .
\end{equation*}

\selfcheck{gammaatvzero}{What is $\gamma$ when $v$=0? What does this mean?}
\end{envsubsection}

\widefig{gammagraph}{The behavior of the $\gamma$ factor.}

\begin{envsubsection}{Space}\index{space!relativistic effects}
The speed of light is supposed to be the same in all frames of reference,
and a speed is a distance divided by a time. We can't change time without
changing distance, since then the speed couldn't come out the same.
If time is distorted by a factor of $\gamma$, then lengths must also be distorted
according to the same ratio. An object in motion appears longest to someone
who is at rest with respect to it, and is shortened along the direction of motion
as seen by other observers.
\end{envsubsection}
%
\begin{envsubsection}{No simultaneity}\index{simultaneity}
Part of the concept of absolute time was the assumption that
it was valid to say things like, ``I wonder what my uncle in
Beijing is doing right now.'' In the nonrelativistic
world-view, clocks in Los Angeles and Beijing could be
synchronized and stay synchronized, so we could
unambiguously define the concept of things happening
simultaneously in different places. It is easy to find
examples, however, where events that seem to be simultaneous
in one frame of reference are not simultaneous in another
frame. In figure \figref{simultaneity}, a flash of light is set off in
the center of the rocket's cargo hold. According to a
passenger on the rocket, the flashes have equal distances to
travel to reach the front and back walls, so they get there
simultaneously. But an outside observer who sees the rocket
cruising by at high speed will see the flash hit the back
wall first, because the wall is rushing up to meet it, and
the forward-going part of the flash hit the front wall
later, because the wall was running away from it. 

\widefigsidecaption{simultaneity}{Different observers don't agree that
the flashes of light hit the front and back of the ship
simultaneously.}

We saw on page \pageref{points-in-space-have-no-identity} that
points in space have no identity of their own: you may think
that two events happened at the same point in space, but
anyone else in a differently moving frame of reference says they
happened at different points in space. Relativity says that time
is the same way --- both simultaneity and ``simulplaceity'' are meaningless
concepts. Only when
the relative velocity of two frames is small compared to the
speed of light will observers in those frames agree on the
simultaneity of events. 


\subsubsection{The garage paradox}\index{garage paradox}
One of the most famous of all the so-called relativity
paradoxes has to do with our incorrect 
feeling that simultaneity is well defined. The idea is that
one could take a schoolbus and drive it at relativistic
speeds into a garage of ordinary size, in which it normally
would not fit. Because of the length contraction, the bus
would supposedly fit in the garage. The paradox arises when
we shut the door and then quickly slam on the brakes of the
bus. An observer in the garage's frame of reference will
claim that the bus fit in the garage because of its
contracted length. The driver, however, will perceive the
garage as being contracted and thus even less able to
contain the bus. The paradox is
resolved when we recognize that the concept of fitting the
bus in the garage ``all at once'' contains a hidden
assumption, the assumption that it makes sense to ask
whether the front and back of the bus can simultaneously be
in the garage. Observers in different frames of reference
moving at high relative speeds do not necessarily agree on
whether things happen simultaneously. The person in the
garage's frame can shut the door at an instant he perceives
to be simultaneous with the front bumper's arrival at the
opposite wall of the garage, but the driver would not agree
about the simultaneity of these two events, and would
perceive the door as having shut long after she plowed
through the back wall.

\widefigsidecaption{schoolbus}{In the garage's frame of reference, 1, the bus
is moving, and can fit in the garage. In the bus's frame of reference,
the garage is moving, and can't hold the bus.}

\end{envsubsection}
%
\begin{envsubsection}{Applications}
\subsubsection{Nothing can go faster than the speed of light.}
What happens if we want to send a rocket ship off
at, say, twice the speed of light, $v=2c$? Then $\gamma$ will be 
$1/\sqrt{-3}$. But
your math teacher has always cautioned you about the severe
penalties for taking the square root of a negative number.
The result would be physically meaningless, so we conclude
that no object can travel faster than the speed of light.
Even travel exactly at the speed of light appears to be
ruled out for material objects, since $\gamma$ would then be
infinite.

Einstein had therefore found a solution to his original
paradox about riding on a motorcycle alongside a beam of
light. The paradox is resolved because it is
impossible for the motorcycle to travel at the speed of
light.


Most people, when told that nothing can go faster than the
speed of light, immediately begin to imagine methods of
violating the rule. For instance, it would seem that by
applying a constant force to an object for a long time, we
would give it a constant acceleration which would eventually
result in its traveling faster than the speed of light. 
We'll take up these issues in section \ref{sec:reldynamics}.

\subsubsection{Cosmic-ray muons}\index{cosmic rays}\index{muons}
A classic experiment to demonstrate time distortion uses
observations of cosmic rays.
Cosmic rays are protons and other atomic nuclei from outer
space. When a cosmic ray happens to come the way of our
planet, the first earth-matter it encounters is an air
molecule in the upper atmosphere. This collision then
creates a shower of particles that cascade downward and can
often be detected at the earth's surface. One of the more
exotic particles created in these cosmic ray showers is the
muon (named after the Greek letter mu, $\mu$). The reason muons
are not a normal part of our environment is that a muon is
radioactive, lasting only 2.2 microseconds on the average
before changing itself into an electron and two neutrinos. A
muon can therefore be used as a sort of clock, albeit a
self-destructing and somewhat random one! Figures \figref{muona} and \figref{muonb}
show the average rate at which a sample of muons decays,
first for muons created at rest and then for high-velocity
muons created in cosmic-ray showers. The second graph is
found experimentally to be stretched out by a factor of
about ten, which matches well with the prediction of
relativity theory:
\begin{align*}
	\gamma	&= 1/\sqrt{1-(v/c)^2} \\
			&= 1/\sqrt{1-(0.995)^2} \\
 			&\approx 10
\end{align*}
Since a muon takes many microseconds to pass through the
atmosphere, the result is a marked increase in the number of
muons that reach the surface.

\margup{-170mm}{\fig{muona}{Decay of muons created at rest with respect to the observer.}
\spacebetweenfigs
\fig{muonb}{Decay of muons moving at a speed of 0.995$c$ with respect to the observer.}}


\subsubsection{Time dilation for objects larger than the atomic scale}\index{supernovae}
Our world is (fortunately) not full of human-scale objects
moving at significant speeds compared to the speed of light.
For this reason, it took over 80 years after Einstein's
theory was published before anyone could come up with a
conclusive example of drastic time dilation that wasn't
confined to cosmic rays or particle accelerators. Recently,
however, astronomers have found definitive proof that entire
stars undergo time dilation. The universe is expanding in
the aftermath of the Big Bang, so in general everything in
the universe is getting farther away from everything else.
One need only find an astronomical process that takes a
standard amount of time, and then observe how long it
appears to take when it occurs in a part of the universe
that is receding from us rapidly. A type of exploding star
called a type Ia supernova fills the bill, and technology is
now sufficiently advanced to allow them to be detected
across vast distances. Figure \figref{supernovae} shows
convincing evidence for time dilation in the brightening and
dimming of two distant supernovae.

\widefigsidecaption{supernovae}{Light curves of supernovae, showing
a time-dilation effect for supernovae that are in motion relative
to us.}

\subsubsection{The twin paradox}\index{twin paradox}
A natural source of confusion in understanding the
time-dilation effect is summed up in the so-called twin
paradox, which is not really a paradox. Suppose there are
two teenaged twins, and one stays at home on earth while the
other goes on a round trip in a spaceship at relativistic
speeds (i.e., speeds comparable to the speed of light, for
which the effects predicted by the theory of relativity are
important). When the traveling twin gets home, he has aged
only a few years, while his brother is now old and gray.
(Robert Heinlein even wrote a science fiction novel on this
topic, although it is not one of his better stories.) 

The ``paradox'' arises from an incorrect application of the
principle of relativity to a description of the story from the
traveling twin's point of view. From his point of view, the
argument goes, his homebody brother is the one who travels
backward on the receding earth, and then returns as the
earth approaches the spaceship again, while in the frame of
reference fixed to the spaceship, the astronaut twin is not
moving at all. It would then seem that the twin on earth is
the one whose biological clock should tick more slowly, not
the one on the spaceship. The flaw in the reasoning is that
the principle of relativity only applies to frames that are
in motion at constant velocity relative to one another, i.e.,
inertial frames of reference. The astronaut twin's frame of
reference, however, is noninertial, because his spaceship
must accelerate when it leaves, decelerate when it reaches
its destination, and then repeat the whole process again on
the way home. Their experiences are not equivalent, because
the astronaut twin feels accelerations and decelerations.
A correct treatment requires some mathematical complication
to deal with the changing velocity of the astronaut twin, but
the result is indeed that it's the traveling twin who is
younger when they are reunited.

The twin ``paradox'' really isn't a paradox at all. It may even
be a part of your ordinary life.
The effect was first verified experimentally by synchronizing
two atomic clocks in the same room, and then sending one for a round
trip on a passenger jet. (They bought the clock its own ticket and
put it in its own seat.) The
clocks disagreed when the traveling one got back, and the discrepancy
was exactly the amount predicted by relativity.
The effects are strong enough to
be important for making the global positioning system (GPS)
work correctly. If you've ever taken a GPS receiver with you on a hiking
trip, then you've used a device that has the twin ``paradox'' programmed
into its calculations. Your handheld GPS box gets signals from a satellite,
and the satellite is moving fast enough that its time dilation is an important
effect.
So far no astronauts have gone fast enough to make time
dilation a dramatic effect in terms of the human lifetime.
The effect on the Apollo
astronauts, for instance, was only a fraction of a second,
since their speeds were still fairly small compared to the
speed of light. (As far as I know, none of the astronauts
had twin siblings back on earth!)

\widefigsidecaption{rhic}{Colliding nuclei show relativistic length contraction.}

\subsubsection{An example of length contraction}
Figure \figref{rhic} shows an
artist's rendering of the length contraction for the collision of two
gold nuclei at relativistic speeds in the RHIC accelerator\index{RHIC accelerator}
in Long Island, New York, which went on line in 2000.
The gold nuclei would appear nearly spherical (or just
slightly lengthened like an American football) in frames
moving along with them, but in the laboratory's frame, they
both appear drastically foreshortened as they approach the
point of collision. The later pictures show the nuclei
merging to form a hot soup, in which experimenters hope to
observe a new form of matter.
\end{envsubsection}

\pagebreak[4]

\dqheader
\begin{dq}
A person in a spaceship moving at 99.99999999\% of the
speed of light relative to Earth shines a flashlight forward
through dusty air, so the beam is visible. What does she
see? What would it look like to an observer on Earth?
\end{dq}
%
\begin{dq}\label{dq:illusion}
A question that students often struggle with is whether
time and space can really be distorted, or whether it just
seems that way. Compare with optical illusions or magic
tricks. How could you verify, for instance, that the lines
in the figure are actually parallel? Are relativistic
effects the same or not?
\end{dq}
%
\begin{dq}
On a spaceship moving at relativistic speeds, would a
lecture seem even longer and more boring than normal?
\end{dq}
%
\margup{-20mm}{\fig{dqillusion}{Discussion question \ref{dq:illusion}}}
\begin{dq}
Mechanical clocks can be affected by motion. For example,
it was a significant technological achievement to build a
clock that could sail aboard a ship and still keep accurate
time, allowing longitude to be determined. How is this
similar to or different from relativistic time dilation?
\end{dq}
%
\begin{dq}
What would the shapes of the two nuclei in figure \figref{rhic} on page \pageref{fig:rhic}
look like to a microscopic observer riding on the
left-hand nucleus? To an observer riding on the right-hand
one? Can they agree on what is happening? If not, why not
--- after all, shouldn't they see the same thing if they
both compare the two nuclei side-by-side at the same instant in time?
\end{dq}
%
\begin{dq}
If you stick a piece of foam rubber out the window of
your car while driving down the freeway, the wind may
compress it a little. Does it make sense to interpret the
relativistic length contraction as a type of strain that
pushes an object's atoms together like this? How does this
relate to the previous discussion question?
\end{dq}

\vfill\pagebreak[4]

\mysection{Dynamics}\label{sec:reldynamics}
So far we have said nothing about how to predict motion in
relativity. Do Newton's laws still work? Do conservation
laws still apply? The answer is yes, but many of the
definitions need to be modified, and certain entirely new
phenomena occur, such as the conversion of mass to energy
and energy to mass, as described by the famous equation
$E=mc^2$, which was discussed in section \ref{sec:massenergy}.
%
\begin{envsubsection}{Combination of velocities}\index{velocity!addition of!relativistic}
The impossibility of motion faster than light is a
 radical difference between relativistic and
nonrelativistic physics, and we can get at most of the
issues in this section by considering the flaws in various
plans for going faster than light. The simplest argument of
this kind is as follows. Suppose Janet takes a trip in a
spaceship, and accelerates until she is moving at $0.8c$ (80\%
of the speed of light) relative to the
earth. She then launches a space probe in the forward
direction at a speed relative to her ship of $0.4c$. Isn't the
probe then moving at a velocity of 1.2 times the speed of
light relative to the earth?

The problem with this line of reasoning is that although Janet
says the probe is moving at $0.4c$ relative to her, earthbound
observers disagree with her perception of time and space.
Velocities therefore don't add the same way they do in Galilean
relativity. Suppose we express all velocities as fractions of the
speed of light. The Galilean addition of velocities can be
summarized in this addition table:

\widefigsidecaption{veltablea}{Galilean addition of velocities.}

\noindent The derivation of the correct relativistic result requires some tedious algebra,
which you can find in my book \emph{Simple Nature} if
you're curious. I'll just state the numerical results here:

\widefigsidecaption{veltableb}{Relativistic addition of velocities. The green oval near the center
of the table describes velocities that are relatively small compared to the speed of
light, and the results are approximately the same as the Galilean ones. The edges of
the table, highlighted in blue, show that everyone agrees on the speed of light.}

Janet's probe, for example, is moving not at $1.2c$ but at $0.91c$, which is
a drastically different result. The difference between the two tables is most evident
around the edges, where all the results are equal to the speed
of light. This is required by the principle of relativity. For example, if Janet sends
out a beam of light instead of a probe, both she and the earthbound observers must
agree that it moves at $1.00$ times the speed of light, not $0.8+1=1.8$.
On the other hand, the correspondence principle requires that the relativistic
result should correspond to ordinary addition for low enough velocities,
and you can see that the tables are nearly identical in the
center.\index{correspondence principle!for relativistic addition of velocities}

\end{envsubsection}
%
\begin{envsubsection}{Momentum}
Here's another flawed scheme for traveling faster than the speed of light.
The basic idea can be demonstrated by dropping a ping-pong ball and a baseball
stacked on top of each other like a snowman. They separate slightly in mid-air,
and the baseball therefore has time to hit the floor and rebound before it
collides with the ping-pong ball, which is still on the way down. The result is
a surprise if you haven't seen it before: the ping-pong ball flies off at high
speed and hits the ceiling! A similar fact is known to people who investigate the
scenes of accidents involving pedestrians. If a car moving at 90 kilometers per
hour hits a pedestrian, the pedestrian flies off at nearly double that speed, 180
kilometers per hour. Now suppose the car was moving at 90 percent of the speed of
light. Would the pedestrian fly off at 180\% of $c$?

To see why not, we have to back up a little and think about where this speed-doubling
result comes from. The introduction of momentum in chapter \ref{ch:momentum} depended
on the idea of finding a frame of reference, the center-of-mass frame, in which the
two colliding objects (assumed to be equal in mass) approached each other symmetrically,
collided, and rebounded with their velocities reversed. In the center-of-mass frame,
the total momentum of the objects was zero both before and after the collision.

\widefigsidecaption{unequalcollision}{An unequal collision, viewed in the center-of-mass frame, 1, and
in the frame where the small ball is initially at rest, 2.}

Figure \subfigref{unequalcollision}{1} shows a similar frame of reference for objects of unequal mass.
Before the collision, the large ball is moving relatively slowly toward the top of the page, but
because of its greater mass, its momentum cancels the momentum of the smaller ball, which is
moving rapidly in the opposite direction. The total momentum is zero. After the collision, the
two balls just reverse their directions of motion. We know that this is the
right result for the outcome of the collision because
it conserves both momentum and kinetic energy, and everything not forbidden is mandatory.

\selfcheck{unequalcollisioncons}{How do we know that momentum and kinetic energy are conserved
in figure \subfigref{unequalcollision}{1}?}

Let's make up some numbers as an example. Say the small ball has a mass of 1 kg, the big
one 8 kg. In frame 1, let's make the velocities as follows:

\newcommand{\smallvelocitytable}[6]{%
\noindent\hspace{5mm}\begin{tabular}{|p{4mm}|p{40mm}|p{40mm}|}
\hline
 & before the collision
       & after the collision  \\
\hline
\anonymousinlinefig{#5} & #1 & #2  \\
\anonymousinlinefig{#6} & #3 & #4  \\
\hline
\end{tabular}
}

\smallvelocitytable{-0.8}{0.8}{0.1}{-0.1}{smallball}{bigball}

Figure \subfigref{unequalcollision}{2} shows the same collision in a frame of reference where
the small ball was initially at rest.
 To find all the velocities in this frame, we
just add 0.8 to all the ones in the previous table.

\smallvelocitytable{0}{1.6}{0.9}{0.7}{smallball}{bigball}

\noindent In this frame, as expected, the small ball flies off with a velocity, 1.6, that
is almost twice the initial velocity of the big ball, 0.9.

If all those velocities were in meters per second, then that's exactly what happened. But
what if all these velocities were in units of the speed of light? Now it's no longer a good
approximation just to add velocities. We need to combine them according to the relativistic
rules. For instance, the table
on page \pageref{fig:veltableb} tells us that combining a velocity of 0.8 times the
speed of light with another velocity of 0.8 results in 0.98, not 1.6. The results are very different:

\smallvelocitytable{0}{0.98}{0.83}{0.76}{smallball}{bigball}
% gamma = 1, 5.0, 1.8, 1.5
% 1/gamma = 1, 0.2, 0.56, .67

\widefigsidecaption{unequalrel}{An 8-kg ball moving at 83\% of the speed of light hits a 1-kg ball. The balls
appear foreshortened due to the relativistic distortion of space.}

We can interpret this as follows. Figure \subfigref{unequalcollision}{1} is one in which the
big ball is moving fairly slowly. This is very nearly the way the scene would be seen by an
ant standing on the big ball. According to an observer in frame \figref{unequalrel}, however,
both balls are moving at nearly the speed of light after the collision. Because of this, the
balls appear foreshortened, but the distance between the two balls is also shortened. To this
observer, it seems that the small ball isn't pulling away from the big ball very fast.

Now here's what's interesting about all this. The outcome shown in figure \subfigref{unequalcollision}{2}
was supposed to be the only one possible, the only one that satisfied both conservation of energy
and conservation of momentum. So how can the \emph{different} result shown in figure
\figref{unequalrel} be possible? The answer is that relativistically, momentum must not equal
$mv$. The old, familiar definition is only an approximation that's valid at low speeds. If
we observe the behavior of the small ball in figure \figref{unequalrel}, it looks as though it
somehow had some extra inertia. It's as though a football player tried to knock another player
down without realizing that the other guy had a three-hundred-pound bag full of lead shot
hidden under his uniform --- he just doesn't seem to react to the collision as much as he should.
This extra inertia is described by redefining momentum as
\begin{equation*}
	\text{momentum} = m\gamma v \qquad .
\end{equation*}
At very low velocities, $\gamma$ is close to 1, and the result is very nearly $mv$, as demanded
by the correspondence principle. But at very high velocities, $\gamma$ gets very big --- the
small ball in figure \figref{unequalrel} has a $\gamma$ of 5.0, and therefore has five times
more inertia than we would expect nonrelativistically.\index{correspondence principle!for relativistic momentum}

This also explains the answer to another paradox often posed by beginners at relativity.
Suppose you keep on applying a steady force to an object that's already moving at $0.9999c$.
Why doesn't it just keep on speeding up past $c$? The answer is that force is the rate of
change of momentum. At $0.9999c$, an object already has a $\gamma$ of 71, and therefore
has already sucked up 71 times the momentum you'd expect at that speed. As its velocity gets closer and
closer to $c$, its $\gamma$ approaches infinity. To move at $c$, it would need an infinite
momentum, which could only be caused by an infinite force.
\end{envsubsection}
%
\begin{envsubsection}{Equivalence of mass and energy}
Now we're ready to see why mass and energy must be equivalent as claimed
in section \ref{sec:massenergy}. So far we've only considered collisions
in which none of the kinetic energy is converted into any other form
of energy, such as heat or sound.
Let's consider what happens if a blob of putty moving at
velocity $v$ hits another blob that is initially at rest,
sticking to it.  The nonrelativistic result is
that to obey conservation of momentum the two blobs must fly
off together at $v/2$. Half of the initial kinetic energy
has been converted to heat.\footnote{A double-mass object moving
at half the speed does not have the same kinetic energy. Kinetic
energy depends on the square of the velocity, so cutting the velocity
in half reduces the energy by a factor of 1/4, which, multiplied
by the doubled mass, makes 1/2 the original energy.}

Relativistically, however, an interesting thing happens. A
hot object has more momentum than a cold object! This is
because the relativistically correct expression for momentum
is $m\gamma v$, and the more rapidly moving atoms in the hot
object have higher values of $\gamma$.
In our collision, the final combined blob must therefore be
moving a little more slowly than the expected $v/2$, since
otherwise the final momentum would have been a little
greater than the initial momentum. To an observer who
believes in conservation of momentum and knows only about
the overall motion of the objects and not about their heat
content, the low velocity after the collision would seem
to be the result of a magical change in the mass, as if the mass
of two combined, hot blobs of putty was more than the sum of
their individual masses.

Now we know that the masses of all the atoms in the blobs
must be the same as they always were. The change is due to
the change in $\gamma$ with heating, not to a change in mass.
The heat energy, however, seems to be acting as if it was
equivalent to some extra mass.

But this whole argument was based on the fact that heat is a
form of kinetic energy at the atomic level. Would $E=mc^2$
apply to other forms of energy as well? Suppose a rocket
ship contains some electrical energy stored in a
battery. If we believed that $E=mc^2$ applied to forms of
kinetic energy but not to electrical energy, then
we would have to believe that the pilot of the rocket could
slow the ship down by using the battery to run a heater!
This would not only be strange, but it would violate the
principle of relativity, because the result of the
experiment would be different depending on whether the ship
was at rest or not. The only logical conclusion is that all
forms of energy are equivalent to mass. Running the heater
then has no effect on the motion of the ship, because the
total energy in the ship was unchanged; one form of energy (electrical)
was simply converted to another (heat).
\end{envsubsection}
%===============================================================================
%===============================================================================

\begin{hwsection}

\begin{hw}{agreeontime}
Astronauts in three different spaceships are communicating with each other.
Those aboard ships A and B agree on the rate at which time is passing, but
they disagree with the ones on ship C. \\
(a) Describe the motion of the other two ships according to Alice, who is aboard
ship A. \\
(b) Give the description according to Betty, whose frame of reference is ship B.\\
(c) Do the same for Cathy, aboard ship C.
\end{hw}

\begin{hw}{lightclockwithruler}
(a) Figure \figref{lightclocktriangle} on page \pageref{fig:lightclocktriangle} is based on a light clock moving at a certain
speed, $v$. By measuring with a ruler on the figure, determine $v/c$.\\
(b) By similar measurements, find the time contraction factor $\gamma$, which equals $T/t$.\\
(c) Locate your numbers from parts a and b as a point on the graph in figure \figref{gammagraph}
on page \pageref{fig:gammagraph}, and check that it actually lies on the curve. Make a sketch showing
where the point is on the curve.\
\end{hw}

\begin{hw}{lightclockdoublespeed}
This problem is a continuation of problem \ref{hw:lightclockwithruler}.
Now imagine that the spaceship speeds up to twice the velocity. Draw a new triangle,
using a ruler to make the lengths of the sides accurate. Repeat parts b and c for this new diagram.
\end{hw}

\begin{hw}{gammafornegativev}
What happens in the equation for $\gamma$ when you put in a negative number for $v$? Explain
what this means physically, and why it makes sense.
\end{hw}

\begin{hw}{supernovagraph}
(a) By measuring with a ruler on the graph in figure \figref{supernovae} on page \pageref{fig:supernovae},
estimate the $\gamma$ values of the two supernovae.\\
(b) Figure \figref{supernovae} gives the values of $v/c$. From these, compute $\gamma$ values and
compare with the results from part a.\\
(c) Locate these
two points on the graph in figure \figref{gammagraph}, and make a sketch showing where they lie.
\end{hw}

\begin{hw}{voyagergamma}\index{Voyager space probe}
The Voyager 1 space probe, launched in 1977, is moving faster relative to the earth than
any other human-made object, at 17,000 meters per second. \\
(a) Calculate the probe's $\gamma$. \\
(b) Over the course of one year on earth, slightly less than one year passes on the probe.
How much less? (There are 31 million seconds in a year.)
\end{hw}

\begin{hw}{freeneutron}
(a) A free neutron (as opposed to a neutron bound into
an atomic nucleus) is unstable, and decays radioactively
into a proton, an electron, and a particle called an
antineutrino, which fly off in three different directions.
 The masses are as follows:

\qquad\begin{tabular}{ll}
	neutron	        & $1.67495\times10^{-27}$  kg\\
	proton	        & $1.67265\times10^{-27}$  kg\\
	electron	& $0.00091\times10^{-27}$  kg\\
	antineutrino	& negligible\\
\end{tabular}

\noindent Find the energy released in the decay of a free neutron. \hwendpart
(b) Neutrons and protons make up essentially all of the mass of the ordinary
matter around us. We observe that the universe around us has no free neutrons, but
lots of free protons
(the nuclei of hydrogen, which is the element that 90\% of the universe
is made of). We find neutrons only inside nuclei along with other neutrons and
protons, not on their own.

If there are processes that can convert neutrons into protons, we might imagine
that there could also be proton-to-neutron conversions, and indeed such a process
does occur sometimes in nuclei that contain both neutrons and protons:
a proton can decay into a
neutron, a positron, and a neutrino. A positron is a particle with the same
properties as an electron, except that its electrical charge is positive
(see chapter 7). A neutrino, like an antineutrino, has negligible mass.

Although such a process
can occur within a nucleus, explain why it cannot happen to
a free proton. (If it could, hydrogen would be radioactive, and you
wouldn't exist!)
\end{hw}

\begin{hw}[2]{vintermsofp}
(a) Find a relativistic equation for the velocity of an
object in terms of its mass and momentum (eliminating
$\gamma$). For momentum, use the symbol $p$, which is standard notation. \hwendpart
(b) Show that your result
is approximately the same as the classical value, $p/m$, at
low velocities.\hwendpart
(c) Show that very large momenta result in
speeds close to the speed of light.
\end{hw}

\begin{hw}{gammasimplefraction}
(a) Show that for $v=(3/5)c$, $\gamma$ comes out to be a simple fraction.\\
(b) Find another value of $v$ for which $\gamma$ is a simple fraction.
\end{hw}

\begin{hw}{sportsinslowlightland}
In Slowlightland, the speed of light is 20 mi/hr = 32 km/hr = 9 m/s. Think of an example of how
relativistic effects would work in sports. Things can get very complex very quickly, so try to think of a
simple example that focuses on just one of the following effects:
\begin{itemize}
 \item[] relativistic momentum
 \item[] relativistic addition of velocities
 \item[] time dilation and length contraction
 \item[] equivalence of mass and energy
 \item[] time it takes for light to get to an athlete
\end{itemize}
\end{hw}

\end{hwsection}

\startscanswers{ch:relativity}

\scanshdr{gammaatvzero} At $v=0$, we get $\gamma=1$, so $t=T$. There is
no time distortion unless the two frames of reference are in relative motion.

\scanshdr{unequalcollisioncons} The total momentum is zero before the collision. After the
collision, the two momenta have reversed their directions, but they still cancel.
Neither object has changed its kinetic energy, so the total energy before and after the collision
is also the same.
