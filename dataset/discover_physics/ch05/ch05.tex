\newcommand{\velocitytable}[8]{%
\noindent\hspace{5mm}\begin{tabular}{|p{4mm}|p{30mm}|p{30mm}|p{20mm}|}
\hline
 & \multicolumn{2}{p{50mm}}{velocity (meters per second)} & \\
\cline{2-4}
 & before the collision
       & after the collision  & change \\
\hline
\anonymousinlinefig{#7} & #1 & #2 & #3 \\
\anonymousinlinefig{#8} & #4 & #5 & #6 \\
\hline
\end{tabular}
}
%
%------------------------------------------------------------------------------
\mychapterwithopener{pool}{Pool balls exchange momentum.}{Conservation of Momentum}\label{ch:momentum}

Physicist Murray Gell-Mann invented a wonderful phrase that has since\index{Gell-Mann, Murray}
entered into popular culture: ``Everything not forbidden is compulsory.''
Although people now use it as a sarcastic political statement,
Gell-Mann was just employing politics as a metaphor for physics. What he meant
was that the laws of physics forbid all the impossible things, and what's
left over is what really happens.
Conservation of mass and energy prevent many things from happening. Objects
can't disappear into thin air, and you can't run your car forever without
putting gas in it.

Some other processes are impossible, but not forbidden
by these two conservation laws. 
In the martial arts movie \emph{Crouching Tiger, Hidden Dragon},
those who have received mystical enlightenment are able to violate the
laws of physics. Some of the violations are obvious, such as their ability
to fly, but others are a little more subtle. The rebellious young
heroine/antiheroine Jen Yu gets into an argument while sitting
at a table in a restaurant. A young tough, Iron Arm Lu, comes running toward her at full
speed, and she puts up one arm and effortlessly makes him bounce
back, without even getting out of her seat or bracing herself against
anything. She does all this between bites.
It's impossible, but how do we know it's impossible? It doesn't violate conservation
of mass, because neither character's mass changes. It conserves energy as well,
since the rebounding Lu has the same energy he started with.\label{tiger}

Suppose you live in a country where the only laws are prohibitions against
murder and robbery. One day someone covers your house with graffiti, and the
authorities refuse to prosecute, because no crime was committed. You're
convinced of the need for a new law against vandalism. Similarly, the story of
Jen Yu and Iron Arm Lu shows that we need a new conservation law.

\mysection[0]{Translation Symmetry}\index{symmetry!translation}
The most fundamental laws of physics are conservation laws, and Noether's theorem
tells us that conservation laws are the way they are because of symmetry. Time-translation
symmetry is responsible for conservation of energy, but time is like a
river with only two directions, past and future. What's impossible about Lu's
motion is the abrupt reversal in the direction of his motion in \emph{space}, but neither
time-translation symmetry nor energy conservation tell us anything about directions
in space. When you put gas in your car, you don't have to decide whether you want to
buy north gas or south gas, east, west, up or down gas. Energy has no direction. What
we need is a new conserved quantity that has a direction in space, and such a
conservation law can only come from a symmetry that relates to space. Since we've already
had some luck with time-translation symmetry, it seems reasonable to turn now to
space-translation symmetry, which I introduced on page
\pageref{space-translation-symmetry} but haven't mentioned since.

\label{points-in-space-have-no-identity}
Space-translation symmetry would seem reasonable to most people, but you'll see that it
ends up producing some very surprising results. To see how, it will be
helpful to imagine the consequences of a violation of space-translation symmetry.
What if, like the laws of nations, the laws of physics were different in different
places? What would happen, and how would we detect it? We could try doing the
same experiment in two different places and comparing the results, but it's
even easier than that. Tap your finger on this spot on the page
\begin{equation*}
	\times
\end{equation*}
and then wait a second and do it again. Did both taps occur at the same point in space?
You're probably thinking that's a silly question; am I just checking whether you followed
my directions? Not at all. Consider the whole scene from the point of view of a Martian
who is observing it through a powerful telescope from her home planet. (You didn't
draw the curtains, did you?) From her point of view, the earth is spinning
on its axis and orbiting the sun, at speeds measured in thousands of kilometers per hour.
According to her, your second finger tap happened at a point in space about
30 kilometers from the first. If you want to impress the Martians and win the Martian version of
the Nobel Prize for detecting a violation of space-translation symmetry,
all you have to do is perform a physics experiment twice in the same laboratory, and
show that the result comes out different.

But who's to say that the Martian point of view is the right one? It gets a little thorny
now. How do you know that what you detected was a violation of space-translation symmetry
at all? Maybe it was just a violation of time-translation symmetry. The Martian Nobel
committee isn't going to give you the prize based on an experiment this ambiguous. A
possible scheme for resolving the ambiguity would be to wait a year and do the same
experiment a third time. After a year, the earth will have completed one full orbit around the
sun, and your lab will be back in the same spot in space. If the third experiment comes out
the same as the first one, then you can make a strong argument that what you've detected is
an asymmetry of space, not time. There's a problem, however. You and the Martians agree
that the earth is back in the same place after a year, but what about an observer from
another solar system, whose planet orbits a different star? This observer says that our
whole solar system is in motion. To him, the earth's motion around our sun looks like a spiral or a
corkscrew, since the sun is itself moving.

\mysection[0]{The Strong Principle of Inertia}\index{inertia!strong principle of}\label{sec:strong-inertia}
\begin{envsubsection}{Symmetry and inertia}
This story shows that space-translation symmetry is closely related to the relative nature
of motion. Riding in a train on a long, straight track at constant speed, how can you even
tell you're in motion? You can look at the scenery outside, but that's irrelevant, because
we could argue that the trees and cows are moving while you stand still. (The Martians
say both train and scenery are moving.) The real point is whether you can
detect your motion without reference to any external object. You can hear the repetitive
thunk-thunk-thunk as the train passes from one piece of track to the next, but again
this is just a reference to an external object --- all that proves is that you're moving
relative to the tracks, but is there any way to tell that you're moving in some
\emph{absolute} sense? Assuming no interaction with the outside world, 
is there any experiment you can do that will come out
different when the train is in motion than when it's at rest?
You could if space-translation symmetry was violated. If the laws of physics were different
in different places, then as the train moved it would pass through them. ``Riding over''
these regions would be like riding over the pieces of track, but you would be able
to detect the transition from one region to the next simply because experiments inside
the train came out different, without referring to any external objects.
Rather than the thunk-thunk-thunk of the rails, you would
detect increases and decreases in some quantity such as the gravitational constant $G$,
or the speed of light, or the mass of the electron.

We can therefore conclude that the following two hypotheses are closely related.

\begin{important}[The principle of inertia (strong version)]
Experiments don't come out different due to the straight-line,
constant-speed motion of the apparatus. 
\end{important}

\begin{important}[Space-translation symmetry]
The laws of physics are the same at every point in space. Specifically, experiments
don't come out different just because you set up your apparatus in a different place.
\end{important}

\begin{eg}{A state of absolute rest}\label{eg:absoluterest}
Suppose that space-translation symmetry is violated. The laws of phys\-ics are different
in one region of space than in another. Cruising in our spaceship, we monitor the
fluctuations in the laws of physics by watching the needle on a meter that measures
some fundamental quantity such as the gravitational constant. We make a short blast
with the ship's engines and turn them off again. Now we see that the needle is wavering
more slowly, so evidently it's taking us more time to move from one region to the next.
We keep on blasting with the ship's engines until the fluctuations stop entirely. Now
we know that we're in a state of absolute rest. The violation of translation
symmetry logically resulted in a violation of the principle of inertia.
\end{eg}

\selfcheck{moonviolatestranslation}{Suppose you do an experiment to see how long it takes
for a rock to drop one meter. This experiment comes out different if you do it on the moon.
Does this violate space-translation symmetry?}

People have a strong intuitive belief that there is a state of absolute rest,
and that the earth's surface defines it. But Copernicus proposed as a mathematical
assumption, and Galileo argued as a matter of physical reality, that the earth spins
on its axis, and also circles the sun. Galileo's opponents objected that this was
impossible, because we would observe the effects of the motion. They said, for example,
that if the earth was moving, then you would never be able to jump up in the air and
land in the same place again --- the earth would have moved out from under you.
Galileo realized that this wasn't really an argument about the earth's motion but
about physics. In one of his books, which were written in the form of dialogues, he has
the three characters debate what would happen if a ship was cruising smoothly across
a calm harbor and a sailor climbed up to the top of its mast and dropped a rock.
Would it hit the deck at the base of the mast, or behind it because the ship had moved out from
under it? This is the kind of experiment referred to in the strong principle of
inertia, and Galileo knew that it would come out the same regardless of the ship's
motion. His opponents' reasoning, as represented by the dialog's stupid character
Simplicio, was based on the assumption that once the rock lost contact with the sailor's
hand, it would naturally start to lose its forward motion. In other words,
they didn't even believe in the weak principle of inertia
(page \pageref{weak-principle-of-inertia}), which states that motion doesn't naturally
slow down.

The strong principle of inertia says more than that. It says that motion isn't even real:
to a sailor standing on the deck of the ship, the deck and the masts and the rigging are not even
moving. People on the shore can tell him that the ship and his own body are moving in a straight
line at constant speed. He can reply, ``No, that's an illusion. I'm at rest. The only reason you think I'm moving is
because you and the sand and the water are moving in the opposite direction.''
The strong principle of inertia says that straight-line, constant-speed motion is a matter of opinion.
The weak principle of inertia is then a logical
byproduct: things can't ``naturally'' slow down and stop moving, because we can't even agree on which things
are moving and which are at rest.

If observers in different frames of reference disagree on velocities, it's natural to
want to be able to convert back and forth. For motion in one dimension, this can be
done by simple addition.\index{velocity!addition of}

\begin{eg}{A sailor running on the deck}\label{eg:runtobow}
\egquestion
A sailor is running toward the front of a ship, and the other sailors say that in
their frame of reference, fixed to the deck, his velocity is $7.0$ m/s. The ship
is moving at $1.3$ m/s relative to the shore. How fast does an observer on the beach
say the sailor is moving?

\eganswer
They see the ship moving at $7.0$ m/s, and the sailor moving even faster than that because
he's running from the stern to the bow. In one second, the ship moves $1.3$ meters,
but he moves $1.3+7.0$ m, so his velocity relative to the beach is $8.3$ m/s. 
\end{eg}

The only way to make this rule come out consistent is if we define velocities in
one direction as positive and velocities in the opposite direction as negative.

\begin{eg}{Running back toward the stern}
\egquestion
The sailor of example \ref{eg:runtobow} turns around and runs back toward the stern
at the same speed relative to the deck. How do the other sailors describe this
velocity mathematically, and what do observers on the beach say?

\eganswer
Since the other sailors described his original velocity as positive, they have to call
this negative. They say his velocity is now $-7.0$ m/s. A person on the shore says
his velocity is $1.3+(-7.0)=-5.7$ m/s.
\end{eg}

\end{envsubsection}
%
\begin{envsubsection}{Inertial and noninertial frames}
Let's not overstate this. Is \emph{all} motion a matter of opinion? No --- try telling
that\index{frame of reference!inertial}\index{frame of reference!noninertial}%
\index{inertial frame of reference}\index{noninertial frame of reference}
to the brave man in figure \figref{sled}! He's the one who feels the effects of the motion,
not the observers standing by the track. Even if he can pull his face together enough to
speak, he won't have much luck convincing them that his motion is an illusion, and that they're
the ones who are really moving backward while his rocket sled is standing still.
Only straight-line, constant-speed motion is a matter of opinion. His speed is changing, and
the change in speed produces real effects. Experiments do come out different if your apparatus
is \emph{changing} its speed. A frame of reference whose motion is changing is called a
noninertial frame of reference, because the principle of inertia doesn't apply to it.

\widefigsidecaption{sled}{This Air Force doctor volunteered to ride a rocket sled as a medical
experiment. The obvious effects on his head and face are not because of the sled's speed
but because of its rapid changes in speed: increasing in 2 and 3, and decreasing in
5 and 6. In 4 his speed is greatest, but because his speed is not increasing or
decreasing very much at this moment, there is little effect on him.}

Experiments also come out different if your apparatus is changing its direction of motion.
The landscape around you is moving in a circle right now due to the rotation
of the Earth, and is therefore changing the direction of its motion continuously
on a 24-hour cycle. However, the curve of the motion is so gentle that under
ordinary conditions we don't notice that the local dirt's frame of reference isn't
quite inertial. The first demonstration of the noninertial nature of the earth-fixed
frame of reference was by Foucault\index{Foucault}
 using a very massive pendulum whose oscillations
would persist for many hours. Although Foucault did
his demonstration in Paris, it's easier to imagine what would happen at the north pole:
the pendulum would keep swinging in the same plane, but the earth would spin underneath
it once every 24 hours. To someone standing in the snow, it would appear that the
pendulum's plane of motion was twisting. The effect at latitudes less than 90
degrees turns out to be slower, but otherwise similar. The Foucault pendulum was
the first definitive experimental proof that the earth really did spin on its axis,
although scientists had been convinced of its rotation for a century based on more
indirect evidence about the structure of the solar system.

\margup{-35mm}{\fig{foucault}{Foucault demonstrates his pendulum to an audience at
		a lecture in 1851.}}
\label{pickuptrucklinear}
Often when we adopt a noninertial frame of reference, there is a vivid illusion
that the laws of physics are being violated. It might seem like the Foucault pendulum was
being influenced by evil spirits, if you forgot that it was actually the ground that was
twisting around, not the pendulum.
A simpler example is shown in figure \figref{pickuptrucklinear}.
A bowling ball is in the back of a pickup truck, and the driver steps on the brakes. Because
the truck is changing its speed, a frame of reference that moves with the truck
is noninertial. For the driver,
there is a strong psychological tendency to adopt this bad frame of reference, \subfigref{pickuptrucklinear}{1}, but then
the bowling ball seems to be violating the laws of physics: according to the weak principle
of inertia, the ball has no reason to start rolling toward the front of the truck. It's not interacting
with any other object that would cause it to do this.
In figure \subfigref{pickuptrucklinear}{2}, we watch the motion in an (approximately) inertial frame of 
reference fixed to the sidewalk, and everything makes sense. The ball obeys the weak principle of
inertia, and moves equal distances in equal time intervals. In this frame, it's the truck that
changes its speed, which makes sense, because the truck's wheels are interacting with the pavement.

\widefigsidecaption{pickuptrucklinear}{A bowling ball in the back of a pickup truck is viewed in a noninertial
frame, 1, and an inertial one, 2.}

\margup{-21mm}{\fig{galileotrial}{Galileo on trial before the Inquisition.}}
Popular belief has Galileo being prosecuted\index{Galileo!trial}\index{Church, Catholic}\index{Catholic Church}
by the Catholic Church for saying the earth rotated on its axis and also orbited the sun, but
Foucault's pendulum was still centuries in the future, so Galileo had no hard
proof; his insights into relative versus absolute motion simply made it
more plausible that the world could be spinning without producing dramatic effects,
but didn't disprove the contrary hypothesis that the sun, moon, and
stars went around the earth every 24 hours. Furthermore, the Church was much more
liberal and enlightened than most people believe. It didn't (and still doesn't)
require a literal interpretation of the Bible, and one of the Church officials
involved in the Galileo affair wrote that ``the Bible tells us how to go to heaven,
not how the heavens go.'' In other words, religion and science should be separate.
The actual reason Galileo got in trouble is shrouded in mystery, since Italy in the
age of the Medicis was a secretive place where unscrupulous people might settle a score
with poison or a false accusation of heresy. What is certain is that Galileo's
satirical style of scientific writing made many enemies among the powerful Jesuit scholars
who were his intellectual opponents --- he compared one to a snake that doesn't know
its own back is broken. Galileo and the Pope were old friends, but someone started a rumor
that the stupid character Simplicio in Galileo's dialogs was really meant to represent the Pope.
It's also possible that the Church was far less upset by his
astronomical work than by his support for atomism, the idea that matter is made of atoms.
Some theologians perceived atomism
as contradicting transubstantiation, the Church's doctrine that
the holy bread and wine are literally transformed into the flesh and blood of
Christ by the priest's blessing.


\end{envsubsection}

\mysection{Momentum}\index{momentum!conservation of}
\begin{envsubsection}{Conservation of momentum}\index{momentum!conservation of}\index{conservation!of momentum}
Let's return to the impossible story of Jen Yu and Iron Arm Lu on page \pageref{tiger}.
For simplicity, we'll model them as two identical, featureless pool balls,
\figref{impossible-a}. This may seem like a drastic simplification, but even a collision
between two human bodies is really just a series of many collisions between atoms.
The film shows a series of instants in time, viewed from overhead. The light-colored ball comes in, hits the
darker ball, and rebounds. It seems strange that the dark ball has such a big
effect on the light ball without experiencing any consequences itself, but how can we show that
this is really impossible?

\widefigsidecaption{impossible-a}{How can we prove that this collision is impossible?}

We can show it's impossible by looking at it in a different frame of reference,
\figref{impossible-b}. This camera follows the light ball on its way in, so in this
frame the incoming light ball appears motionless. (If you ever get hauled into court
on an assault charge for hitting someone, try this defense: ``Your honor, in my fist's
frame of reference, it was his face that assaulted my knuckles!'') After the collision,
the camera keeps moving in the same direction, because if it didn't, it wouldn't be
showing us an inertial frame of reference. To help convince yourself that figures
\figref{impossible-a} and \figref{impossible-b} represent the same motion seen in two
different frames, note that both films agree on the distances between the balls at
each instant. After the collision, frame \figref{impossible-b} shows the light ball moving twice as fast as the dark
ball; an observer who prefers frame \figref{impossible-a} explains this
by saying that the camera that produced film \figref{impossible-b} was moving one
way, while the ball was moving the opposite way.

\widefigsidecaption{impossible-b}{The collision of figure \figref{impossible-a} is viewed
in a different frame of reference.}

Figures \figref{impossible-a} and \figref{impossible-b} record the same
events, so if one is impossible, the other is too. But figure \figref{impossible-b}
is definitely impossible, because it violates conservation of energy. Before
the collision, the only kinetic energy is the dark ball's. After the collision, light
ball suddenly has some energy, but where did that energy come from? It can only have
come from the dark ball. The dark ball should then have lost some energy, which it hasn't,
since it's moving at the same speed as before.

Figure \figref{possible} shows what really does happen. This kind of behavior is
familiar to anyone who plays pool. In a head-on collision, the incoming ball stops
dead, and the target ball takes all its energy and flies away. In \subfigref{possible}{1},
the light ball hits the dark ball. In \subfigref{possible}{2}, the camera is initially following
the light ball; in this frame of reference, the dark ball hits the light one (``Judge,
his face hit my knuckles!''). The frame of reference shown in
\subfigref{possible}{3} is particularly interesting. Here the
camera always stays at the midpoint between the two balls. This is called the
center-of-mass frame of reference.

\widefigsidecaption{possible}{This is what really happens. Three films represent the same
collision viewed in three different frames of reference. Energy is conserved in
all three frames.}

\selfcheck{markcm}{In each picture in figure \subfigref{possible}{1}, mark an x at
the point half-way in between the two balls. This series of five x's represents the motion of the
camera that was used to make the bottom film. How fast is the camera moving? Does it represent
an inertial frame of reference?}

What's special about the center-of-mass frame is its symmetry. In this frame, both balls
have the same initial speed. Since they start out with the same speed, and they have the
same mass, there's no reason for them to behave differently from each other after the collision.
By symmetry, if the light ball feels a certain effect from the dark ball, the dark ball must feel
the same effect from the light ball. 

This is exactly like the rules of accounting. Let's say two big corporations are
doing business with each other. If Glutcorp pays a million dollars to Slushco,
two things happen: Glutcorp's bank account goes down by a million dollars, and
Slushco's rises by the same amount. The two companies' books have to show
transactions on the same date that are equal in size, but one is positive
(a payment) and one is negative. What if Glutcorp records $-1,000,000$ dollars,
but Slushco's books say $+920,000$? This indicates that a law has been broken;
the accountants are going to call the police and start looking for the employee who's
driving a new 80,000-dollar Jaguar. Money is supposed to be conserved.

In figure \figref{possible}, let's define velocities as positive if the motion
is toward the top of the page. In figure \subfigref{possible}{1}
let's say the incoming light ball's velocity is 1 m/s.

\newcommand{\half}{\ensuremath{0.5}}
\velocitytable{0}{1}{$+1$}{1}{0}{$-1$}{darkball}{lightball}

\noindent The books balance. The light ball's payment, $-1$, matches the dark ball's
receipt, $+1$. Everything also works out fine in the center of mass frame, \subfigref{possible}{3}:

\velocitytable{$-\half$}{$+\half$}{$+1$}{$+\half$}{$-\half$}{$-1$}{darkball}{lightball}

\selfcheck{thirdframepcons}{Make a similar table for figure \subfigref{possible}{2}. What
do you notice about the change in velocity when you compare the three tables?}

Accounting works because money is conserved. Apparently, something is also conserved when
the balls collide. We call it momentum. Momentum is not the same as velocity, because
conserved quantities have to be additive. Our pool balls are like identical atoms, but
atoms can be stuck together to form molecules, people, and planets. Because conservation
laws work by addition, two atoms stuck together and moving at a certain velocity must have
double the momentum that a single atom would have had. We therefore define momentum as
velocity multiplied by mass.\index{Noether's theorem!for momentum}

\begin{important}[Conservation of momentum]
The quantity defined by
\begin{equation*}
	\text{momentum} = mv
\end{equation*}
is conserved.
\end{important}


This is our second example of Noether's theorem:\nopagebreak

\begin{noethertable}
time translation & \noetherimplies & mass-energy \\
space translation & \noetherimplies & momentum \\
\end{noethertable}

\vfill


\begin{eg}{Conservation of momentum for pool balls}
\egquestion
Is momentum conserved in figure \subfigref{possible}{1}?

\eganswer
We have to check whether the total initial momentum is the same as the total
final momentum.
\begin{multline*}
	\text{dark ball's initial momentum} + \text{light ball's initial momentum}\\
	=? \\
       \text{dark ball's final momentum} + \text{light ball's final momentum}
\end{multline*}
Yes, momentum was conserved:
\begin{equation*}
	0+mv = mv+0
\end{equation*}
\end{eg}

\marg{\fig{skatersmomentum}{example \ref{eg:skatersmomentum}}}
\begin{eg}{Figure skaters push off from each other}\label{eg:skatersmomentum}
Let's revisit the figure skaters from the example on page
\pageref{original-skaters}. I argued there that if they had equal masses,
then mirror symmetry would imply that they moved off with equal speeds
in opposite directions. Let's check that this is consistent with conservation
of momentum:
\begin{multline*}
	\text{left skater's initial momentum} + \text{right skater's initial momentum}\\
	=? \\
       \text{left skater's final momentum} + \text{right skater's final momentum}
\end{multline*}
Momentum was conserved:
\begin{equation*}
	0+0 = m\times(-v)+mv
\end{equation*}
This is an interesting example, because if these had been pool balls instead
of people, we would have accused them of violating conservation of energy. Initially
there was zero kinetic energy, and at the end there wasn't zero. (Note that the energies
at the end don't cancel, because kinetic energy is always positive, regardless
of direction.) The mystery is resolved because they're people, not pool balls. They
both ate food, and they therefore have chemical energy inside their bodies:
\begin{equation*}
	\text{food energy} \rightarrow \text{kinetic energy} + \text{kinetic energy} + \text{heat}
\end{equation*}
\end{eg}

\begin{eg}{Unequal masses}
\egquestion Suppose the skaters have unequal masses: 50 kg for the one on the left, and 55 kg for
the other. The more massive skater, on the right, moves off at 1.0 m/s. How fast does
the less massive skater go?

\eganswer Their momenta (plural of momentum) have to be the same amount, but with opposite signs.
The less massive skater  must have a greater velocity if her momentum is going to be
as much as the more massive one's.
\begin{align*}
	0+0 &= (50\ \kgunit)(-v)+(55\ \kgunit)(1.0\ \munit/\sunit)\\
	(50\ \kgunit)(v) &= (55\ \kgunit)(1.0\ \munit/\sunit)\\
	v &= \frac{(55\ \kgunit)}{50\ \kgunit}(1.0\ \munit/\sunit)\\
	  &= 1.1\ \munit/\sunit
\end{align*}
\end{eg}
\end{envsubsection}
\begin{envsubsection}{Momentum compared to kinetic energy}\index{momentum!compared to kinetic energy}\index{kinetic energy!compared to momentum}
Momentum and kinetic energy are both measures of the
amount of motion, and a sideshow in the Newton-Leibniz
controversy over who invented calculus was an argument over
which quantity was the ``true'' measure of
motion. The modern student can certainly be excused for
wondering why we need both quantities, when their complementary
nature was not evident to the greatest minds of the 1700's.
The following table highlights their differences.

\noindent\begin{tabular}{|p{52mm}|p{52mm}|}
\hline
\textbf{Kinetic energy\ldots}	& \textbf{Momentum\ldots}\\
\hline
has no direction in space. & has a direction in space.\\
\hline
is always positive, and cannot cancel out. & cancels with momentum in the opposite direction.\\
\hline
can be traded for forms of energy that do not involve motion. KE is not
a conserved quantity by itself. & is always conserved.\\
\hline
is quadrupled if the velocity is doubled (lab \ref{ch:energy}\ref{lab:energy}). & is doubled if the velocity is doubled.\\
\hline
\end{tabular}


Here are some examples that show the different behaviors
of the two quantities.

\marg{\fig{coinmomentum}{example \ref{eg:coinmomentum}}
\spacebetweenfigs
\fig{earthmoondivorce}{example \ref{eg:earthmoondivorce}}}
\begin{eg}{A spinning coin}\label{eg:coinmomentum}
A spinning coin has zero total momentum, because for every
moving point, there is another point on the opposite side
that cancels its momentum. It does, however, have kinetic energy.
\end{eg}

\begin{eg}{Momentum and kinetic energy in firing a rifle}
The rifle and bullet have zero momentum and zero kinetic
energy to start with. When the trigger is pulled, the bullet
gains some momentum in the forward direction, but this is
canceled by the rifle's backward momentum, so the total
momentum is still zero. The kinetic energies of the gun and
bullet are both positive numbers, however, and do not
cancel. The total kinetic energy is allowed to increase,
because both objects' kinetic energies are destined to be
dissipated as heat --- the gun's ``backward'' kinetic energy
does not refrigerate the shooter's shoulder!
\end{eg}

\begin{eg}{The wobbly earth}
As the moon completes half a circle around the earth, its
motion reverses direction. This does not involve any change
in kinetic energy, because the moon doesn't speed up or slow
down, nor is there any change in gravitational energy, because
the moon stays at the same distance from the earth.\footnote{Actually these
statements are both only approximately true. The moon's orbit isn't exactly a circle.}
 The reversed velocity 
does, however, imply a reversed momentum, so
conservation of momentum 
tells us that the earth must also change its momentum. In
fact, the earth wobbles in a little ``orbit'' about a point
below its surface on the line connecting it and the moon.
The two bodies' momenta always point in opposite
directions and cancel each other out.
\end{eg}

\begin{eg}{The earth and moon get a divorce}\label{eg:earthmoondivorce}
Why can't the moon suddenly decide to fly off one way and
the earth the other way? It is not forbidden by conservation
of momentum, because the moon's newly acquired momentum in
one direction could be canceled out by the change in the
momentum of the earth, supposing the earth headed the
opposite direction at the appropriate, slower speed. The
catastrophe is forbidden by conservation of energy, because
both their kinetic energies would have increased greatly.
\end{eg}

\begin{eg}{Momentum and kinetic energy of a glacier}
A cubic-kilometer glacier would have a mass of about
$10^{12}$  kg --- 1 followed by 12 zeroes. If it moves at a speed of $0.00001$  m/s,
then its momentum\footnote{The units of this number are
kilograms times meters per second, or $\kgunit\unitdot\munit/\sunit$.} is $10,000,000$. 
This is the kind of
heroic-scale result we expect, perhaps the equivalent of the
space shuttle taking off, or all the cars in LA driving in
the same direction at freeway speed. Its kinetic energy,
however, is only 50 joules, the equivalent of the calories
contained in a poppy seed or the energy in a drop of
gasoline too small to be seen without a microscope. The
surprisingly small kinetic energy is because kinetic energy
is proportional to the square of the velocity, and the
square of a small number is an even smaller number.
\end{eg}
\end{envsubsection}
\begin{envsubsection}{Force}
\subsubsection{Definition of force}\index{force!definition}
When momentum is being transferred, we refer to the rate of transfer as the force.\footnote{This
definition is known as Newton's second law of motion. Don't memorize that!} The metric
unit of force is the newton (N).\index{force!unit}\index{newton (unit)}
The relationship between force and momentum is like the relationship between
power and energy, or the one between your cash flow and your bank balance:

\begin{tabular}{|p{20mm}p{20mm}|p{20mm}p{20mm}|}
\hline
\multicolumn{2}{|p{40mm}|}{\textbf{conserved quantity}} &\multicolumn{2}{|p{40mm}|}{\textbf{rate of transfer}} \\
\hline
\textbf{name} & \textbf{units} & \textbf{name} & \textbf{units} \\
energy & joules (J) & power & watts (W) \\
momentum & $\kgunit\unitdot\munit/\sunit$ & force & newtons (N)\\
\hline
\end{tabular}

\begin{eg}{A bullet}\label{eg:bulletforce}
\egquestion A bullet emerges from a gun with a momentum of 1.0 units,\footnote{metric units of $\kgunit\unitdot\munit/\sunit$} after
having been acted on for 0.01 seconds by the force of the gases from the explosion of the gunpowder.
What was the force on the bullet?

\eganswer 
The force is\footnote{This is really only an estimate of the average force over the time it takes for the bullet
to move down the barrel. The force probably starts out stronger than this, and then gets weaker because the gases
expand and cool.}
\begin{equation*}
	\frac{1.0}{0.01} = 100\ \text{newtons} \qquad .
\end{equation*}
\end{eg}

There's no new physics happening here, just a definition of the word ``force.'' Definitions are neither right nor
wrong, and just because the Chinese call it \raisebox{-0.2mm}{\anonymousinlinefig{li}} instead, that doesn't mean they're
incorrect. But when Isaac Newton first started using the term ``force'' according to this
technical definition, people already had some definite ideas about what the word meant.

\subsubsection{Forces occur in equal-strength pairs}\index{force!pairs}\index{Newton, Isaac!third law}
In some cases Newton's definition matches our intuition. In example \ref{eg:bulletforce}, we divided by a small
time, and the result was a big force; this is intuitively reasonable, since we expect the force on the bullet to
be strong.
In other situations, however, our intuition rebels against reality.

\begin{eg}{Extra protein}\label{eg:extraprotein}
\egquestion
While riding my bike fast down a steep hill, I pass through a cloud of gnats, and one of them goes into my
mouth. Compare my force on the gnat to the gnat's force on me.

\eganswer
Momentum is conserved, so the momentum gained by the gnat equals the momentum lost by me. Momentum
conservation holds true at every instant over the fraction of a second that it takes for the collision to
happen. The rate of transfer
of momentum out of me must equal the rate of transfer into the gnat. Our forces on each other have the
same strength, but they're in opposite directions.
\end{eg}

\noindent Most people would be willing to believe that the momentum gained by the gnat is the same as the
momentum lost by me, but they would not believe that the forces are the same strength. Nevertheless, 
the second statement follows from the first merely as a matter of definition. 
Whenever two objects, A and B, interact,\label{thirdlaw}
A's force on B is the same strength as B's force on A, and the forces are in opposite directions.\footnote{This
is called Newton's third law. Don't memorize that name!}
\begin{equation*}
	\text{(A on B)} = -\text{(B on A)}
\end{equation*}
Using the metaphor of money, suppose Alice and Bob are adrift in a life raft, and pass the time by playing
poker. Money is conserved, so if they count all the money in the boat every night, they should always come
up with the same total. A completely equivalent statement is that their cash flows are equal and opposite.
If Alice is winning five dollars per hour, then Bob must be losing at the same rate.

%people may believe cons of momentum but disbelieve when stated in force terms; two Israeli psychologists
%kahneman tversky

\margup{-40mm}{\fig{bartab}{It doesn't make sense to add his debts to her assets.}
\spacebetweenfigs
\fig{squeezescale}{I squeeze the bathroom scale.
It does make sense to add my fingers' force to my thumbs', because they both act on the same object --- the scale.}}
This statement about equal forces in opposite directions implies to many students a kind of mystical
principle of equilibrium that explains why things don't move. That would be a useless principle, since it
would be violated every time something moved.\footnote{During the Scopes monkey trial, William Jennings Bryan
claimed that every time he picked his foot up off the ground, he was violating the law of gravity.} The ice skaters
of figure \figref{skatersmomentum} on page \pageref{fig:skatersmomentum} make forces on each
other, and their forces are equal in strength and opposite in direction. That doesn't mean
they won't move. They'll both move --- in opposite directions.

The fallacy comes from trying to
add things that it doesn't make sense to add, as suggested by the cartoon in figure \figref{bartab}.
We only add forces that are acting on the same object. It doesn't make sense to say that the skaters'
forces on each other add up to zero, because it doesn't make sense to add them. One is a force on the
left-hand skater, and the other is a force on the right-hand skater.

In figure \figref{squeezescale}, my fingers'
force and my thumbs' force are both acting on the bathroom scale.
It does make sense to add these forces, and they may possibly add up to zero, but
that's not guaranteed by the laws of physics. 
If I throw the scale at you, my thumbs' force is stronger that my fingers', and the forces no longer cancel:
\begin{equation*}
	\text{(fingers on scale)} \ne -\text{(thumbs on scale)} \qquad .
\end{equation*}
What's guaranteed by conservation of momentum is a whole different relationship:
\begin{align*}
	\text{(fingers on scale)} &= -\text{(scale on fingers)} \\
	\text{(thumbs on scale)} &= -\text{(scale on thumbs)} \\
\end{align*}
\subsubsection{The force of gravity}\index{force!of gravity}\index{gravity!force of}
How much force does gravity make on an object? From everyday experience, we know
that this force is proportional to the object's mass.\footnote{This follows from the
additivity of forces.} Let's find the force on
a one-kilogram object. If we release this object from rest, then after it has
fallen one meter, its kinetic energy equals the strength of the gravitational field,
\begin{equation*}
10\ \text{joules per kilogram per meter}\times1\ \text{kilogram}\times1\ \text{meter} = 10\ \text{joules} \qquad .
\end{equation*}
Using the equation for kinetic energy from lab \ref{ch:energy}\ref{lab:energy} and doing a little simple algebra, we find that
its final velocity is 4.4 m/s. It starts from 0 m/s, and ends at 4.4 m/s, so
its average velocity is 2.2 m/s, and the time takes to fall one meter is therefore
(1 m)/(2.2 m/s)=0.44 seconds. Its final momentum is 4.4 units, so the force on it was
evidently
\begin{equation*}
	\frac{4.4}{0.44} = 10\ \text{newtons} \qquad .
\end{equation*}
This is like one of those card tricks where the magician makes you go through a bunch
of steps so that you end up revealing the card you had chosen --- the result is just
equal to the gravitational field, 10, but in units of newtons! If algebra makes you feel
warm and fuzzy, you may want to replay the derivation using symbols and convince yourself
that it had to come out that way. If not, then I hope the numerical result is enough
to convince you of the general fact that the force of gravity on a one-kilogram mass
equals $g$. For masses other than one kilogram, we have the handy-dandy result that
\begin{equation*}
	(\text{force of gravity on a mass $m$}) = mg \qquad .
\end{equation*}
In other words, $g$ can be interpreted not just as the gravitational energy per kilogram
per meter of height, but also as the gravitational force per kilogram.
\end{envsubsection}
%
\begin{envsubsection}{Motion in two dimensions}
\subsubsection{Projectile motion}
Galileo was an innovator in more than one way. He was arguably the inventor of open-source
software: he invented a mechanical calculating device for certain engineering applications,
and rather than keeping the device's design secret as his competitors did, he made it
public, but charged students for lessons in how to use it. Not only that, but he was
the first physicist to make money as a military consultant. Galileo understood projectiles
better than anyone else, because he understood the principle of inertia. Even if you're not
planning on a career involving artillery, projectile motion is a good thing to learn about
because it's an example of how to handle motion in two or three dimensions. \index{Galileo!projectile motion}\index{projectile motion}

\marg{%
\fig{dropball}{A ball is falling (or rising).}
\spacebetweenfigs
\fig{parabolaball}{The same ball is viewed in a frame of reference that is moving
horizontally.}
\spacebetweenfigs
\fig{hose}{The drops of water travel in parabolic arcs.}
}
Figure \figref{dropball} shows a ball in the process of falling --- or rising,
it really doesn't matter which. Let's say the ball has a mass of one kilogram, each square in
the grid is 10 meters on a side, and the positions of the ball are shown at time intervals
of one second. The earth's gravitational force on the ball is 10 newtons, so with each
second, the ball's momentum increases by 10 units, and its speed also increases by
10 m/s. The ball falls 10 m in the first second, 20 m in the next second, and so on.

\selfcheck{heavierball}{What would happen if the ball's mass was 2 kilograms?}

Now let's look at the ball's motion in a new frame of reference, \figref{parabolaball},
which is moving at 10 meters per second to the left compared to the frame of reference
used in figure \figref{dropball}. An observer in this frame of reference sees the ball
as moving to the right by 10 meters every second. The ball traces an arc of a specific
mathematical type called a parabola:\index{parabola}

\hspace{15mm}\parbox{80mm}{\begin{tabular}{p{60mm}}
1 step over and 1 step down\\
1 step over and 2 steps down\\
1 step over and 3 steps down\\
1 step over and 4 steps down\\
\ldots
\end{tabular}}

It doesn't matter which frame of reference is the ``real'' one. Both diagrams
show the possible motion of a projectile. The interesting point here is that the vertical
force of gravity has no effect on the horizontal motion, and the horizontal motion also
has no effect on what happens in the vertical motion. The two are completely independent.
If the sun is directly overhead, the motion of the ball's shadow on the ground seems perfectly
natural: there are no horizontal forces, so it either sits still or moves at constant
velocity. (Zero force means zero rate of transfer of momentum.) The same is true if we shine
a light from one side and cast the ball's shadow on the wall. Both shadows obey the laws
of physics.

\begin{eg}{The moon}\label{eg:moonorbit}
In example \ref{eg:newtonsapple} on page \pageref{eg:newtonsapple}, I promised an
explanation of how Newton knew that the gravitational field experienced by the moon
due to the earth was 1/3600 of the one we feel here on the earth's surface.
The radius of the moon's orbit had been known since ancient times
(see page \pageref{hipparchusmoondistance}), 
so Newton knew its speed to be 1,100 m/s (expressed in modern units).
If the earth's gravity wasn't acting on the moon, the moon would fly off straight,
along the straight line shown in figure \figref{moonorbit}, and it
would cover 1,100 meters in one second. We observe instead
that it travels the arc of a circle centered on the earth. Straightforward
geometry shows that the amount by which the arc drops below the straight line
is 1.6 millimeters. Near the surface of the earth, an object falls 5 meters
in one second,\footnote{Its initial speed is 0, and its final speed is 10 m/s, so its
average speed is 5 m/s over the first second of falling.} which is
indeed about 3600 times greater than 1.6 millimeters.

\vfill\pagebreak[4]

The tricky part about this argument is that although I said the path of a projectile
was a parabola, in this example it's a circle. What's going on here? What's different
here is that as the moon moves 1,100 meters, it changes its position relative to the
earth, so down is now in a new direction. We'll discuss circular motion more
carefully soon, but in this example, it really doesn't matter. The curvature of the
arc is so gentle that a parabola and a circle would appear almost identical. (Actually
the curvature is so gentle --- 1.6 millimeters over a distance of 1,100 meters! --- that
if I had drawn the figure to scale, you wouldn't have even been able to tell that it
wasn't straight.)\index{moon!orbit}

As an interesting historical note, Newton claimed that he first did this
calculation while confined to his family's farm during the plague of 1666, and
found the results to ``answer pretty nearly.''
His notebooks, however, show that although he did the calculation on that date,
the result didn't quite come out quite right, and he became uncertain about
whether his theory of gravity was correct as it stood or needed to be modified.
Not until 1675 did he learn of more accurate astronomical data, which convinced
him that his theory didn't need to be tinkered with.
It appears that he rewrote his own life story a little bit
in order to make it appear that his work was more advanced at an earlier
date, which would have helped him in his dispute with Leibniz over priority
in the invention of calculus.
\end{eg}

\margup{-104mm}{%
\fig{moonorbit}{example \ref{eg:moonorbit}}
\vspace{50mm}
\fig{memoryofmotion}{The memory of motion: the default would be for the ball to continue
doing what it was already doing. The force of gravity makes it deviate downward, ending
up one square below the default.}
}
\subsubsection{The memory of motion}
There's another useful way of thinking about motion
along a curve.
The weak principle of inertia tells us that in the absence of a force, an
object will continue moving in the same speed and in the same direction. One of my students
invented a wonderful phrase for this: the memory of motion. 
Over the first second of its motion, the ball in figure
\figref{memoryofmotion} moved 1 square over and 1 square down, which is 10 meters and
10 meters. The default for the next one-second
interval would be to repeat this, ending up at the location marked with the 
first dashed circle. The earth's 10-newton gravitational force on the ball, however, changes
the vertical part of the ball's momentum by 10 units. The ball actually ends up 10 meters
(1 square) below the default.

\vfill\pagebreak[4]

\subsubsection{Circular motion}\index{circular motion}
Figure \figref{hammerpolygons} shows how to apply the memory-of-motion idea to circular motion. It
should convince you that only an inward force is needed to produce circular motion.
One of the reasons Newton was the first to make any progress in analyzing the motion of
the planets around the sun was that his contemporaries were confused on this point. Most
of them thought that in addition to an attraction from the sun, a second, forward force
must exist on the planets, to keep them from slowing down. This is incorrect Aristotelian
thinking; objects don't naturally slow down. Car 1 in figure \figref{carincircle} only
needs a forward force in order to cancel out the backward force of friction; the total force on
it is zero. Similarly, the forward and backward forces on car 2 are canceling out, and the
only force left over is the inward one. There's no friction in the vacuum of outer space,
so if car 2 was a planet, the backward force wouldn't exist; the forward force wouldn't
exist either, because the only force would be the force of the sun's gravity.

\widefigsidecaption{hammerpolygons}{A large number of gentle taps gives a good approximation to circular
motion. A steady inward force would give exactly circular motion.}%

\marg{
\fig{carincircle}{The forces on car 1 cancel, and the total force on it is zero. The forward
and backward forces on car 2 also cancel. Only the inward force remains.}}
On page \pageref{pickuptrucklinear} we saw that when we tried to visualize motion in
a noninertial frame of reference, we experienced the vivid illusion of a violation of
the laws of physics. In circular motion, this temptation is especially strong.
Frame \subfigref{pickuptruckcircular}{1}, attached to the turning truck, is noninertial, because 
it changes the direction of its motion. The ball violates the weak principle of inertia
by accelerating from rest for no apparent reason. Is there some mysterious outward force
that is slamming the ball into the side of the truck's bed? No. By analyzing everything in
a proper inertial frame of reference, \subfigref{pickuptruckcircular}{2}, we see that it's
the truck that swerves and hits the ball. That makes sense, because the truck is interacting
with the asphalt.\index{frame of reference!noninertial}

\widefig{pickuptruckcircular}{A bowling ball is in the back of a pickup truck turning left.
The motion is viewed first in a frame that turns along with the truck, 1, and then in
an inertial frame, 2.}

\end{envsubsection}

% Because a floating fig ends up on this page, have to convince LaTeX to make a page break
% before the hw section:
\raisebox{0mm}[100mm][0mm]{\quad}

%===============================================================================
%===============================================================================

% blocks on butcher paper works well as galilean symmetry
% lab: fill in cases where KE isn't the only form of energy, e.g. carts hit and stick with velcro
% hw: redo addition of vel and pool ball analysis in opposite coordinate system
% hw: fill in details of moon orbit example
% hw: two projectiles on grid, one going at theta, the other at 90-theta; show ranges are equal
% activity: George's hammer hitting bowling ball, parabola; cf circle?
% hw: analyze completely inelastic collision; need this as preparation for treatment of mass
%    - energy equivalence in next chapter
\begin{hwsection}

\begin{hw}{beer}
The beer bottle shown in the figure is resting on a table in the dining car of a train.
The tracks are straight and level. What can you tell about the motion of the
train? Can you tell whether the train is currently moving forward, moving backward,
or standing still? Can you tell what the train's speed is?
\end{hw}

\begin{hw}{flag-in-balloon}
You're a passenger in the open basket hanging under a hot-air balloon.
The balloon is being carried along by the wind at a constant velocity.
If you're holding a flag in your hand, will the flag wave? If so,
which way? (Based on a question from PSSC Physics.)
\end{hw}

\margup{-45mm}{\fig{hw-beer}{Problem \ref{hw:beer}.}
\spacebetweenfigs
\fig{hw-flag-in-balloon}{Problem \ref{hw:flag-in-balloon}}}
\begin{hw}{foot-off-gas}
Driving along in your car, you take your foot off the gas, and your speedometer
shows a reduction in speed. Describe an inertial frame in which your car
was \emph{speeding up} during that same period of time.
\end{hw}

\begin{hw}{airsettlesonfloor}
If all the air molecules in the room settled down in a thin film on the floor,
would that violate conservation of momentum as well as conservation of energy?
\end{hw}

\begin{hw}{bulletthroughbook}
A bullet flies through the air, passes through a
paperback book, and then continues to fly through the air
beyond the book. When is there a force? When is there energy?
\end{hw}

\begin{hw}{findthetop}
(a) Continue figure \figref{parabolaball} farther to the left, and do the
same for the numerical table in the text.\hwendpart
(b) Sketch a smooth curve (a parabola) through all the points on the figure, including all
the ones from the original figure and all the ones you added. Identify the very top of
its arc.\hwendpart
(c) Now consider figure \figref{dropball}. Is the highest point shown in the figure the
top of the ball's up-down path? Explain by comparing with your results from parts a and b.
\end{hw}

\begin{hw}{interpret-film-time-not-space}
Criticize the following statement about the top panel of figure \figref{possible} on page \pageref{fig:possible}:
In the first few pictures, the light ball is moving up and to the right, while the dark ball moves directly to the
right.
\end{hw}


\begin{hw}{film-dropping}
The figure on page \pageref{hw-film-dropping-fig}
shows a ball dropping to the surface of the earth. Energy is conserved: over the
whole course of the film, the gravitational energy between the ball and the earth decreases
by 1 joule, while the ball's kinetic energy increases by 1 joule.\hwendpart
(a) How can you tell \emph{directly from the figure}
 that the ball's speed isn't staying the same?\hwendpart
(b) Draw what the film would look like if the camera was following the ball.\hwendpart
(c) Explain how you can tell that in this new frame of reference, energy is not conserved.\hwendpart
(d) Does this violate the strong principle of inertia? Isn't every frame of reference supposed to be equally valid?
\end{hw}

\widefigsidecaption{film-dropping}{Problem \ref{hw:film-dropping}.}\label{hw-film-dropping-fig}


\end{hwsection}


%========================================== labs ===============================================
%--------------------------------- interactions lab --------------------------------------

\begin{lab}{Interactions}

\apparatus
\equipn{single neodymium magnet}{1/group}\\
\equipn{triple neodymium magnet}{1/group}\\
\equip{compass}\\
\equipn{triple-arm balance}{2}\\
\equip{clamp and 50-cm rod for holding balance up}\\
\equip{string}\\
\equip{tape}\\
\equip{scissors}\\
\equip{heavy-duty spring scales}\\
\equip{rubber stoppers}

\goal{Form hypotheses about forces and test them.}\index{Aristotle}

\labintroduction

Why does a rock fall if you drop it?  The ancient Greek
philosopher Aristotle theorized that it was because the rock
was trying to get to its natural place, in contact with the
earth.  Why does a ball roll if you push it? Aristotle would
say that only living things have the ability to move of
their own volition, so the ball can only move if you give
motion to it.  Aristotle's explanations were accepted by
Arabs and Europeans for two thousand years, but beginning in
the Renaissance, his ideas began to be modified drastically.
 Today, Aristotelian physics is discussed mainly by physics
teachers, who often find that their students intuitively
believe the Aristotelian world-view and strongly resist the
completely different version of physics that is now
considered correct.  It is not uncommon for a student to
begin a physics exam and then pause to ask the instructor,
``Do you want us to answer these questions the way you told
us was true, or the way we really think it works?''  The
idea of this lab is to make observations of objects, mostly
magnets, pushing and pulling on each other, and to figure
out some of the corrections that need to be made to
Aristotelian physics.

Some people might say that it's just a matter of definitions
or semantics whether Aristotle is correct or not.  Is
Aristotle's theory even testable?  One testable feature of
the theory is its asymmetry.  The Aristotelian description
of the rock falling and the ball being pushed outlines two
relationships involving four objects:

\labfig{lab-force-aristotle}

According to Aristotle, there are asymmetries involved in both situations.

(1) The earth's role is not interchangeable with that of the
rock.  The earth functions only as a place where the rock
tends to go, while the rock is an object that moves from
one place to another.

(2) The hand's role is not analogous to the ball's.  The
hand is capable of motion all by itself, but the ball can't
move without receiving the ability to move from the hand.

If we do an experiment that shows these types of asymmetries,
then Aristotle's theory is supported.  If we find a more
symmetric situation, then there's something wrong with Aristotle's theory.

\labobservations

\labpart{ Comparing magnets' strengths}

To make an interesting hypothesis about what will happen in
part C, the main event of the lab, you'll need to know how
the top (single) and bottom (triple) magnets' strengths
compare. It would seem logical that the triple magnet would
be three times stronger than the single, but in this part
of the lab you're going to find out for sure.

\labfigcaption{lab-force-strength}{Orient your magnet this way, as
if it's rolling toward the compass from the north. With no magnet
nearby, the compass points to magnetic north (dashed arrow).
 The magnet deflects the compass to a new direction. }

One way of measuring the strength of a magnet is to place
the magnet to the north or south of the compass and see how
much it deflects (twists) the needle of a compass. You need
to test the magnets at equal distances from the compass,
which will produce two different angles.\footnote{There are two
reasons why it wouldn't make sense to find different distances
that produced the same angle. First, you don't know how the
strengths of the effect falls off with distance; it's not necessarily
true, for instance, that the magnetic field is half as strong at
twice the distance. Second, the point of this is to help you
interpret part C, and in part C, the triple magnet's distance
from the single magnet is the same as the single magnet's distance
from the first magnet.}
It's also important to get everything oriented
properly, as in the figure.\footnote{Although you don't yet know
enough about magnetism to be able to see from first
principles why it should be this way, you can easily
convince yourself empirically that other setups (e.g.
rotating the magnet 90 degrees) give results that are
inaccurate and hard to reproduce, because the compass acts ``fidgety.''}

Make sure
to take your data with the magnets far enough from the
compass that the deflection angle is fairly small (say 5 to
30\degunit).  If the magnet is close enough to the compass
to deflect it by a large angle, then the ratio of the angles
does not accurately represent the ratio of the magnets'
strengths. After all, just about any magnet is capable of
deflecting the compass in any direction if you bring it
close enough, but that doesn't mean that all magnets are
equally strong.

\labpart{ Qualitative observations of the interaction of two magnets}

Play around with the two magnets and see how they interact
with each other. Can one attract the other?  Can one repel
the other?  Can they act on each other simultaneously? Do
they need to be touching in order to do anything to each
other?   Can A act on B while at the same time B does
not act on A at all?  Can A pull B toward itself at the
same time that B pushes A away?  When holding one of the
heavier magnets, it may be difficult to feel when there is
any push or pull on it; you may wish to have one person hold
the magnet with her eyes closed while the other person moves
the other magnet closer and farther.

\labpart{ Measurement of interactions between two magnets}

Once you have your data from parts A and B, you are
ready to form a hypothesis about the following situation. 
Suppose we set up two balances as shown in the figure.  The
magnets are not touching.  The top magnet is hanging from a
hook underneath the pan, giving the same result as if it was
on top of the pan.  Make sure it is hanging under the
\emph{center} of the pan. You will want to make sure the
magnets are pulling on each other, not pushing each other
away, so that the top magnet will stay in one place.

\labfig{lab-force-two-scales}

The balances will not show the magnets' true masses, because
the magnets are exerting forces on each other.  The top
balance will read a higher number than it would without any
magnetic forces, and the bottom balance will have a lower
than normal reading.  The difference between each magnet's
true mass and the reading on the balance gives a measure of
how strongly the magnet is being pushed or pulled by the other magnet.

How do you think the amount of pushing or pulling experienced
by the two magnets will compare?  In other words, which
reading will change more, or will they change by the same
amount?  Write down a hypothesis;  you'll test this
hypothesis in part C of the lab. If you think the forces
will be unequal predict their ratio.

Discuss with your instructor your results from parts A and
B, your hypothesis about what will happen with the two
balances, and your plan for how you do error analysis. 

Now set up the experiment described above with two balances.
 Since we are interested in the changes in the scale
readings caused by the magnetic forces, you will need to
take a total of four scale readings: one pair with the
balances separated and one pair with the magnets close
together as shown in the figure above.

When the balances are together and the magnetic forces are
acting, it is not possible to get both balances to reach
equilibrium at the same time, because sliding the weights on
one balance can cause its magnet to move up or down, tipping
the other balance.  Therefore, while you take a reading from
one balance, you need to immobilize the other in the
horizontal position by taping its tip so it points
exactly at the zero mark.

You will also probably find that as you slide the weights,
the pointer swings suddenly to the opposite side, but you
can never get it to be stable in the middle (zero) position.
 Try bringing the pointer manually to the zero position and
then releasing it.  If it swings up, you're too low, and if
it swings down, you're too high.  Search for the dividing
line between the too-low region and the too-high region.

If the changes in the scale readings are very small (say a
few grams or less), you need to get the magnets closer
together.  It should be possible to get the scale readings
to change by large amounts (up to 10 or 20 g).

\labpart{ Measurement of interactions involving objects in contact}

You'll recall that Aristotle gave completely different
interpretations for situations where one object was in
contact with another, like the hand pushing the ball, and
situations involving objects not in contact with each other,
such as the rock falling down to the earth.  Your magnets
were not in contact with each other.  Now suppose we try the
situation shown above, with one person's hand exerting a
force on the other's.  All the forces involved are forces
between objects in contact, although the two people's hands
cannot be in direct contact because the spring scales have
to be inserted to measure how strongly each person is
pulling.  Suppose the two people do not make any special
arrangement in advance about how hard to pull.  How do you
think the readings on the two scales will compare?  Write
down a hypothesis, and discuss it with your instructor before continuing.

\labfig{lab-force-two-hands}

Now carry out the measurement shown in the figure.


\end{lab}

%--------------------------------- frames of reference lab --------------------------------------
\begin{lab}{Frames of Reference}
\apparatus
\equip{track and 2 carts}\\
\equip{2-meter piece of butcher paper}\\
\equip{wood blocks with hooks and felt pads}\\
\equip{string}\\
\equip{1-kg masses}\\
\equip{spring scales calibrated in newtons}

\labintroduction

The little girl in the photo on page \pageref{ch:symmetry} spins around, but to her, it seems
like the \emph{world} is spinning around \emph{her}. She has her own frame of reference,
which is different from that of someone standing on the ground. Likewise, you
may have had the experience of sitting in a train in a station when you suddenly notice that
the station has started to move!
The idea of this lab is to perform the same experiments in different frames of reference, and
see if the results come out different.

\labsection{Collisions}

First you'll do some experiments involving collisions between two
carts rolling on a track.

Try gently pressing the two carts together on the track. As they come close to each
 other, you'll feel them repelling each other! That's because they have magnets built
 into the ends. The magnets act like perfect springs. For instance, if you hold one cart
 firmly in place and let the other one roll at it, the incoming cart will bounce back at almost exactly
 the same speed. It's like a perfect superball. This is called an elastic collision.

You can also make collisions in which the carts will stick together rather than rebounding.
You can do this by letting the velcro ends hit each other instead of the magnet ends. 
This is known as an inelastic collision.

\labpart{Elastic collision, projectile's frame}
Set the carts up so that their magnet sides are facing each other.
Roll one cart, A, toward the other, B, coming from the left. Cart B
is initially at rest. Observe the results.

Now imagine the whole thing from a frame of reference that is initially moving
with cart A. In this frame, A is initially at rest, and B is hitting A.

The question now arises of how to define this frame of reference after the collision.
We could define it as the frame of reference of a bug holding on tight to cart A
the whole time. In this frame of reference, cart A is always at rest,
both before and after the collision. Think about the results of the collision you just did, and
imagine what it would look like to the bug. Write down the bug's description:

\newcommand{\smallspaceforwriting}{\_\_\_\_\_\_\_\_\_\_\_\_\_\_\_\_\_\_}
\newcommand{\onelineforwriting}{\_\_\_\_\_\_\_\_\_\_\_\_\_\_\_\_\_\_\_\_\_\_\_\_\_\_\_\_\_\_\_\_\_\_\_\_\_\_\_\_\_\_\_\_\_\_\_\_\_\_\_\_\_\_}
\newcommand{\spaceforwriting}{\onelineforwriting\\ \onelineforwriting}
\spaceforwriting

On the other hand, we could imagine that the scene is being viewed by a video camera
moving along another track parallel to the real track. The camera keeps on moving after
the collision --- let's call this the coasting frame of reference, because the camera
keeps on coasting along. In this frame of reference, cart A is not at rest after the collsion.
Write down a description of the collision as viewed in the coasting frame:

\spaceforwriting

OK, now observers using the bug's frame and the coasting frame agree on what carts A and
B are doing before the collision, but they disagree after the collision.
Let's start the whole thing going so that to you, standing on the floor, the motion of
the carts looks just like the descriptions you wrote above. This means that you have
to do different physical motions than you did before.

Do the actual results agree with the bug's description, or with the coasting camera's
description?

\onelineforwriting

\labpart{Inelastic collision, center of mass frame}

Now turn the carts around so their velcro sides are toward each other.
Send cart A toward cart B, with B initially at rest. After the collision,
the two carts move off together to the right. Estimating by eye, how do you
think their speed after the collision compares with cart A's speed before it hit B?

\onelineforwriting

Now imagine a coasting frame of reference that moves along with the two carts \emph{after}
the collision. After the collision it's moving at the speed you described,
and because it's a coasting frame of reference, it was also moving at that same
speed \emph{before} the collision. What would the collision have looked like
in this frame of reference?

\spaceforwriting

This frame of reference is called the
center of mass frame. It's a frame of reference in which the collision has mirror
symmetry.

Now act out the collision so that what you see before the collision, from your frame of reference standing
on the lab's floor, matches what you wrote above.

Do the results after the collision agree with the description in the coasting frame of reference?

\onelineforwriting

\labpart{Elastic collision, center of mass frame}

What would the collision from part A have looked like in the center of mass frame?

\spaceforwriting

Act this out so that the center of mass frame corresponds to the frame of the lab's
floor. Do the results match?

\labsection{Stop and Think}

Let's think about what you've learned so far about frames of reference. Discuss the following
questions with your partners.

1. Based on what you've done so far, does it seem like all frames are equally valid, or
are there some frames in which the laws of physics don't seem to be functioning normally?

2. Here's a way to get some more evidence about whether all frames are equally valid.
So far we've only been discussing how the motion of the \emph{carts} looks in various frames
of reference. But in many of these frames of reference, the track, the room, the table, and
your body are moving as well. Go back and consider the motion of these external objects
in all the frames of reference you've
tried out. In each one, consider whether the external objects obey the weak principle
of inertia (page \pageref{weak-principle-of-inertia}).\index{inertia!weak principle of}

\labsection{Force and Motion}

We haven't yet defined force formally. For now, think of it on an intuitive basis as a push
or a pull. A force can be relatively steady, like a person pushing a crate across the floor, or jerky, like the forces
in the collisions between the carts. The metric unit of force is the newton, and we can measure
forces using spring scales.

Suppose a person pushes a crate, sliding it across the floor at a certain speed, and then repeats the same
 thing but at a higher speed. This is essentially the situation you will act out in this exercise. What
 do you think is different about her force on the crate in the two situations? Discuss this with your group and
 write down your hypothesis:

\onelineforwriting

\labpart{Measurement of friction}

First you'll measure the amount of friction between the wood block and the
butcher paper when the wood and paper surfaces are slipping over each other.
It isn't the point of this lab to measure things about friction, but you'll need this
information in order to interpret your later results.
The idea is to attach a spring scale to the block and then slide the butcher paper
under the block while using the scale to keep the block from moving with it.
Put the block on the paper with the felt side down.
You'll need to
 put an extra two-kilogram mass on top of the block in order to increase the amount of friction.
It's a good idea to use long piece of string to attach the block to the spring scale, since
 otherwise one tends to pull at an angle instead of directly horizontally.

First measure the amount of friction force when sliding the butcher paper as slowly as possible: \smallspaceforwriting

Now measure the amount of friction force at a significantly higher speed, say 1
meter per second. (If you try to go too fast, the motion is jerky, and it is impossible to get an accurate
 reading.) \smallspaceforwriting

Discuss your results. Why are we justified in assuming that the string's force on the block (i.e., the scale
reading) is the same amount as the paper's frictional force on the block?

\labpart{Motion}

Now try the same thing, but with the block moving and the paper standing still. Try two different
speeds.

Do your results agree with your original hypothesis? If not, discuss what's going on. How does the
block ``know'' how fast to go? How does all of this relate to the main idea of this lab?

\end{lab}
%--------------------------------- momentum lab --------------------------------------
\newcommand{\fillinblank}{\_\_\_\_\_\_}
\begin{lab}{Conservation of Momentum}\index{momentum!conservation of!lab}

\apparatus
\equip{computer with Logger Pro software}\\
\equip{track and 2 carts}\\
\equip{1-kg weight}\\
\equip{masking tape}\\
\equip{2 force sensors with rubber corks}

\labsection{Qualitative Observations}
First you're going to observe some collisions between two carts and see how conservation of momentum plays out. If you really wanted to take numerical data, it would be a hassle, because momentum depends on mass and velocity, and there would be four different velocity numbers you'd have to measure: cart 1 before the collision, cart 1 after the collision, cart 2 before, and cart 2 after. To avoid all this complication, the first part of the lab will use only visual observations.

\labpart{Equal masses, target at rest, elastic collision}
Roll one cart toward the other. The target cart is initially at rest. Conservation of momentum reads like this,

        M x \fillinblank + M x \fillinblank =? M x \fillinblank + M x \fillinblank   ,

where the two blanks on the left stand for the two carts' velocities before the collision, and the two blanks on the right are for their velocities after the collision. All conservation laws work like this: the total amount of something remains the same. You don't have any real numbers, but just from eyeballing the collision, what seems to have happened? Let's just arbitrarily say that the mass of a cart is one unit, so that wherever it says ``M x'' in the equation, you're just multiplying by one. You also don't have any numerical values for the velocities, but suppose we say that the initial velocity of the incoming cart is one unit. Does it look like conservation of momentum was satisfied?

\labpart{Mirror symmetry}
Now reenact the collision from part A, but do everything as a mirror image. The roles of the target cart and incoming cart are reversed, and the direction of motion is also reversed.

        M x \fillinblank + M x \fillinblank =? M x \fillinblank + M x \fillinblank   

What happens now? Note that mathematically, we use positive and negative signs to indicate the direction of a velocity in one dimension.

\labpart{An explosion}
Now start with the carts held together, with their magnets repelling. As soon as you release them, they'll break contact and fly apart due to the repulsion of the magnets.

        M x \fillinblank + M x \fillinblank =? M x \fillinblank + M x \fillinblank   

Does momentum appear to have been conserved?

\labpart{Head-on collision}
Now try a collision in which the two carts head towards each other at equal speeds (meaning that one cart's initial velocity is positive, while the other's is negative).

        M x \fillinblank + M x \fillinblank =? M x \fillinblank + M x \fillinblank   

\labpart{Sticking}
Arrange a collision in which the carts will stick together rather than rebounding. You can do this by letting the velcro ends hit each other instead of the magnet ends. Make a collision in which the target is initially stationary.

         M x \fillinblank + M x \fillinblank =? M x \fillinblank + M x \fillinblank   

The collision is no longer perfectly springy. Did it seem to matter, or was conservation of momentum still valid?

\labpart{Hitting the end of the track}
One end of the track has magnets in it. Take one cart off the track entirely, and let the other cart roll all the way to the end of the track, where it will experience a repulsion from the fixed magnets built into the track. Was momentum conserved? Discuss this with your instructor.

\labpart{Unequal masses, elastic collision}
Now put a one kilogram mass on one of the carts, but leave the other cart the way it was.
Attach the mass to it securely using masking tape. 
Use the magnets to make the collision elastic, as in part A.
A bare cart has a mass of half a kilogram, so you've now tripled the mass of one cart. In terms of our silly (but convenient) mass units, we now have masses of one unit and three units for the two carts. Make the triple-mass cart hit the initially stationary one-mass-unit cart.

        3M x \fillinblank + M x \fillinblank =? 3M x \fillinblank + M x \fillinblank   

These velocities are harder to estimate by eye, but if you estimate numbers roughly, does it seem possible that momentum was conserved?

\labsection{Quantitative Observations}
Now we're going to explore the reasons why momentum always seems to be conserved.

Attach the force sensors to the carts, and put on the rubber
 stoppers. Make sure that the rubber stoppers are positioned
 sufficiently far out from the body of the cart so that they
 will not rub against the edge of the cart. Put the switch on
 the sensor in the +10 N position. Plug the sensors into the 
DIN1 and DIN2 ports on the interface box. Start up the Logger
 Pro software, and do File$>$Open$>$Probes \& Sensors$>$Force
 Sensors$>$Dual Range Forrce$>$2-10 N Dual Range. (Refer back to lab
lab \ref{ch:energy}\ref{lab:energy} on page
\pageref{lab:energy} for more detailed instructions and troubleshooting
information.)

Try collecting data while pushing and pulling on the rubber stopper. You should get a graph showing how the force went
 up and down over time. The sensor uses negative numbers (bottom half of the graph) for forces that squish the sensor,
 and positive numbers (top half) for forces that stretch it. Try both sensors, and make sure you understand what the red
 and blue traces on the graph are showing you.

\labpart{Slow acceleration}

Put the extra 1-kilogram weight on one of the carts. Put the cart on the track by itself, without the other cart. Try accelerating it from rest
 with a gentle, steady force from your finger.  You'll want to set the collection
time to a longer period than the default. Position the track so that you can walk all the way along its
length (not diagonally across the bench). Even after you hit the Collect button in Logger Pro, the software
won't actually start collecting data until it's triggered by a sufficiently strong force; squeeze on
one of the sensors to trigger the computer, and then go ahead and do the real experiment with the steady,
gently force.

What does the graph on the computer look like?

\labpart{Rapid acceleration}
 Now repeat H, but use a more rapid acceleration to bring the cart up to the same momentum. Sketch a comparison of the graphs from parts H and I:




Discuss with your instructor how this relates to momentum.

\labpart{Measuring the forces}
You are now going to reenact collision A, but don't do it yet! You'll let the carts' rubber corks bump into each other,
 and record the forces on the sensors. The carts will have equal mass, and both forces will be recorded simultaneously. Before you do it,
predict what you think the graphs will look like, and show your sketch to your instructor.

This relatively violent collision will produce large forces for short periods of time, so the 10-newton scale is no
longer appropriate. Switch the switches on the sensors to 50 N, and open the file 2-50 N Dual Range.

Now try it. You will notice by eye that the motion after the collision is a tiny bit different than it was with the magnets, but
 it's still pretty similar. Looking at the graphs, how do you explain the fact that one cart lost exactly as much momentum
 as the other one gained? Discuss this with your instructor before going on. In order to see the
graph clearly, you'll need to zoom in by clicking and dragging diagonally to draw a rectangular box around
it, and then clicking on the magnifying glass icon with a plus sign in it.

\labpart{Forces with unequal masses}
Now imagine -- but don't do it yet -- that you are going to reenact part G, with unequal masses. Sketch your prediction for the
 two graphs, and show your sketch to your instructor before you go on.

Now try it.

\labsection{The Wrap-Up}
Now let's try to wrap all of this up in a nice package with a bow on it.

What was the basic point of parts A-G?

Parts H and I?

Parts J and K?

How do parts A-G relate to parts J-K?

Discuss this with your instructor.

\end{lab}

%--------------------------------- angular momentum lab --------------------------------------
\begin{lab}{Conservation of Angular Momen\-tum}\index{angular momentum!conservation of!lab}\index{torque!lab}\label{lab:torque}
\apparatus
\equip{meter stick with a hole in the center}\\
\equip{fulcrum}\\
\equip{sliding weight holders}\\
\equip{weights with hooks}

\labintroduction

Why can't the coin in the figure spontaneously reverse the direction it's spinning?
We don't observe this to happen, and since everything not forbidden is mandatory, we
expect that there must be some conservation law that forbids it. But what is this
conservation law? It's not conservation of energy, because the coin would have the
same energy regardless of which way it was spinning. It's not conservation of momentum,
either, because whichever way it's spinning, its total momentum is zero. This is
evidence that there is some new conservation law, which we call conservation of
angular momentum. A mass moving in a straight line has momentum. A spinning mass
has angular momentum. A frisbee has both, since it spins as it sails through the air.

\labfig{coinmomentum}

Noether's theorem tells us that conservation laws come from symmetry. What symmetry
does conservation of angular momentum come from? Think of a gyroscope. Suppose you initially
started a gyroscope spinning, but then it spontaneously decided to twist around and
spin along some other axis, pointing in some mysterious direction in space. What's
so special about that direction in space, and why do gyroscopes want to point that way?
This \emph{doesn't} happen, because no direction in space is special; the laws of
physics are symmetric with respect to rotation. Experiments don't come out any different
if you turn the laboratory building around to face a different way. Here's a summary
of all the conservation laws we know about so far:

\begin{tabular}{|p{20mm}|p{20mm}|p{20mm}|}
\hline
\emph{symmetry} & \emph{conserved quantity} & \emph{rate of transfer} \\
\hline
time translation & energy & power \\
\hline
space translation & momentum & force \\
\hline
rotation & angular momentum & torque\\
\hline
\end{tabular}

You've probably noticed that force is usually easier to measure than momentum. The
same is true with torque and angular momentum: it's easier to measure the rate of
transfer than it is to measure the accumulated amount that's been transferred. Logically,
it doesn't really matter which end you approach it from. For instance, you can look at
your bank statement and see how the balance changes, or you can look at the list of
deposits and withdrawals; either one has all the information you need in order to find
out about the other. In this lab, you're actually going to figure out a workable definition
of torque, which logically is enough to pin down the definition of angular momentum as well.

\labfig{lab-angmom}

As shown in the figure, the apparatus is a kind of seesaw, which you'll be balancing in
various ways.

\labsection{Observations}

As a preliminary, we'd like the meter-stick to balance all by itself, with no weights or
weight holders at  all. Unfortunately, it's not possible to drill the hole exactly at the
center of the stick, and the stick may also be asymmetric, e.g., there may be a piece
of brass on one end. To deal with this problem, you can put two extra, empty weight holders on
the stick, close to the center, and move them around so that the stick balances as well as
possible.
Even so, you may not be able to get the stick to be perfectly stable, and that's OK.
If the hole is
a little bit below the center of the stick, then it's an unstable equilibrium, like trying
to balance a pencil on its tip. 
Just try to get as close as possible to balancing.

\labpart{Plus and minus signs}
Let's start out by putting equal
weights at equal distances from the fulcrum, one on each side. You will now have a total
of four weight holders on the stick, including the two empty ones used for the initial
balancing.

What rate of transfer of
angular momentum do you seem to have? This tells you what the total torque is. If the
two torques add up to this value, what does that tell you about the individual torques?

\labpart{Additivity}
Conservation laws are supposed to be additive, and we've already implicitly assumed this
in part A. Let's now test that assumption. In addition to equal weights \#1 and \#2 that are
already on the seesaw, add two more weights, \#3 and \#4. Weights \#3 and \#4 should be
equal to each other, but unequal to weights \#1 and \#2. Weights \#3 and \#4 should also be
placed symmetrically on either side of the fulcrum, but not at the same distance from the
fulcrum as \#1 and \#2.

Is the result what you'd expect if torque is additive?

\labpart{Distance from the axis}
Now change to two weights, one of which is different from the other.
What do you have to do in order to make them balance? 
\footnote{Note that in this setup, the effects of the weight holders themselves will
not automatically cancel out. You should weigh the holders themselves and add them
into your weights.}

Let $F$ be the force the weight is making, and
$d$ the distance from the axis. What have you learned about how torque depends on
$F$ and $d$?

\labpart{Does it really work?}
Now put on four or five different weights, all unequal, and all at different distances
from the axis. Once you get them balanced, compute the total torque. Does your definition
of torque work correctly here?

\end{lab}

%========================================== self-check answers ===============================================

\startscanswers{ch:momentum}
\scanshdr{moonviolatestranslation} No, it doesn't violate symmetry. Space-translation symmetry
only says that space itself has the same properties everywhere. It doesn't say that all regions
of space have the same stuff in them. The experiment on the earth comes out a certain way because
that region of space has a planet in it. The experiment on the moon comes out different because
that region of space has the moon in it. of the apparatus, which you forgot to take with you.

\scanshdr{markcm} The camera is moving at half the speed at which the light ball is initially
moving. After the collision, it keeps on moving at the same speed --- your five x's all line
on a straight line. Since the camera moves in a straight line with constant speed, it is
showing an inertial frame of reference.

\scanshdr{thirdframepcons} The table looks like this:

\velocitytable{$-1$}{0}{$+1$}{0}{$-1$}{$-1$}{darkball}{lightball}

\noindent Observers in all three frames agree on the changes in velocity, even though they disagree
on the velocities themselves.

\scanshdr{heavierball} The motion would be the same. The force on the ball would be 20 newtons,
so with each second it would gain 20 units of momentum. But 20 units of momentum for a 2-kilogram
ball is still just 10 m/s of velocity.
